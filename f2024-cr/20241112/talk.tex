\date{}
\title{}
\date{}
\begin{document}
\begin{frame}
    \titlepage
\end{frame}

\begin{frame}{last time}
\end{frame}


\begin{frame}{NAT illusion}
    \begin{itemize}
    \item NAT illusion:
    \item private IP address communicating directly with public IP
    \vspace{.5cm}
    \item inside network, talking to outside:
        \begin{itemize}
        \item use private local address
        \item use public remote address
        \item never see router's address
        \end{itemize}
    \item outside network, talking to inside
        \begin{itemize}
        \item use public local address
        \item use router's public address
        \end{itemize}
    \end{itemize}
\end{frame}

\begin{frame}{quiz Q1}
    \begin{itemize}
    \item forwarded packet needs to use \textit{unaltered source address+port}:
    \item local machine should think it's talking to public remote IP
        \begin{itemize}
        \item because that what it's connnecting to
        \end{itemize}
    \item local machine will use source address to find correct socket
        \begin{itemize}
        \item adjusting source address to router will confuse this process
        \end{itemize}
    \vspace{.5cm}
    \item forwarded packet needs to \textit{use private IP+port as destination}:
    \item local machine original made connection with this address
    \item local machine needs to match private IP+port versus its socket
    \end{itemize}
\end{frame}

\begin{frame}{on IP source addresses}
    \begin{itemize}
    \item if IP packet sent from A to D via routers B and C\ldots
    \item routers do not change addresses in the packet!
    \item include source + destination
    \vspace{.5cm}
    \item otherwise, hard for D to know who they're talking to
        \begin{itemize}
        \item example: to send acknowledgments back
        \end{itemize}
    \end{itemize}
\end{frame}

\section{tools without shared keys}

\subsection{asymmetric encryption}

\begin{frame}{asymmetric encryption}
\begin{itemize}
\item we'll have two functions:
    \begin{itemize}
    \item encrypt: $E(\text{public key}, \text{message}) = \text{ciphertext}$
    \item decrypt: $D(\text{private key}, \text{ciphertext}) = \text{message}$
    \end{itemize}
\item (public key, private key) = ``key pair''
    \begin{itemize}
    \item generated together
    \item public key can be shared with everyone
    \item knowing $E$, $D$, the public key, and ciphertext shouldn't make it too easy to find message \\
        (or the private key)
    \end{itemize}
\end{itemize}
\end{frame}

\begin{frame}{public keys}
    \begin{itemize}
    \item public key used to encrypt
    \item can share this with everyone!
    \item conceptually: just post where everyone can see it\ldots \\
        and attacker can't replace/change it
        \begin{itemize}
        \item enough versus passive attackers w/o sharing info advance
        \item not enough versus active attackers
        \end{itemize}
    \vspace{.5cm}
    \item private key used to decrypt
    \item kept secret
    \item don't even share with people sending us messages
    \end{itemize}
\end{frame}


\subsection{digital signatures}

\begin{frame}{digital signatures}

symmetric encryption : asymetric encryption :: \\
message authentication codes : digital signatures
\end{frame}

\begin{frame}{digital signatures}
    \begin{itemize}
    \item pair of functions:
        \begin{itemize}
        \item sign: $S(\text{private key}, \text{message}) = \text{signature}$
        \item verify: $V(\text{public key}, \text{signature}, \text{message}) = 1$ (``yes, correct signature'') 
        \end{itemize}
    \item (public key, private key) = key pair (similar to asymmetric encryption)
        \begin{itemize}
        \item public key can be shared with everyone
        \item knowing $S$, $V$, $\text{public key}$, $\text{message}$, $\text{signature}$ \\
            doesn't make it too easy to find another message + signature so that\\
            $V(\text{public key}, \text{other message}, \text{other signature}) = 1$
        \end{itemize}
    \end{itemize}
\end{frame}

\begin{frame}{using?}
    \begin{itemize}
    \item in advance: A generates private key + public key
    \item in advance: A sends public key to B (and maybe others) securely
    \vspace{.5cm}
    \item A computes $S$(private key, `Please pay ...') = *******
    \item send on network: \\
    A $\rightarrow$ B: `Please pay ...', ********
    \item B computes $V$(public key, `Please pay ...', *******) = 1
    \end{itemize}
\end{frame}


\section{encryption + authentication pitfalls}

\begin{frame}{tools, but...}
    \begin{itemize}
    \item have building blocks, but less than straightforward to use
    \vspace{.5cm}
    \item lots of issues from using building blocks poorly
    \item start of art solution: formal proof sytems
    \end{itemize}
\end{frame}


\subsection{replay attacks}

\begin{frame}{replay attacks}
    \begin{itemize}
    \item A$\rightarrow$B: Did you order lunch? [signed by A]
    \item B$\rightarrow$A: Yes. [signed by B]
    \item A$\rightarrow$B: Vegetarian? [signed by A]
    \item B$\rightarrow$A: No, not this time. [signed by B]
    \item \ldots
    \item A$\rightarrow$B: There's a guy at the door, says he's here to repair the TV. Should I let him in? [signed by A]
    \item how can attacker hijack the reponse to A's inquiry?
    \vspace{.5cm}
    \item<2-> as an attacker, I can copy/paste B's earlier message!
        \begin{itemize}
        \item it's still signed!
        \end{itemize}
    \end{itemize}
\end{frame}

\begin{frame}{nonces}
    \begin{itemize}
    \item one solution to replay attacks:
    \item A$\rightarrow$B: \myemph{\#1} Did you order lunch? [signed by A]
    \item B$\rightarrow$A: \myemph{\#1} Yes. [signed by B]
    \item A$\rightarrow$B: \myemph{\#2} Vegetarian? [signed by A]
    \item B$\rightarrow$A: \myemph{\#2} No, not this time. [signed by B]
    \item \ldots
    \item A$\rightarrow$B: \myemph{\#54} There's a guy at the door, says he's here to repair the TV. Should I let him in? [signed by A]
    \vspace{.5cm}
    \item (assuming A actually checks the numbers)
    \end{itemize}
\end{frame}



\subsection{other attacks}

\begin{frame}{other attacks without breaking math}
\end{frame}

\begin{frame}{TLS state machine attack}
    \begin{itemize}
    \item from \url{https://mitls.org/pages/attacks/SMACK}
    \item protocol:
        \begin{itemize}
        \item step 1: verify server identity
        \item step 2: receive messages from server
        \end{itemize}
    \item attack:
        \begin{itemize}
        \item if server sends ``here's your next message'', \\
            instead of ``here's my identity'' \\
        \item then broken client ignores verifying server's identity
        \end{itemize}
    \end{itemize}
\end{frame}

\begin{frame}{Matrix vulnerabilties}
    \begin{itemize}
    \item one example from \url{https://nebuchadnezzar-megolm.github.io/static/paper.pdf}
    \item system for confidential multi-user chat
    \vspace{.5cm}
    \item protocol + goals:
        \begin{itemize}
        \item each device (my phone, my desktop) has public key
        \item to talk to me, you verify one of my public keys
        \item to add devices, my client can forward my other devices' public keys
        \end{itemize}
    \item bug:
        \begin{itemize}
        \item when receiving new keys, clients did not check who they were forwarded from correctly
        \end{itemize}
    \end{itemize}
\end{frame}


\section{on the lab}
\begin{frame}{on the lab}
\end{frame}

\section{certificate authorities}

\begin{frame}{getting public keys?}
    \begin{itemize}
    \item browser talking to websites \\
    needs public keys of every single website?
    \vspace{.5cm}
    \item not really feasible, but\ldots
    \end{itemize}
\end{frame}

\begin{frame}{certificate idea}
    \begin{itemize}
        \item let's say A has B's public key already.
        \item if C wants B's public key and knows A's already:
            \vspace{.5cm}
        \item \myemph<2>{A can generate ``certificate'' for B}:
            \begin{itemize}
            \item ``B's public key is XXX'' AND
            \item Sign(A's private key, ``B's public key is XXX'')
            \end{itemize}
        \item \myemph<3>{B send copy of their ``certificate'' to C} (most common idea)
        \item if C trusts A, now C has B's public key
            \begin{itemize}
            \item if C does not trust A, well, can't trust this either
            \end{itemize}
    \end{itemize}
\end{frame}

\begin{frame}{certificate authorities}
    \begin{itemize}
    \item websites (and others) go to \textit{certificates authorities} with their public key
    \item certificate authorities sign messages like: \\
        ``The public key for foo.com is XXX.''
    \item signed message called \textit{certificate}
    \item send certificates to browsers to verify identity
    \end{itemize}
\end{frame}

\begin{frame}[fragile]{example web certificate (1)}
\begin{Verbatim}[fontsize=\scriptsize]
    Version: 3 (0x2)
    Serial Number: 7b:df:f6:ae:2e:d7:db:74:d3:c5:77:ac:bc:44:bf:1b
    Signature Algorithm: sha256WithRSAEncryption
    Issuer:
        countryName               = US
        stateOrProvinceName       = MI
        localityName              = Ann Arbor
        organizationName          = Internet2
        organizationalUnitName    = InCommon
        commonName                = InCommon RSA Server CA
    Validity
        Not Before: Apr 25 00:00:00 2023 GMT
        Not After : Apr 24 23:59:59 2024 GMT
    Subject:
        countryName               = US
        stateOrProvinceName       = Virginia
        organizationName          = University of Virginia
        commonName                = canvas.its.virginia.edu
....
    X509v3 extensions:
....
        X509v3 Subject Alternative Name: DNS:canvas.its.virginia.edu
\end{Verbatim}
\end{frame}

\begin{frame}[fragile]{example web certificate (2)}
\begin{Verbatim}[fontsize=\scriptsize]
....
    Subject Public Key Info:
        Public Key Algorithm: rsaEncryption
            RSA Public-Key: (2048 bit)
            Modulus:
                00:a2:fb:5a:fb:2d:d2:a7:75:7e:eb:f4:e4:d4:6c:
                94:be:91:a8:6a:21:43:b2:d5:9a:48:b0:64:d9:f7:
                f1:88:fa:50:cf:d0:f3:3d:8b:cc:95:f6:46:4b:42:

....
Signature Algorithm: sha256WithRSAEncryption
Signature Value:
    24:3a:67:c8:0d:ef:eb:8c:eb:ba:8f:d5:11:d2:1e:ea:44:eb:
    fe:af:93:7d:d9:4a:2b:44:a3:7f:47:50:aa:d1:b3:9c:a8:a8:
....
\end{Verbatim}
\end{frame}

\begin{frame}{certificate chains}
    \begin{itemize}
    \item That certificate signed by ``InCommon RSA Server CA''
    \item CA = certificate authority
    \vspace{.5cm}
    \item so their public key, comes with my OS/browser?
        \begin{itemize}
        \item not exactly\ldots
        \end{itemize}
    \item they have their own certificate signed by ``USERTrust RSA Certification Authority''
    \item and their public key comes with your OS/browser?
    \vspace{.5cm}
    \item (but both CAs now operated by UK-based Sectigo)
    \end{itemize}
\end{frame}
\usetikzlibrary{fit}
\begin{frame}{certificate hierarchy}
\begin{tikzpicture}
    \tikzset{
        cert/.style={draw, very thick,align=left,font=\small},
        root/.style={alt=<2>{fill=blue!10,line width=3pt}},
        root weak/.style={alt=<2>{fill=blue!10,line width=3pt,dashed}},
    }
    \node[cert,root] (usertrust) at (-3, 0) {
        USERTrust RSA \\ Certification Authority \\
        \parbox{2.25in}{
        \scriptsize originally operated by USERTrust, Inc.  \\
        \scriptsize acquired by Comodo, Inc (2004) \\
        \scriptsize Comodo's CA division renamed Sectigo (2018)
        }
    };
    \node[cert] (incommon) at (-4, -3) {
        InCommon \\ RSA Server CA \\
        \parbox{2in}{
        \scriptsize operated by Sectigo \\
        \scriptsize on behalf of the Internet2 (not-for-profit)
        }
    };
    \node[cert] (collab) at (-4, -5) {
        collab.its.virginia.edu
    };
    \node[overlay] (other incommon) at (-7, -5) {
        \ldots
    };
    \node[overlay] (other incommon 2) at (-1, -5) {
        \ldots
    };
    \node (other usertrust) at (0, -3) {
        \ldots
    };
    \begin{scope}[very thick,->]
        \draw (usertrust) -- (incommon);
        \draw (incommon) -- (collab);
        \draw[overlay] (incommon) -- (other incommon);
        \draw[overlay] (incommon) -- (other incommon 2);
        \draw (usertrust) -- (other usertrust);
    \end{scope}

    \begin{scope}[xshift=5cm]
        \node[cert,root] (globalsign) at (0, 0) {
            GlobalSign Root CA \\
            \parbox{2.25in}{
                \scriptsize
                operated by GlobalSign nv-sa \\
                subsid. of GMO Internet Group since 2007
            }
        };
        \node (globalsign other) at (3, -2) {\ldots};
        \node [cert,root weak] (goog root ca) at (-1, -2) {
            GTS Root R1 \\
            \parbox{2.25in}{
                \scriptsize
                operated by Google Trust Services LLC
            }
        };
        \node [cert] (goog subca) at (-1, -3.5) {
            GTS CA 1C3
        };
        \node (goog subother) at (2, -3.5) {\ldots};
        \node [cert] (goog) at (-1, -5) {
            www.google.com
        };
        \node (goog other) at (-4, -5) {\ldots};
        \begin{scope}[very thick,->]
            \draw (globalsign) -- (goog root ca);
            \draw (globalsign) -- (globalsign other);
            \draw (goog root ca) -- (goog subca);
            \draw (goog root ca) -- (goog subother);
            \draw (goog subca) -- (goog);
            \draw (goog subca) -- (goog other);
        \end{scope}
    \end{scope}

    \begin{visibleenv}<2>
        \begin{pgfonlayer}{fg}
        \begin{visibleenv}<2>
        \node[draw=yellow!50!black, fill=white, align=center,line width=3pt,fill=yellow!20] (over box) at (0, -5) {
            some ``trust anchors'' included with browsers and OSes \\
            (for GTS Root R1, only more recent browsers/OSes)
        };
        \end{visibleenv}
        \end{pgfonlayer}
        \node[fit=(over box),inner sep=.5cm,fill=white,fill opacity=0.5,overlay] {};
    \end{visibleenv}
\end{tikzpicture}
\end{frame}

\begin{frame}{how many trust anchors?}
    \begin{itemize}
    \item Mozilla Firefox (as of 27 Feb 2023)
        \begin{itemize}
            \item 155 trust anchors
            \item operated by 55 distinct entities
        \end{itemize}
    \item Microsoft Windows (as of 27 Feb 2023)
        \begin{itemize}
            \item 237 trust anchors
            \item operated by 86 distinct entities
        \end{itemize}
    \end{itemize}
\end{frame}

\begin{frame}{public-key infrastructure}
    \begin{itemize}
    \item ecosystem with certificate authorities \\
        and certificates for everyone
    \item called ``public-key infrastructure''
        \vspace{.5cm}
    \item several of these:
        \begin{itemize}
        \item for verifying identity of websites 
        \item for verifying origin of domain name records (kind-of)
        \item for verifying origin of applications in some OSes/app stores/etc.
        \item for encrypted email in some organizations
        \item \ldots
        \end{itemize}
    \end{itemize}
\end{frame}


\subsection{how certificates verified}
\begin{frame}{exercise}
    \begin{itemize}
    \item exercise: how should website certificates verify identity?
    \end{itemize}
\end{frame}

\begin{frame}{how do certificate authorities verify}
    \begin{itemize}
        \item for web sites, set by CA/Browser Forum
        \item organization of:
            \begin{itemize}
            \item everyone who ships code with list of valid certificate authorities
                \begin{itemize}
                \item Apple, Google, Microsoft, Mozilla, Opera, Cisco, Qihoo 360, Brave, \ldots
                \end{itemize}
            \item certificate authorities
            \end{itemize}
        \item decide on rules (``baseline requirements'') for what CAs do
    \end{itemize}
\end{frame}

\begin{frame}{BR domain name identity validation}
    \begin{itemize}
        \item options involve CA choosing random value and:
        \vspace{.5cm}
        \item sending it to domain contact (with domain registrar) and receive response with it, or
        \item observing it placed in DNS or website or sent from server in other specific way
        \vspace{1cm}
        \item exercise: problems this doesn't deal with?
    \end{itemize}
\end{frame}

\begin{frame}{some other things public CAs do}
    \begin{itemize}
    \item keep their private keys in tamper-resistant hardware
    \item maintain publicly-accessible database of \myemph<2>{\textit{revoked} certificates}
        \begin{itemize}
        \item some browsers check these, sometimes
        \end{itemize}
    \item \myemph<3>{certificate transparency}
        \begin{itemize}
            \item \myemph<3>{public logs} of every certificate issued
            \item some browsers reject non-logged certificates
            \item so you can tell if bad certificate exists for your website
        \end{itemize}
    \item `CAA' records in the domain name system
        \begin{itemize}
            \item can indicate \myemph<4>{which CAs are allowed to issue certificates in DNS}
        \item (but CAs apparently not required to use DNSSEC (certificate infrastructure for signing domain name records) when looking this up)
        \end{itemize}
    \end{itemize}
\end{frame}


\section{preview: additional tools}
\begin{frame}{additional crypto tools}
    \begin{itemize}
    \item cryptographic hash functions (summarize data)
    \item `secure' random numbers
    \item key agreement
    \end{itemize}
\end{frame}

\section{cryptographic hashes}

% FIXME: hashes
\begin{frame}{motivation: summary for signature}
    \begin{itemize}
    \item digital signatures typically have size limit
    \item \ldots but we want to sign very large messages
    \vspace{.5cm}
    \item solution: get secure ``summary'' of message
    \end{itemize}
\end{frame}

\begin{frame}{cryptographic hash}
    \begin{itemize}
    \item hash(M) = X
    \vspace{.5cm}
    \item given X:
        \begin{itemize}
        \item hard to find message other than by guessing
        \end{itemize}
    \item given X, M:
        \begin{itemize}
        \item hard to find second message so that hash(second message) = X
        \end{itemize}
    \vspace{.5cm}
    \item example uses:
        \begin{itemize}
        \item substitute for original message in digital signature
        \item building message authentication codes
        \end{itemize}
    \end{itemize}
\end{frame}



\subsection{password hashing}

\begin{frame}{password hashing}
    \begin{itemize}
        \item cryptographic hash functions are good at requiring guesses to `reverse'
            \vspace{.5cm}
        \item problem: guessing passwords is very fast
        \item solution: slow/resource-intensive cryptographic hash functions
            \begin{itemize}
            \item Argon2i
            \item scrypt
            \item PBKDF2
            \end{itemize}
    \end{itemize}
\end{frame}


\section{random numbers}
\begin{frame}{random numbers}
    \begin{itemize}
    \item need a lot of keys that no one else knows
    \vspace{.5cm}
    \item common task: choose a \textit{random} number
    \item question: what does \textit{random} mean here?
    \end{itemize}
\end{frame}

\begin{frame}{cryptographically secure random numbers}
    \begin{itemize}
        \item security properties we might want for random numbers:
        \vspace{.5cm}
    \item attacker cannot guess (part of) number better than chance
    \item knowing prior `random' numbers shouldn't help predict next `random' numbers
    \item compromising machine now shouldn't reveal older random numbers
    \end{itemize}
\end{frame}

\begin{frame}{exercise: how to generate?}
\end{frame}

\begin{frame}{/dev/urandom}
    \begin{itemize}
    \item Linux kernel random number generator
    \vspace{.5cm}
    \item collects ``entropy'' from hard-to-predict events
        \begin{itemize}
        \item e.g. exact timing of I/O interrupts
        \item e.g. some processor's built-in random number circuit
        \end{itemize}
    \item turned into as many random bytes as you want
    \end{itemize}
\end{frame}

% \begin{frame}{turning `entropy' into random bytes}
%     \begin{itemize}
%     \item lots of ways to do this; one (rough/incomplete) idea:
%     \item internal variable \textit{state}
%     \item to add `entropy'
%         \begin{itemize}
%         \item state $\leftarrow$ SecureHash(state + entropy)
%         \end{itemize}
%     \item to extract value:
%         \begin{itemize}
%         \item random bytes $\leftarrow$ SecureHash(1 + state) \\
%             \small give bytes that can't be reversed to compute state
%                 \vspace{.5cm}
%         \item state $\leftarrow$ SecureHash(2 + state) \\
%             \small change state so attacker can't take us back to old state if compromised
%         \end{itemize}
%     \end{itemize}
% \end{frame}


\section{key agreement}

\begin{frame}{just asymmetric?}
    \begin{itemize}
    \item given public-key encryption + digital signatures\ldots
    \item why bother with the symmetric stuff?
    \vspace{.5cm}
    \item symmetric stuff much faster
    \item symmetric stuff much better at supporting larger messages
    \end{itemize}
\end{frame}

\begin{frame}{key agreement}
    \begin{itemize}
    \item problem: A has B's public encryption key \\
        wants to choose shared secret 
    \vspace{.5cm}
    \item some ideas:
        \begin{itemize}
        \item A chooses a key, sends it encrypted to B
        \item A sends a public key encrypted B, B chooses a key and sends it back
        \end{itemize}
    \item<2-> alternate model:
        \begin{itemize}
        \item both sides generate random values
        \item derive public-key like ``key shares'' from values
        \item use math to combine ``key shares''
        \item kinda like A + B both sending each other public encryption keys
        \end{itemize}
    \end{itemize}
\end{frame}

\begin{frame}{Diffie-Hellman key agreement (2)}
\begin{itemize}
\item A and B want to agree on shared secret
\vspace{.5cm}
\item A chooses random value Y
\item A sends public value derived from Y (``key share'')
\item B chooses random value Z
\item B sends public value derived from Z (``key share'')
\item A combines Y with public value from B to get number
\item B combines Z with public value from B to get number
    \begin{itemize}
    \item and b/c of math chosen, both get same number
    \end{itemize}
\end{itemize}
\end{frame}

\begin{frame}{Diffie-Hellman key agreement (1)}
    \begin{itemize}
    \item math requirement:
        \begin{itemize}
        \item some $f$, so $f(f(X, Y), Z) = f(f(X, Z), Y)$
        \item (that's hard to invert, etc.)
        \end{itemize}
    choose X in advance and:
    \end{itemize}
\begin{tabular}{l|l}
A randomly chooses $Y$ & B randomly chooses $Z$ \\
A sends $f(X, Y)$ to B & B sends $f(X, Z)$ to A \\
A computes $f(f(X, Z), Y)$ & B computes $f(f(X, Y), Z)$ \\
\end{tabular}
\end{frame}



\subsection{aside: key agreement to public key encrypt}
\begin{frame}{key agreement and asym. encryption}
    \begin{itemize}
    \item can construct public-key encryption from key agreeement
    \vspace{.5cm}
    \item private key: generated random value Y
    \item public key: key share generated from that Y
    \item<2-> PE(public key, message) =
        \begin{itemize}
        \item generate random value Z
        \item combine with public key to get shared secret
        \item use symmetric encryption + MAC using shared secret as keys
        \item output: (key share generated from Z) (sym. encrypted data) (mac tag)
        \end{itemize}
    \item<3-> PD(private key, message) =
        \begin{itemize}
        \item extract (key share generated from Z)
        \item combine with private key to get shared secret, \ldots
        \end{itemize}
    \end{itemize}
\end{frame}


\section{putting it together: TLS}

\subsection{handshake}
\usetikzlibrary{arrows.meta,shapes.callouts,positioning}

\begin{frame}{typical TLS handshake}
\begin{tikzpicture}
    \tikzset{
        >=Latex,
        comp box/.style={draw, thick, align=center, minimum width=1.5cm,minimum height=6.25cm},
        explain box/.style={draw=red,very thick, align=left},
        msg/.style={font=\small},
        cmd/.style={font=\small},
    }
        \node[comp box] (client) at (-6.5, 0) {client};
        \node[draw,cloud,line width=1pt,minimum width=4cm,minimum height=2cm,aspect=3,opacity=0.25] (network) at (0,0) {~~};
        \node[comp box] (server) at (6.5, 0) {server};
        \draw[very thick,->] ([yshift=-.5cm]client.north east) -- ([yshift=-.5cm]server.north west) 
            node[midway,below,msg] (client key share) {ClientHello,KeyShare};
        \draw[very thick,<-] ([yshift=-1.5cm]client.north east) -- ([yshift=-1.5cm]server.north west) 
            node[midway,below,msg] (server key share) {ServerHello,KeyShare};
        \draw[very thick,<-] ([yshift=-2.5cm]client.north east) -- ([yshift=-2.5cm]server.north west) 
            node[midway,below,msg] (certificate) {Certificate,CertificateVerify};
        \draw[very thick,<-] ([yshift=-3.5cm]client.north east) -- ([yshift=-3.5cm]server.north west) 
            node[midway,below,msg] (finished1) {Finished};
        \draw[very thick,->] ([yshift=-4.5cm]client.north east) -- ([yshift=-4.5cm]server.north west) 
            node[midway,below,msg] (finished2) {Finished};
    \begin{visibleenv}<2>
        \node[my callout2=client key share,anchor=north] at ([yshift=-1cm]client key share) {
            KeyShare = key parts for key exchange
        };
    \end{visibleenv}
    \begin{visibleenv}<3>
        \node[my callout2=certificate,anchor=north,align=left] at ([yshift=-1cm]certificate) {
            Certificate = certificate (``foo.com's public key is X'' + CA signature) \\
            \myemph<3>{CertificateVerify = Sign(foo.com's private key, server's key share)}
        };
    \end{visibleenv}
    \begin{visibleenv}<4-5>
        \node[my callout2=finished1,anchor=north,align=left] at ([yshift=-1cm]finished1) {
            MAC(\myemph<4>{key made from key shares}, Hash(everything so far))
        };
    \end{visibleenv}
    \begin{visibleenv}<6>
        \node[my callout2=finished2,anchor=north,align=left] at ([yshift=-1cm]finished2) {
            MAC(\myemph<4>{key made from key shares}, Hash(everything so far))
        };
    \end{visibleenv}
\end{tikzpicture}
\end{frame}


\subsection{after handshake}
\begin{frame}{TLS: after handshake}
    \begin{itemize}
    \item use key shares results to get \textbf{several} keys
        \begin{itemize}
        \item take hash(something + shared secret) to derive each key
        \end{itemize}
    \item separate keys for each direction (server $\rightarrow$ client and vice-versa)
    \item often separate keys for encryption and MAC
    \vspace{.5cm}
    \item later messages use encryption + MAC + nonces
    \end{itemize}
\end{frame}


\subsection{TLS properties}
\begin{frame}{things modern TLS usually does}
    \begin{itemize}
        \item (not all these properties provided by all TLS versions and modes)
    \vspace{.5cm}
    \item confidentiality/authenticity 
        \begin{itemize}
        \item server = one ID'd by certificate
        \item client = same throughout whole connection
        \end{itemize}
    \item forward secrecy
        \begin{itemize}
        \item can't decrypt old conversations (data for KeyShares is temporary)
        \end{itemize}
    \item fast
        \begin{itemize}
        \item most communication done with more efficient symmetric ciphers
        \item 1 set of messages back and forth to setup connection
        \end{itemize}
    \end{itemize}
\end{frame}

\subsection{summary}
\begin{frame}{network security summary (1)}
    \begin{itemize}
    \item communicating securely with math
        \begin{itemize}
        \item secret value (shared key, public key) that attacker can't have
        \item symmetric: shared keys used for (de)encryption + auth/verify; fast
        \item asymmetric: public key used by any for encrypt + verify; slower
        \item asymmetric: private key used by holder for decrypt + sign; slower
        \end{itemize}
    \item protocol attacks --- repurposing encrypt/signed/etc. messages
    \item certificates --- verifiable forwarded public keys 
    \item key agreement --- for generated shared-secret ``in public''
        \begin{itemize}
        \item publish key shares from private data
        \item combine private data with key share for shared secret
        \end{itemize}
    \end{itemize}
\end{frame}

\begin{frame}{network security summary (2)}
    \begin{itemize}
    \item TLS: combine all cryptography stuff to make ``secure channel''
    \vspace{.5cm}
    \item (things we probably didn't get to:)
    \item denial-of-service --- attacker just disrupts/overloads (not subtle)
    \item firewalls
    \end{itemize}
\end{frame}




\section{backup slides}
\begin{frame}{backup slides}
\end{frame}

\section{misc. security issues}

\subsection{denial of service}
\begin{frame}{denial of service}
    \begin{itemize}
    \item if you just want to inconvenience\ldots
    \item attacker just sends lots of stuff to my server
    \item my server becomes overloaded?
    \item my network becomes overloaded?
    \vspace{.5cm}
    \item but: doesn't this require a lot of work for attacker?
    \item exercise: why is this often not a big obstacle
    \end{itemize}
\end{frame}

\begin{frame}{denial of service: asymmetry}
    \begin{itemize}
    \item work for attacker > work for defender
    \item how much computation per message?
        \begin{itemize}
        \item complex search query?
        \item something that needs tons of memory?
        \item something that needs to read tons from disk?
        \end{itemize}
    \item how much sent back per message?
    \vspace{.5cm}
    \item resources for attacker > resources of defender
    \item how many machines can attacker use?
    \end{itemize}
\end{frame}


    % FIXME: example of small request w/ big response

\subsubsection{amplification example}
% FIXME: https://blog.cloudflare.com/technical-details-behind-a-400gbps-ntp-amplification-ddos-attack/

\subsection{firewalls} % FIXME: complete
\begin{frame}{firewalls}
    \begin{itemize}
    \item don't want to expose network service to everyone?
    \item solutions:
        \begin{itemize}
        \item service picky about who it accepts connections from
        \item filters in OS on machine with services
        \item filters on router
        \end{itemize}
    \item later two called ``firewalls''
    \end{itemize}
\end{frame}

\begin{frame}{firewall rules examples?}
    \begin{itemize}
    \item ALLOW tcp port 443 (https) FROM everyone
    \item ALLOW tcp port 22 (ssh) FROM \myemph{my desktop's IP address}
    \item BLOCK tcp port 22 (ssh) FROM everyone else
    \item ALLOW from address X to address Y
    \item \ldots
    \end{itemize}
\end{frame}


% FIXME: update readings to mention IPv6 equivalent



\end{document}
