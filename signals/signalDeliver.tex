\usetikzlibrary{arrows.meta,matrix}
\begin{frame}{x86-64 Linux signal delivery (1)}
\begin{itemize}
\item suppose: signal happens while {\tt foo()} is running
\item OS saves registers \myemph{to user stack}
\item OS modifies user registers, PC to call signal handler
\end{itemize}
\begin{tikzpicture}
\tikzset{every node/.style={font=\small}}
\matrix[tight matrix,nodes={text width=6cm},label={the stack}] (stack) {
    address of {\tt \_\_restore\_rt} \\
    saved registers \\
    PC when signal happened \\
    local variables for foo \\
    \ldots{} \\
};
\draw[thick,-Latex] (stack-4-1.east) -- ++(.5cm,0cm) node[right,align=left] {stack pointer \\ before signal delivered};
\draw[thick,-Latex] (stack-1-1.east) -- ++(.5cm,0cm) node[right,align=left] {stack pointer \\ when signal handler started};
\end{tikzpicture}
\end{frame}

\begin{frame}[fragile,label=sigReturn]{x86-64 Linux signal delivery (2)}
\lstset{
    language=myasm,
    style=small,
    morekeywords={ret,syscall,movq}
}
\begin{lstlisting}
handle_sigint:
    ...
    ret
...
__restore_rt:
    // 15 = "sigreturn" system call
    movq $15, %rax
    syscall
\end{lstlisting}
\begin{itemize}
\item {\tt \_\_restore\_rt} is \myemph{return address} for signal handler
\item sigreturn syscall restores pre-signal state
\begin{itemize}
    \item needed to handle caller-saved registers
    \item also might unblock signals (like un-disabling interrupts)
\end{itemize}
\end{itemize}
\end{frame}


