\begin{frame}{exercise (1)}
    \begin{itemize}
    \item initial cache: 64-byte blocks, 64 sets, 8 ways/set
    \vspace{.5cm}
\item If we leave the other parameters listed above unchanged, which will probably reduce
    the number of \myemph{capacity misses} in a typical program? 
    (Multiple may be correct.) \\
    \begin{tabular}{ll}
        A. & quadrupling  the block size {\small (256-byte blocks, 64 sets, 8 ways/set)}\\
        B. & quadrupling the number of sets \\
        C. & quadrupling the number of ways/set\\
    \end{tabular}
    \end{itemize}
\end{frame}

\begin{frame}{exercise (2)}
    \begin{itemize}
    \item initial cache: 64-byte blocks, 8 ways/set, 64KB cache
    \vspace{.5cm}
    \item If we leave the other parameters listed above unchanged, which will probably reduce
          the number of \myemph{capacity misses} in a typical program? (Multiple may be correct.) \\
    \begin{tabular}{ll}
        A. & quadrupling the block size {\small (256-byte block, 8 ways/set, 64KB cache)}\\
        B. & quadrupling the number of ways/set \\
        C. & quadrupling the cache size \\
    \end{tabular}
    \end{itemize}
\end{frame}

\begin{frame}{exercise (3)}
    \begin{itemize}
    \item initial cache: 64-byte blocks, 8 ways/set, 64KB cache
    \vspace{.5cm}
    \item If we leave the other parameters listed above unchanged, which will probably reduce
          the number of \myemph{conflict misses} in a typical program? (Multiple may be correct.) \\
    \begin{tabular}{ll}
        A. & quadrupling the block size {\small (256-byte block, 8 ways/set, 64KB cache)}\\
        B. & quadrupling the number of ways/set \\
        C. & quadrupling the cache size \\
    \end{tabular}
    \end{itemize}
\end{frame}
