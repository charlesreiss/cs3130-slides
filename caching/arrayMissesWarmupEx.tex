\usetikzlibrary{decorations.pathreplacing,matrix}

\begin{frame}[fragile,label=arrayMissesWarmup1]{C and cache misses (warmup 1)}
\begin{lstlisting}
int array[4];
...
int even_sum = 0, odd_sum = 0;
even_sum += array[0];
odd_sum += array[1];
even_sum += array[2];
odd_sum += array[3];
\end{lstlisting}
\begin{itemize}
\item {\small
Assume everything but {\tt array} is kept in registers (and the compiler does not do
anything funny).}
\item How many data cache misses on a 1-set direct-mapped cache with 8B blocks?
\end{itemize}
\end{frame}

\begin{frame}<1>[fragile,label=arrayMissesWarmup1Answers]{some possiblities}
    \newcommand{\mywidth}{0.39}
\begin{tikzpicture}
    % FIXME: picture
        % array versus cache blocks layout
        % show each each block maps to array, with cache blocks lining up properly
    \foreach \offset in {-2,-1,0,1,2,...,31,32,33} {
        \draw[black!25,thin] (\offset * \mywidth, 0) rectangle ++(\mywidth, -1);
    }
    \node[font=\large,anchor=east] at (-2 * \mywidth, -.5) {\ldots};
    \node[font=\large,anchor=west] at (34 * \mywidth, -.5) {\ldots};
    \foreach \offset/\name in {8/0,12/1,16/2,20/3} {
        \draw[thin,fill=blue!10] (\offset * \mywidth, 0) rectangle ++(\mywidth * 4, -1)
            node[fill=none,opacity=1.0,black,midway,font=\fontsize{9}{10}\tt\selectfont] {array[\name]};
    }
    \begin{visibleenv}<1>
    \node[anchor=north west,align=left] at (0, -1.25) {
        Q1: how do cache blocks correspond to array elements? \\
        not enough information provided!
    };
    \end{visibleenv}
    \begin{visibleenv}<2>
    \node[anchor=north west,align=left] at (0, -1.25) {
        if array[0] starts at beginning of a cache block\ldots \\
        array split across two cache blocks
    };
    \begin{scope}
    \clip (-2.1 * \mywidth, .2) rectangle (34.1 * \mywidth, -1.1);
    \foreach \offset in {-8,0,8,16,24,32} {
        \draw[ultra thick,red!30!black] (\offset * \mywidth, 0)
            rectangle ++(\mywidth * 8, -1);
    }
    \end{scope}
        \draw[very thick, decorate, decoration={brace}] (16 * \mywidth, 0.05) -- ++(8 * \mywidth, 0)
            node[above,midway,overlay] {one cache block};
    \matrix[tight matrix,
        nodes={minimum height=0.7cm},
        column 1/.append style={nodes={text width=6cm}},
        column 2/.append style={nodes={text width=6cm,font=\tt}},
        row 1/.append style={nodes={font=\bfseries}},
        anchor=north west] at (0, -3) {
        memory access \& cache contents afterwards \\ 
        --- \& (empty) \\
        read \lstinline|array[0]| (miss) \& \{array[0], array[1]\} \\
        read \lstinline|array[1]| (hit) \& \{array[0], array[1]\} \\
        read \lstinline|array[2]| (miss) \& \{array[2], array[3]\} \\
        read \lstinline|array[3]| (hit) \& \{array[2], array[3]\} \\
    };
    \end{visibleenv}
    \begin{visibleenv}<3>
    \node[anchor=north west,align=left] at (0, -1.25) {
        if array[0] starts right in the middle of a cache block \\
        array split across three cache blocks
    };
    \begin{scope}
    \clip (-2.1 * \mywidth, .2) rectangle (34.1 * \mywidth, -1.1);
    \foreach \offset in {-4,4,12,20,28} {
        \draw[ultra thick,red!30!black] (\offset * \mywidth, 0)
            rectangle ++(\mywidth * 8, -1);
    }
    \end{scope}
        \draw[very thick, decorate, decoration={brace}] (12 * \mywidth, 0.05) -- ++(8 * \mywidth, 0)
            node[above,midway,overlay] {one cache block};
    \foreach \offset/\name in {4/****,24/++++} {
        \path (\offset * \mywidth, 0) rectangle ++(\mywidth * 4, -1)
            node[fill=none,opacity=1.0,black,midway,font=\fontsize{9}{10}\tt\selectfont] {\name};
    }
    \matrix[tight matrix,
        nodes={minimum height=0.7cm},
        column 1/.append style={nodes={text width=6cm}},
        column 2/.append style={nodes={text width=6cm,font=\tt}},
        row 1/.append style={nodes={font=\bfseries}},
        anchor=north west] at (0, -3) {
        memory access \& cache contents afterwards \\ 
        --- \& (empty) \\
        read \lstinline|array[0]| (miss) \& \{****, array[0]\} \\
        read \lstinline|array[1]| (miss) \& \{array[1], array[2]\} \\
        read \lstinline|array[2]| (hit) \& \{array[1], array[2]\} \\
        read \lstinline|array[3]| (miss) \& \{array[3], ++++\} \\
    };
    \end{visibleenv}
    \begin{visibleenv}<4>
    \node[anchor=north west,align=left] at (0, -1.25) {
        if array[0] starts at an odd place in a cache block, \\
        need to read two cache blocks to get most array elements
    };
    \begin{scope}
    \clip (-2.1 * \mywidth, .2) rectangle (34.1 * \mywidth, -1.1);
    \foreach \offset in {-7,1,9,17,25,33} {
        \draw[ultra thick,red!30!black] (\offset * \mywidth, 0)
            rectangle ++(\mywidth * 8, -1);
    }
    \end{scope}
        \draw[very thick, decorate, decoration={brace}] (9 * \mywidth, 0.05) -- ++(8 * \mywidth, 0)
            node[above,midway,overlay] {one cache block};
    \foreach \offset/\name in {4/****,24/++++} {
        \path (\offset * \mywidth, 0) rectangle ++(\mywidth * 4, -1)
            node[fill=none,opacity=1.0,black,midway,font=\fontsize{9}{10}\tt\selectfont] {\name};
    }
    \matrix[tight matrix,
        nodes={minimum height=0.5cm},
        column 1/.append style={nodes={text width=6cm,font=\small}},
        column 2/.append style={nodes={text width=6cm,align=center,font=\fontsize{8}{9}\selectfont}},
        row 1/.append style={nodes={font=\bfseries}},
        anchor=north west] at (0, -3) {
        memory access \& cache contents afterwards \\ 
        --- \& (empty) \\
        read \lstinline|array[0]| byte 0 (miss) \& \{ ****, array[0] byte 0 \} \\
        read \lstinline|array[0]| byte 1-3 (miss) \& \{ array[0] byte 1-3, array[2], array[3] byte 0 \} \\
        read \lstinline|array[1]| (hit) \&  \{ array[0] byte 1-3, array[2], array[3] byte 0 \} \\
        read \lstinline|array[2]| byte 0 (hit) \& \{ array[0] byte 1-3, array[2], array[3] byte 0 \} \\
        read \lstinline|array[2]| byte 1-3 (miss) \& \{part of array[2], array[3], ++++\} \\
        read \lstinline|array[3]| (hit) \& \{part of array[2], array[3], ++++\} \\
    };
    \end{visibleenv}
\end{tikzpicture}
\end{frame}

\iftoggle{heldback}{}{
\againframe<2>{arrayMissesWarmup1Answers}
\againframe<3>{arrayMissesWarmup1Answers}
\againframe<4>{arrayMissesWarmup1Answers}
}

\begin{frame}{aside: alignment}
    \begin{itemize}
    \item compilers and malloc/new implementations usually try \myemph{align} values
    \item align = make address be multiple of something
    \vspace{.5cm}
    \item most important reason: don't cross cache block boundaries
    \end{itemize}
\end{frame}

\begin{frame}[fragile,label=arrayMissesWarmup2]{C and cache misses (warmup 2)}
\begin{lstlisting}
int array[4];
int even_sum = 0, odd_sum = 0;
even_sum += array[0];
even_sum += array[2];
odd_sum += array[1];
odd_sum += array[3];
\end{lstlisting}
\begin{itemize}
\item {\small
Assume everything but {\tt array} is kept in registers (and the compiler does not do
anything funny).}
\item Assume array[0] at beginning of cache block.
\item How many data cache misses on a 1-set direct-mapped cache with 8B blocks?
\end{itemize}
\end{frame}

\begin{frame}<0>[fragile,label=arrayMissesWarmup2Answers]{exercise solution}
    \newcommand{\mywidth}{0.39}
\begin{tikzpicture}
    \foreach \offset in {-2,-1,0,1,2,...,31,32,33} {
        \draw[black!25,thin] (\offset * \mywidth, 0) rectangle ++(\mywidth, -1);
    }
    \node[font=\large,anchor=east] at (-2 * \mywidth, -.5) {\ldots};
    \node[font=\large,anchor=west] at (34 * \mywidth, -.5) {\ldots};
    \foreach \offset/\name in {8/0,12/1,16/2,20/3} {
        \draw[thin,fill=blue!10] (\offset * \mywidth, 0) rectangle ++(\mywidth * 4, -1)
            node[fill=none,opacity=1.0,black,midway,font=\fontsize{9}{10}\tt\selectfont] {array[\name]};
    }
    \begin{scope}
    \clip (-2.1 * \mywidth, .2) rectangle (34.1 * \mywidth, -1.1);
    \foreach \offset in {-8,0,8,16,24,32} {
        \draw[ultra thick,red!30!black] (\offset * \mywidth, 0)
            rectangle ++(\mywidth * 8, -1);
    }
    \end{scope}
        \draw[very thick, decorate, decoration={brace}] (8 * \mywidth, 0.05) -- ++(8 * \mywidth, 0)
            node[above,midway] {one cache block};
    \matrix[tight matrix,
        nodes={minimum height=0.7cm},
        column 1/.append style={nodes={text width=6cm}},
        column 2/.append style={nodes={text width=6cm,font=\tt}},
        row 1/.append style={nodes={font=\bfseries}},
        anchor=north west] at (0, -3) {
        memory access \& cache contents afterwards \\ 
        --- \& (empty) \\
        read \lstinline|array[0]| (miss) \& \{array[0], array[1]\} \\
        read \lstinline|array[2]| (miss) \& \{array[2], array[3]\} \\
        read \lstinline|array[1]| (miss) \& \{array[0], array[1]\} \\
        read \lstinline|array[3]| (miss) \& \{array[2], array[3]\} \\
    };
\end{tikzpicture}
\end{frame}

\iftoggle{heldback}{
\againframe<1->{arrayMissesWarmup2Answers}
}

\begin{frame}[fragile,label=arrayMissesWarmup3]{C and cache misses (warmup 3)}
\begin{lstlisting}[style=size10]
int array[8];
...
int even_sum = 0, odd_sum = 0;
even_sum += array[0];
odd_sum += array[1];
even_sum += array[2];
odd_sum += array[3];
even_sum += array[4];
odd_sum += array[5];
even_sum += array[6];
odd_sum += array[7];
\end{lstlisting}
\begin{itemize}
\item {\small
Assume everything but {\tt array} is kept in registers (and the compiler does not do
anything funny), and array[0] at beginning of cache block.}
\item How many data cache misses on a \textbf{2}-set direct-mapped cache with 8B blocks?
\end{itemize}
\end{frame}

\begin{frame}<0>[fragile,label=arrayMissesWarmup3Answers]{exercise solution}
\newcommand{\mywidth}{0.39}
\begin{tikzpicture}
    \foreach \offset in {-2,-1,0,1,2,...,31,32,33} {
        \draw[black!25,thin] (\offset * \mywidth, 0) rectangle ++(\mywidth, -1);
    }
    \node[font=\large,anchor=east] at (-2 * \mywidth, -.5) {\ldots};
    \node[font=\large,anchor=west] at (34 * \mywidth, -.5) {\ldots};
    \begin{scope}
    \clip (-2.1 * \mywidth, .2) rectangle (34.1 * \mywidth, -1.1);
    \foreach \offset/\name in {8/0,12/1,16/2,20/3,24/4,28/5,32/6} {
        \draw[thin,fill=blue!10] (\offset * \mywidth, 0) rectangle ++(\mywidth * 4, -1)
            node[fill=none,opacity=1.0,black,midway,font=\fontsize{9}{10}\tt\selectfont] {array[\name]};
    }
    \foreach \offset in {-8,0,8,16,24,32} {
        \draw[ultra thick,red!30!black] (\offset * \mywidth, 0)
            rectangle ++(\mywidth * 8, -1);
    }
    \end{scope}
        \draw[very thick, decorate, decoration={brace}] (8 * \mywidth, 0.05) -- ++(8 * \mywidth, 0)
            node[inner sep=0mm,above=1mm,midway,align=center,font=\small] {one cache block \\
                                (index 0)};
        \begin{visibleenv}<2->
        \draw[very thick, decorate, decoration={brace}] (16 * \mywidth, 0.05) -- ++(8 * \mywidth, 0)
            node[inner sep=0mm,above=1mm,midway,align=center,font=\small] {one cache block \\
                                (index 1)};
        \draw[very thick, decorate, decoration={brace}] (24 * \mywidth, 0.05) -- ++(8 * \mywidth, 0)
            node[inner sep=0mm,above=1mm,midway,align=center,font=\small] {one cache block \\
                                (index 0)};
        \draw[very thick, decorate, decoration={brace}] (0 * \mywidth, 0.05) -- ++(8 * \mywidth, 0)
            node[inner sep=0mm,above=1mm,midway,align=center,font=\small] {one cache block \\
                                (index 1)};
        \end{visibleenv}
    \begin{visibleenv}<3->
    \tikzset{
        set both/.style={alt=<5-7>{nodes={fill=red!10}}},
        set 0/.style={
            alt=<5-6>{nodes={fill=red!10}},
            alt=<7>{opacity=0.1},
        },
        set 1/.style={
            alt=<5-6>{opacity=0.1},
            alt=<7>{nodes={fill=red!10}},
        },
    }
    \matrix[tight matrix,
        nodes={minimum height=0.5cm,font=\fontsize{8}{9}\selectfont},
        column 1/.append style={nodes={text width=4cm}},
        column 2/.append style={nodes={text width=5cm,font=\fontsize{10}{11}\tt\selectfont,set 0}},
        column 3/.append style={nodes={text width=5cm,font=\fontsize{10}{11}\tt\selectfont,set 1}},
        row 1/.append style={nodes={font=\fontsize{9}{10}\bfseries\selectfont}},
        row 2/.append style={set both},
        row 3/.append style={set 0},
        row 4/.append style={set 0},
        row 5/.append style={set 1},
        row 6/.append style={set 1},
        row 7/.append style={set 0},
        row 8/.append style={set 0},
        row 9/.append style={set 1},
        row 10/.append style={set 1},
        anchor=north west] at (-1, -0.75) {
        memory access \& set 0 afterwards \& set 1 afterwards \\ 
        --- \& (empty) \& (empty) \\
        read \lstinline|array[0]| (miss) \& \{array[0], array[1]\} \& (empty) \\
        read \lstinline|array[1]| (hit) \& \{array[0], array[1]\} \& (empty) \\
        read \lstinline|array[2]| (miss) \& \{array[0], array[1]\} \& \{array[2], array[3]\} \\
        read \lstinline|array[3]| (hit) \& \{array[0], array[1]\} \& \{array[2], array[3]\} \\
        read \lstinline|array[4]| (miss) \& \{array[4], array[5]\} \& \{array[2], array[3]\} \\
        read \lstinline|array[5]| (hit) \& \{array[4], array[5]\} \& \{array[2], array[3]\} \\
        read \lstinline|array[6]| (miss) \& \{array[4], array[5]\} \& \{array[6], array[7]\} \\
        read \lstinline|array[7]| (hit) \& \{array[4], array[5]\} \& \{array[6], array[7]\} \\
    };
    \end{visibleenv}
    \begin{visibleenv}<4-5>
        \node[align=left,draw=red,ultra thick,fill=white] at (6.5, -0.25) {
            observation: what happens in set 0 doesn't affect set 1 \\
            \myemph<5>{when evaluating set 0 accesses,} \\
            \myemph<5>{can ignore non-set 0 accesses/content}
        };
    \end{visibleenv}
\end{tikzpicture}
\end{frame}

\iftoggle{heldback}{}{
\againframe<1-3>{arrayMissesWarmup3Answers}
\againframe<4->{arrayMissesWarmup3Answers}
}

