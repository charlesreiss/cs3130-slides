\begin{frame}{privilege levels?}
    \begin{itemize}
    \item vulnerable code runs with higher privileges
    \item so far: higher privileges = kernel mode
    \vspace{.5cm}
    \item but other common cases of higher privileges
    \item example: scripts in web browsers
    \end{itemize}
\end{frame}

\begin{frame}[fragile]{JavaScript}
\begin{itemize}
\item JavaScript: scripts in webpages
\item not supposed to be able to read arbitrary memory, but\ldots
\item can access arrays to examine caches
\item and could take advantage of some browser function being vulnerable
\vspace{.5cm}
\item<2-> or --- \myemph<2>{doesn't even need browser to supply vulnerable code itself!}
\end{itemize}
\end{frame}

\begin{frame}[fragile]{just-in-time compilation?}
\begin{itemize}
\item for performance, compiled to machine code, run in browser
\item not supposed to be access arbitrary browser memory
\item example JavaScript code from paper:
\end{itemize}
\begin{lstlisting}[language=JavaScript,style=small]
if (index < simpleByteArray.length) {
    index = simpleByteArray[index | 0];
    index = (((index * 4096)|0) & (32*1024*1024-1))|0;
    localJunk ˆ= probeTable[index|0]|0;
}
\end{lstlisting}
\begin{itemize}
\item web page runs a lot to train branch predictor
\item then does run with out-of-bounds index
\item examines what's evicted by probeTable access
\end{itemize}
\end{frame}
