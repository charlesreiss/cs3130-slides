\usetikzlibrary{shapes.callouts,positioning,calc}

\begin{frame}{on implementing try/catch}
\begin{itemize}
\item could do something like setjmp()/longjmp()
\item but setjmp is \myemph{slow}
\end{itemize}
\end{frame}

\begin{frame}[fragile,label=lowOverTryCatch1]{low-overhead try/catch (1)}
\lstset{language=C,style=small}
\begin{lstlisting}
main() {
  printf("about to read file\n");
  try {
    read_file();
  } catch(...) {
    printf("some error happened\n");
  }
}
read_file() {
  ...
  if (open failed) {
      throw IOException();
  }
  ...
}
\end{lstlisting}
\end{frame}

\begin{frame}[fragile,label=lowOverTryCatch2]{low-overhead try/catch (2)}
\lstset{language=C,style=small,
    basicstyle={\tt\fontsize{10}{11}\selectfont},
    moredelim=**[is][\btHL<2|handout:2>]{@2}{@},
    moredelim=**[is][\btHL<3|handout:3>]{@3}{@},
    moredelim=**[is][\btHL<4|handout:4>]{@4}{@},
}
\begin{tikzpicture}
\tikzset{every node/.style={font=\small}}
\node[draw] (code1) {
\begin{lstlisting}
main:
  ...
  call printf
start_try:
  call read_file
end_try:
  ret
\end{lstlisting}
};
\node[draw,anchor=north west] (code2) at ([xshift=.25cm]code1.north east) {
\begin{lstlisting}
main_catch:
  movq $str, %rdi 
  call printf
  jmp end_try
\end{lstlisting}
};
\node[draw,anchor=north west] (code3) at ([xshift=.25cm]code2.north east) {
\begin{lstlisting}
read_file:
  pushq %r12
  ...
  @2call do_throw@
  ...
end_read:
  popq %r12
  ret
\end{lstlisting}
};
\matrix[label={north:lookup table},anchor=north west,
        tight matrix,
        nodes={font=\small,execute at begin node={\tt\strut\normalfont}},
        column 1/.append style={nodes={text width=6cm}},
        column 2/.append style={nodes={text width=3.9cm}},
        column 3/.append style={nodes={text width=1.8cm}},    
        row 1/.append style={nodes={font=\small\bfseries,execute at begin node={\tt\strut\normalfont\bfseries}}},
] (tbl) at ([yshift=-1cm]code1.south west){
    program counter range \& action \& recurse? \\
    {\tt start\_try} to {\tt end\_try} \& \myemph<3>{\tt jmp main\_catch} \& \myemph<3>{no} \\
    {\tt read\_file} to {\tt end\_read} \& \myemph<2>{\tt popq \%r12, ret} \& \myemph<2>{yes} \\
    anything else \& error \& --- \\
};
\begin{visibleenv}<4>
\node[align=left,my callout2=tbl-3-2.north,anchor=south] at ([xshift=-1cm,yshift=2cm]tbl-3-2.north) {
    not actual x86 code to run \\
    track a ``virtual PC'' while looking for catch block
};
\end{visibleenv}
\end{tikzpicture}
\end{frame}

\begin{frame}{lookup table tradeoffs}
\begin{itemize}
\item no overhead if throw not used
\item handles local variables on registers/stack, but\ldots{}
\vspace{.5cm}
\item larger executables (probably)
\item extra complexity for compiler 
\end{itemize}
\end{frame}


