\usetikzlibrary{matrix}

\begin{frame}{branch target buffer}
    \begin{itemize}
    \item what if we can't decode LABEL from machine code for \texttt{jmp LABEL} or \texttt{jle LABEL} fast?
        \begin{itemize}
        \item will happen in more complex pipelines
        \end{itemize}
    \item what if we can't decode that there's a RET, CALL, etc. fast?
    \end{itemize}
\end{frame}

\begin{frame}[fragile,label=btbStructure]{BTB: cache for branch targets}
\begin{tikzpicture}
\tikzset{
    marked idx/.style={fill=blue!15},
    marked tag/.style={fill=green!15},
    marked target/.style={fill=cyan!15},
}
\matrix[
    tight matrix,
    nodes={font=\fontsize{9}{10}\tt\selectfont},
    row 1/.append style={nodes={font=\bfseries\fontsize{9}{10}\selectfont}},
    column 1/.style={nodes={text width=1cm,draw=none}},
    column 2/.style={nodes={text width=0.8cm}},
    column 3/.style={nodes={text width=1.5cm}},
    column 4/.style={nodes={text width=0.8cm}},
    column 5/.style={nodes={text width=2cm}},
    column 6/.style={nodes={text width=3cm}},
    column 7/.style={nodes={text width=2cm}},
    column 8/.style={nodes={text width=0.25cm,draw=none}},
    column 9/.style={nodes={text width=0.8cm}},
] (btb) at (0, 0) {
    idx \& valid \& tag \& ofst \& type \& target \& (more info?) \& ~ \& valid \& \ldots\\
    |[alt=<3>{marked idx}]| 0x00 \& 1 \& |[alt=<3>{marked tag}]| 0x400 \& |[alt=<3>{marked tag}]| 5 \& Jxx \& |[alt=<3>{marked target}]| 0x3FFFF3 \& \ldots \& ~ \& 1 \& \ldots\\
    0x01 \& 1 \& 0x401 \& C \& JMP \& 0x401035 \& --- \& ~ \& 0 \& \ldots \\
    0x02 \& 0 \& --- \& ---\& --- \& --- \& ---  \& ~ \& 0 \&\ldots \\
    0x03 \& 1 \&  0x400 \& 9 \& RET \& --- \& \ldots \& ~ \& 0 \&\ldots\\
    \ldots \& \ldots \& \ldots \& \ldots \& \ldots \& \ldots \& \ldots \& ~ \& \ldots \& \ldots \\
    |[alt=<2>{marked idx}]| 0xFF \& 1 \& |[alt=<2>{marked tag}]| 0x3FF \& |[alt=<2>{marked tag}]| 8 \& CALL \& |[alt=<2>{marked target}]| 0x404033 \& \ldots \& ~ \& 0 \& \ldots \\
};
\matrix[tight matrix,
    nodes={font=\fontsize{9}{10}\tt\selectfont},
    column 1/.style={nodes={text width=2cm,draw=none}},
    column 2/.style={nodes={text width=8cm,draw=none}},
    anchor=north west,
] at ([yshift=-1cm]btb.south west) {
    0x3FFFF3: \& movq \%rax, \%rsi \\
    0x3FFFF7: \& pushq \%rbx\\
    \alt<2>{\fboxsep=0pt0x\colorbox{green!15}{3FF}\colorbox{blue!15}{FF}\colorbox{green!15}{8}}{0x3FFFF8}: \& call \alt<2>{\fboxsep0pt\colorbox{cyan!15}{0x404033}}{0x404033} \\
    0x400001: \& popq \%rbx \\
    0x400003: \& cmpq \%rbx, \%rax \\
    \alt<3>{\fboxsep=0pt0x\colorbox{green!15}{400}\colorbox{blue!15}{00}\colorbox{green!15}{5}}{0x400005}: \& jle \alt<3>{\fboxsep0pt\colorbox{cyan!15}{0x3FFFF3}}{0x3FFFF3} \\
    \ldots \& \ldots \\
    0x400031:  \& ret \\
    \ldots \& \ldots \\
};
\end{tikzpicture}
\end{frame}

\begin{frame}{indirect branch prediction}
    \begin{itemize}
        \item for instructions like: \texttt{jmp *\%rax} or \texttt{jmp *(\%rax, \%rcx, 8)}
        \item simple idea: record what happened last time, predict the same
        \item example: Intel's Haswell strategy:
        \vspace{.5cm}
        \item maintain hash of last several jump instruction addresses
        \item lookup hash in table of targets
        \item use result as prediction, update if prediction wrong
    \end{itemize}
\end{frame}
