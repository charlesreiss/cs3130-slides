\usetikzlibrary{matrix,fit}
\begin{frame}[fragile,label=addqDataHazard]{a data hazard}
\begin{tikzpicture}[overlay,remember picture]
\node[anchor=north east] at ([xshift=-.5cm,yshift=-1cm]current page.north east)
    {\resizebox{!}{0.16\textwidth}{\usebox{\pipeCpuBox}}};
\end{tikzpicture}
\lstset{
    style=small,
    moredelim=**[is][\btHL<2|handout:0>]{@2@}{@},
    moredelim=**[is][\btHL<3|handout:0>]{@3@}{@},
    moredelim=**[is][\btHL<4|handout:0>]{@4@}{@},
    moredelim=**[is][\btHL<5|handout:0>]{@5@}{@},
}
\begin{lstlisting}
// initially %r8 = 800,
//           %r9 = 900, etc.
addq %r8, %r9   // R8 + R9 -> R9
addq %r9, %r8   // R9 + R8 -> R9
addq ...
addq ...        
\end{lstlisting}
\begin{tikzpicture}
\matrix[tight matrix,nodes={text width=1cm,font=\scriptsize\tt},
        column 8/.append style={nodes={text width=1.2cm}},
        column 10/.append style={nodes={text width=1.2cm}},
        row 1/.append style={nodes={font=\scriptsize}},
        row 2/.append style={nodes={font=\scriptsize}},
        ] (tbl) {
 \& |[fill=yellow!10!white,align=center]| fetch \& rA \& rB \& R[rA] \& R[rB] \& rB \& sum \& rB \& sum \& rB \\
cycle \& PC \& rA \& rB \& R[rB] \& R[rB] \& rB \& sum \& rB \& sum \& rB \\
0 \& 0x0 \\
1 \& 0x2 \& 8  \& 9    \\ 
2 \&     \& 9  \& 8   \& 800  \& 900 \& 9 \\
3 \&     \&    \&     \& 900 \&  800 \& 8  \& 1700 \& 9 \\
4 \&     \&    \&     \&      \&      \&    \& 1700 \& 8  \& 1700 \& 9 \\
5 \&     \&    \&     \&      \&       \&    \&      \&   \& 1700 \& 8 \\
};
\begin{scope}[every node/.style={draw,inner sep=0pt},
              every label/.style={font=\scriptsize,draw=none}]
    \node[fit=(tbl-1-3) (tbl-1-4),fill=yellow!50!orange!10!white,label={center:fetch/decode}] {};
    \node[fit=(tbl-1-5) (tbl-1-7),fill=orange!50!green!10!white,label={center:decode/execute}] {};
    \node[fit=(tbl-1-8) (tbl-1-9),fill=green!50!blue!10!white,label={center:execute/memory}] {};
    \node[fit=(tbl-1-10) (tbl-1-11),fill=blue!50!violet!10!white,label={center:memory/writeback}] {};
\end{scope}

\begin{visibleenv}<2->
\node[red,draw,thick,label={[red]-90:should be 1700},fit=(tbl-6-5)] {};
\end{visibleenv}
\end{tikzpicture}
\end{frame}

\begin{frame}[fragile,label=DataHazardTime]{data hazard}
\begin{lstlisting}
addq %r8, %r9  // (1)
addq %r9, %r8  // (2)
\end{lstlisting}
\begin{tikzpicture}
\matrix[tight matrix,
    nodes={execute at begin node={\strut}},
    column 1/.append style={nodes={text width=1.4cm}},
    column 2/.append style={nodes={text width=6cm}},
    column 3/.append style={nodes={text width=4cm}},
    ] {
step\# \& pipeline implementation \& ISA specification\\
1      \& read r8, r9 for (1) \& read r8, r9 for (1) \\
2      \& read r9, r8 for (2) \& write r9 for (1) \\
3      \& write r9 for (1) \& read r9, r8 for (2) \\
4      \& write r8 for (2) \& write r8 ror (2) \\
};
\end{tikzpicture}
\begin{itemize}
\item pipeline reads \myemph{older value}\ldots
\item instead of value ISA says was just written
\end{itemize}
\end{frame}

\begin{frame}[fragile,label=dataHazardNoop]{data hazard compiler solution}
\begin{lstlisting}[style=small]
addq %r8, %r9
nop
nop
addq %r9, %r8
\end{lstlisting}
\begin{itemize}
\item one solution: \myemph{change the ISA}
    \begin{itemize}
    \item all \lstinline|addq|s take effect \myemph{three instructions later} \\
        (assuming can read register value while it is being written back)
    \end{itemize}
\item make it \myemph{compiler's  job}
\item problem: recompile everytime processor changes?
\end{itemize}
\end{frame}

\begin{frame}[fragile,label=dataHazardStall]{data hazard hardware solution}
\begin{lstlisting}[style=small]
addq %r8, %r9
// hardware inserts: nop
// hardware inserts: nop
addq %r9, %r8
\end{lstlisting}
\begin{itemize}
    \item how about hardware add nops?
    \item called \myemph{stalling}
    \item extra logic:
        \begin{itemize}
        \item sometimes don't change PC
        \item sometimes put do-nothing values in pipeline registers
        \end{itemize}
\end{itemize}
% FIXME: picture of MUX in front of PC, in front of fetch/decode
\end{frame}
\begin{frame}{stalling/nop pipeline diagram (1)}
\begin{tikzpicture}
    \stagestyles
    \providecommand{\fdemw}{ \& |[fetch]| F \& |[decode]| D \& |[execute]| E \& |[memory]| M \& |[writeback]| W}
        \matrix[tight matrix no line,
                nodes={text width=.6cm,font=\tt,minimum height=.7cm,align=center,text opacity=1.0,fill opacity=0.5},
                column 1/.style={nodes={text width=9cm,align=left}},anchor=north west] (tbl) {
        |[align=right]| \normalfont\textit{cycle \#} \hspace{.5cm}
                                            \& 0 \& 1 \& 2 \& 3 \& 4 \& 5 \& 6 \& 7 \& 8 \\
        add \%r8, \%r9    \fdemw \& ~ \& ~ \& ~ \& ~ \& ~ \& ~\\
        (nop) \& ~ \fdemw \& ~ \& ~\\
        (nop) \& ~ \& ~ \fdemw \& ~ \& ~\\
        addq \%r9, \%r8  \& ~ \& ~ \& ~ \fdemw \& ~ \& ~\\
        };
    \begin{visibleenv}<2>
        \node[draw=red,line width=1mm,inner sep=0mm,fit=(tbl-2-6)] {};
        \node[draw=red,line width=1mm,inner sep=0mm,fit=(tbl-5-6)] {};
        \draw[red,line width=1mm,dotted,->] (tbl-2-6) -- (tbl-5-6);
        \node[anchor=north,align=left] at (tbl.south) {assumption: \\
                if writing register value \\
                register file will return that value  for reads \\
                ~ \\
                not actually way register file worked in single-cycle CPU \\
                (e.g. can read old \%r9 while writing new \%r9)
            };
    \end{visibleenv}
\end{tikzpicture}
\end{frame}

\begin{frame}{stalling/nop pipeline diagram (2)}
\begin{tikzpicture}
    \stagestyles
    \providecommand{\fdemw}{ \& |[fetch]| F \& |[decode]| D \& |[execute]| E \& |[memory]| M \& |[writeback]| W}
        \matrix[tight matrix no line,
                nodes={text width=.6cm,font=\tt,minimum height=.7cm,align=center,text opacity=1.0,fill opacity=0.5},
                column 1/.style={nodes={text width=9cm,align=left}},anchor=north west] (tbl) {
        |[align=right]| \normalfont\textit{cycle \#} \hspace{.5cm}
                                            \& 0 \& 1 \& 2 \& 3 \& 4 \& 5 \& 6 \& 7 \& 8 \\
        add \%r8, \%r9    \fdemw \& ~ \& ~ \& ~ \& ~ \& ~ \& ~\\
        (nop) \& ~ \fdemw \& ~ \& ~\\
        (nop) \& ~ \& ~ \fdemw \& ~ \& ~\\
        |[fill=red!20]| (nop) \& ~ \& ~ \& ~  \fdemw \& ~ \& ~\\
        addq \%r9, \%r8  \& ~ \& ~ \& ~ \& ~\& ~ \fdemw \& ~ \& ~\\
        };
    \begin{visibleenv}<2>
        \node[draw=red,line width=1mm,inner sep=0mm,fit=(tbl-2-6)] {};
        \node[draw=red,line width=1mm,inner sep=0mm,fit=(tbl-6-7)] {};
        \draw[red,line width=1mm,dotted,->] (tbl-2-6) -- (tbl-6-7);
        \node[red,anchor=north] at (tbl.south) {
            if we didn't modify the register file, we'd need an extra cycle
            };
    \end{visibleenv}
\end{tikzpicture}
\end{frame}
