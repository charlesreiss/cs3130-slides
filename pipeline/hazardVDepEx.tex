\begin{frame}[fragile,label=nonHazard]{ex.: dependencies and hazards (1)}
\begin{tikzpicture}
    \matrix[tight matrix no line,
            row sep=4mm,
            nodes={font=\tt,
                   text width=1.5cm,minimum width=2.5cm,text depth=.5ex},
            column 1/.style={nodes={text width=1.9cm,font=\tt\bfseries}},
    ] (instrs){
        addq \& \%rax, \& \%rbx \\
        subq \& \%rax, \& \%rcx \\
        irmovq \& \$100, \& \%rcx \\
        addq \& \%rcx, \& \%r10 \\
        addq \& \%rbx, \& \%r10 \\
    };
    \node[draw,below=.4cm of instrs,align=left] {
        where are dependencies? \\
        which are hazards in our pipeline? \\
        which are resolved wiht forwarding?
    };
    \tikzset{
        depHi/.style={draw,blue,thick,inner sep=0mm,inner xsep=-5mm,rounded corners},
        depLine/.style={blue,very thick,-Latex},
        hazHi/.style={draw,red,thick,inner sep=0mm,inner xsep=-5mm,rounded corners},
        hazLine/.style={red,very thick,-Latex},
    }
    \begin{visibleenv}<2->
        \node[fit=(instrs-1-3),depHi] (hiRBXWrite) {};
        \node[fit=(instrs-5-2),depHi] (hiRBXRead) {};
        \draw[depLine] (hiRBXWrite) -- (hiRBXRead);
    \end{visibleenv}
    \begin{visibleenv}<3->
        \node[fit=(instrs-3-3),hazHi] (hiRCXWrite) {};
        \node[fit=(instrs-4-2),hazHi] (hiRCXRead) {};
        \draw[hazLine] (hiRCXWrite) -- (hiRCXRead);
    \end{visibleenv}
    \begin{visibleenv}<4->
        \node[fit=(instrs-4-3),hazHi] (hiRTenWrite) {};
        \node[fit=(instrs-5-3),hazHi] (hiRTenRead) {};
        \draw[hazLine] (hiRTenWrite) -- (hiRTenRead);
    \end{visibleenv}
\end{tikzpicture}
\end{frame}
