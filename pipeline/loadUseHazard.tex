\usetikzlibrary{positioning,matrix}
\begin{frame}[fragile,label=loadUse]{unsolved problem}
\lstset{
    language=myasm,
    morekeywords={mrmovq,irmovq,rmmovq,rrmovq},
    moredelim=**[is][\btHL<2|handout:0>]{~2~}{~end~},
    moredelim=**[is][\btHL<4|handout:0>]{~3~}{~end~},
}
\begin{tikzpicture}
\tikzset{
    forwardLine/.style={blue!60!black,thick,-Latex},
    hiForwardOn/.style={alt=#1{blue}{}},
    forwardLineB/.style={red,dashed,thick,-Latex},
    sepLine/.style={black!40,densely dotted},
};
\tikzset{fetch/.style={fill=yellow!15},decode/.style={fill=orange!15},execute/.style={fill=green!15},
memory/.style={fill=blue!15},writeback/.style={fill=violet!15}}
\providecommand{\fdemw}{ \& |[fetch]| F \& |[decode]| D \& |[execute]| E \& |[memory]| M \& |[writeback]| W}
\matrix[tight matrix no line,
        nodes={text width=.6cm,font=\tt,minimum height=.7cm,align=center},
        column 1/.style={nodes={text width=6cm,align=left}},
        row 3/.style={alt=<2->{opacity=0.25}{}},
        row 4/.style={visible on=<2->}] (tbl) {
|[align=right]| \normalfont\textit{cycle \#} \hspace{.5cm}
                                    \& 0 \& 1 \& 2 \& 3 \& 4 \& 5 \& 6 \& 7 \& 8 \\
\lstinline|movq 0(%rax), %rbx|   \fdemw \& ~ \& ~ \& ~ \& ~ \\
\lstinline|subq %rbx, %rcx|        \& ~ \fdemw \& ~ \& ~ \& ~ \\
\lstinline|subq %rbx, %rcx|        \& ~ \& |[fetch]| F \& |[decode]| D \& |[decode]| D \& |[execute]| E \& |[memory]| M \& |[writeback]| W \& ~ \& ~ \\
};
\foreach \x in {1,...,10} {
    \draw (tbl-1-\x.north east) -- (tbl-4-\x.south east);
}
\newcommand{\fwd}[4]{
    \pgfmathtruncatemacro{\idx}{#1+2}
    \draw[forwardLine,hiForwardOn=#4] ([xshift=-.15cm]tbl-#2-\idx.west) -- ([xshift=-.05cm]tbl-#3-\idx.west);
}
\newcommand{\fwdA}[4]{
    \pgfmathtruncatemacro{\idx}{#1+2}
    \draw[forwardLineB] ([xshift=-.15cm]tbl-#2-\idx.east) -- ([xshift=-.05cm]tbl-#3-\idx.east);
}

\newcommand{\fwdNo}[3]{
    \pgfmathtruncatemacro{\idx}{#1+2}
    \draw[forwardLineB] ([xshift=-.15cm]tbl-#2-\idx.west) -- ([xshift=-.05cm]tbl-#3-\idx.west);
}
\fwdNo{4}{2}{3}
\begin{visibleenv}<2->
    %\draw[thick,black!40] ([yshift=.1cm]tbl-3-1.west) -- ([yshift=-.1cm]tbl-3-10.east);
    \fwd{4}{2}{4}{<0|handout:0>}
\end{visibleenv}
\begin{visibleenv}<2->
    \node[draw,thick,red,fit=(tbl-4-4) (tbl-4-5)] {};
    \node [red,below=0cm of tbl-4-4] {stall};
\end{visibleenv}
\end{tikzpicture}
\begin{itemize}
    \item \textbf{combine} stalling and forwarding to resolve hazard
    \item assumption in diagram: hazard detected in \texttt{subq}'s decode stage
        \begin{itemize}
        \item (since easier than detecting it in fetch stage)
        \end{itemize}
\end{itemize}
\end{frame}
