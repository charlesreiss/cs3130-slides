\begin{frame}{secure communication context}
    \begin{itemize}
    \item ``secure'' communication
    \item mostly talk about on network
    \item between \textit{principals} $\approx$ people/servers/programs
    \vspace{.5cm}
    \item but same ideas apply to, e.g., messages on disk
        \begin{itemize}
        \item communicating with yourself
        \end{itemize}
    \end{itemize}
\end{frame}

\begin{frame}{A to B}
    \begin{itemize}
    \item running example: A talking with B
        \begin{itemize}
        \item maybe sometimes also with C
        \end{itemize}
    \item attacker E --- eavesdropper
        \begin{itemize}
        \item passive
        \item gets to read all messages over network
        \end{itemize}
    \item attacker M (man-in-the-middle)
        \begin{itemize}
        \item active
        \item gets to read and replace and add messages on the network
        \end{itemize}
    \end{itemize}
\end{frame}

\begin{frame}{possible security properties?}
    \begin{itemize}
    \item what we'll talk about: 
    \item confidentiality --- information shared only with those who should have it
    \item authenticity --- message genuinely comes from right principal (and not manipulated)
    \vspace{.5cm}
    \item important ones we won't talk about\ldots:
    \item repudiation --- if A sends message to B, B can't prove to C it came from A
        \begin{itemize}
        \item (takes extra effort to get along with authenticity)
        \end{itemize}
    \item forward-secrecy --- if A compromised now, E can't use that to decode past conversations with B
    \item anonymity --- A can talk to B without B knowing who it is
    \item \ldots
    \end{itemize}
\end{frame}


