\begin{frame}{just asymmetric?}
    \begin{itemize}
    \item given public-key encryption + digital signatures\ldots
    \item why bother with the symmetric stuff?
    \vspace{.5cm}
    \item symmetric stuff much faster
    \item symmetric stuff much better at supporting larger messages
    \end{itemize}
\end{frame}

\begin{frame}{key agreement}
    \begin{itemize}
    \item problem: A has B's public encryption key \\
        wants to choose shared secret 
    \vspace{.5cm}
    \item some ideas:
        \begin{itemize}
        \item A chooses a key, sends it encrypted to B
        \item A sends a public key encrypted B, B chooses a key and sends it back
        \end{itemize}
    \item<2-> alternate model (not needed, but usually used by TLS, SSH, \ldots):
        \begin{itemize}
        \item both sides generate random values
        \item derive public-key like ``key shares'' from values
        \item use math to combine ``key shares''
        \item kinda like A + B both sending each other public encryption keys
        \end{itemize}
    \end{itemize}
\end{frame}

\begin{frame}{Diffie-Hellman key agreement}
\begin{itemize}
\item A and B want to agree on shared secret
\vspace{.5cm}
\item A chooses random value Y
\item A sends public value derived from Y (``key share'')
\item B chooses random value Z
\item B sends public value derived from Z (``key share'')
\item A combines Y with public value from B to get number
\item B combines Z with public value from A to get number
    \begin{itemize}
    \item and b/c of math chosen, both get same number
    \end{itemize}
\end{itemize}
\end{frame}

\begin{frame}{Diffie-Hellman key agreement (details, if needed)}
    \begin{itemize}
    \item math requirement:
        \begin{itemize}
        \item some $f$, so $f(f(X, Y), Z) = f(f(X, Z), Y)$
        \item (that's hard to invert, etc.)
        \end{itemize}
    choose X in advance and:
    \end{itemize}
\begin{tabular}{l|l}
A randomly chooses $Y$ & B randomly chooses $Z$ \\
A sends $f(X, Y)$ to B & B sends $f(X, Z)$ to A \\
A computes $f(f(X, Z), Y)$ & B computes $f(f(X, Y), Z)$ \\
\end{tabular} \\
\vspace{.5cm}
\only<2->{
    example $f(a, b) = a^b \pmod{p}$
}
\end{frame}

