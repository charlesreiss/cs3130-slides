\begin{frame}{symmetric encryption}
    \begin{itemize}
    \item some magic math!
    \vspace{.5cm}
    \item we'll be given two functions by expert:
        \begin{itemize}
        \item encrypt: $E(\text{key}, \text{message}) = \text{ciphertext}$
        \item decrypt: $D(\text{key}, \text{ciphertext}) = \text{message}$
        \end{itemize}
    \item key = shared secret
        \begin{itemize}
        \item ideally chosen at random
        \item should be small --- otherwise impractical to share
        \item unsolved problem: how do A and B both know this?
        \end{itemize}
    \end{itemize}
\end{frame}

\begin{frame}{symmetric encryption properties}
    \begin{itemize}
    \item our functions:
        \begin{itemize}
        \item encrypt: $E(\text{key}, \text{message}) = \text{ciphertext}$
        \item decrypt: $D(\text{key}, \text{ciphertext}) = \text{message}$
        \end{itemize}
    \item knowing $E$ and $D$, it should be hard to:
        \begin{itemize}
        \item learn about the message from the ciphertext without key; or
        \item learn about the key from the ciphertext and message
        \end{itemize}
    \item ``hard'' $\approx$ would have to try every possible key
    \end{itemize}
\end{frame}
