\begin{frame}{random numbers}
    \begin{itemize}
    \item need a lot of keys that no one else knows
    \vspace{.5cm}
    \item common task: choose a \textit{random} number
    \item question: what does \textit{random} mean here?
    \end{itemize}
\end{frame}

\begin{frame}{cryptographically secure random numbers}
    \begin{itemize}
        \item security properties we might want for random numbers:
        \vspace{.5cm}
    \item attacker cannot guess (part of) number better than chance
    \item knowing prior `random' numbers shouldn't help predict next `random' numbers
    \item compromising machine now shouldn't reveal older random numbers
    \end{itemize}
\end{frame}

\begin{frame}{exercise: how to generate?}
\end{frame}

\begin{frame}{/dev/urandom}
    \begin{itemize}
    \item Linux kernel random number generator
    \vspace{.5cm}
    \item collects ``entropy'' from hard-to-predict events
        \begin{itemize}
        \item e.g. exact timing of I/O interrupts
        \item e.g. some processor's built-in random number circuit
        \end{itemize}
    \item turned into as many random bytes as you want
    \end{itemize}
\end{frame}

% \begin{frame}{turning `entropy' into random bytes}
%     \begin{itemize}
%     \item lots of ways to do this; one (rough/incomplete) idea:
%     \item internal variable \textit{state}
%     \item to add `entropy'
%         \begin{itemize}
%         \item state $\leftarrow$ SecureHash(state + entropy)
%         \end{itemize}
%     \item to extract value:
%         \begin{itemize}
%         \item random bytes $\leftarrow$ SecureHash(1 + state) \\
%             \small give bytes that can't be reversed to compute state
%                 \vspace{.5cm}
%         \item state $\leftarrow$ SecureHash(2 + state) \\
%             \small change state so attacker can't take us back to old state if compromised
%         \end{itemize}
%     \end{itemize}
% \end{frame}
