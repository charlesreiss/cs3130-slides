\begin{frame}{getting public keys?}
    \begin{itemize}
    \item browser talking to websites \\
    needs public keys of every single website?
    \vspace{.5cm}
    \item not really feasible, but\ldots
    \end{itemize}
\end{frame}

\begin{frame}{certificate idea}
    \begin{itemize}
        \item let's say A has B's public key already.
        \item if C wants B's public key and knows A's already:
            \vspace{.5cm}
        \item A can send C:
            \begin{itemize}
            \item ``B's public key is XXX'' AND
            \item Sign(A's private key, ``B's public key is XXX'')
            \end{itemize}
        \item if C trusts A, now C has B's public key
            \begin{itemize}
            \item if C does not trust A, well, can't trust this either
            \end{itemize}
    \end{itemize}
\end{frame}

\begin{frame}{certificate authorities}
    \begin{itemize}
    \item instead, have public keys of trusted \textit{certificate authorities}
    \item only 10s of them, probably
    \vspace{.5cm}
    \item websites go to certificates authorities with their public key
    \item certificate authorities sign messages like: \\
        ``The public key for foo.com is XXX.''
    \item these signed messages called ``certificates''
    \end{itemize}
\end{frame}

\begin{frame}[fragile]{example web certificate (1)}
\begin{Verbatim}[fontsize=\scriptsize]
Certificate:
    Data:
        Version: 3 (0x2)
        Serial Number:
            81:13:c9:49:90:8c:81:bf:94:35:22:cf:e0:25:20:33
        Signature Algorithm: sha256WithRSAEncryption
        Issuer:
            commonName                = InCommon RSA Server CA
            organizationalUnitName    = InCommon
            organizationName          = Internet2
            localityName              = Ann Arbor
            stateOrProvinceName       = MI
            countryName               = US
        Validity
            Not Before: Feb 28 00:00:00 2022 GMT
            Not After : Feb 28 23:59:59 2023 GMT
        Subject:
            commonName                = collab.its.virginia.edu
            organizationalUnitName    = Information Technology and Communication
            organizationName          = University of Virginia
            stateOrProvinceName       = Virginia
            countryName               = US
.....
\end{Verbatim}
\end{frame}

\begin{frame}[fragile]{example web certificate (1)}
\begin{Verbatim}[fontsize=\scriptsize]
Certificate:
    Data:
....
        Subject Public Key Info:
            Public Key Algorithm: rsaEncryption
                RSA Public-Key: (2048 bit)
                Modulus:
                    00:a2:fb:5a:fb:2d:d2:a7:75:7e:eb:f4:e4:d4:6c:
                    94:be:91:a8:6a:21:43:b2:d5:9a:48:b0:64:d9:f7:
                    f1:88:fa:50:cf:d0:f3:3d:8b:cc:95:f6:46:4b:42:
....
        X509v3 extensions:
....
            X509v3 Extended Key Usage: 
                TLS Web Server Authentication, TLS Web Client Authentication
....
            X509v3 Subject Alternative Name: 
                DNS:collab.its.virginia.edu
                DNS:collab-prod.its.virginia.edu
                DNS:collab.itc.virginia.edu
    Signature Algorithm: sha256WithRSAEncryption
         39:70:70:77:2d:4d:0d:0a:6d:d5:d1:f5:0e:4c:e3:56:4e:31:
....
\end{Verbatim}
\end{frame}

\begin{frame}{certificate chains}
    \begin{itemize}
    \item That certificate signed by ``InCommon RSA Server CA''
    \item CA = certificate authority
    \vspace{.5cm}
    \item so their public key, comes with my OS/browser?
        \begin{itemize}
        \item not exactly\ldots
        \end{itemize}
    \item they have their own certificate signed by ``USERTrust RSA Certification Authority''
    \item and their public key comes with your OS/browser?
    \vspace{.5cm}
    \item (but both CAs now operated by UK-based Sectigo)
    \end{itemize}
\end{frame}
\usetikzlibrary{fit}
\begin{frame}{certificate hierarchy}
\begin{tikzpicture}
    \tikzset{
        cert/.style={draw, very thick,align=left,font=\small},
        root/.style={alt=<2>{fill=blue!10,line width=3pt}},
        root weak/.style={alt=<2>{fill=blue!10,line width=3pt,dashed}},
    }
    \node[cert,root] (usertrust) at (-3, 0) {
        USERTrust RSA \\ Certification Authority \\
        \parbox{2.25in}{
        \scriptsize originally operated by USERTrust, Inc.  \\
        \scriptsize acquired by Comodo, Inc (2004) \\
        \scriptsize Comodo's CA division renamed Sectigo (2018)
        }
    };
    \node[cert] (incommon) at (-4, -3) {
        InCommon \\ RSA Server CA \\
        \parbox{2in}{
        \scriptsize operated by Sectigo \\
        \scriptsize on behalf of the Internet2 (not-for-profit)
        }
    };
    \node[cert] (collab) at (-4, -5) {
        collab.its.virginia.edu
    };
    \node[overlay] (other incommon) at (-7, -5) {
        \ldots
    };
    \node[overlay] (other incommon 2) at (-1, -5) {
        \ldots
    };
    \node (other usertrust) at (0, -3) {
        \ldots
    };
    \begin{scope}[very thick,->]
        \draw (usertrust) -- (incommon);
        \draw (incommon) -- (collab);
        \draw[overlay] (incommon) -- (other incommon);
        \draw[overlay] (incommon) -- (other incommon 2);
        \draw (usertrust) -- (other usertrust);
    \end{scope}

    \begin{scope}[xshift=5cm]
        \node[cert,root] (globalsign) at (0, 0) {
            GlobalSign Root CA \\
            \parbox{2.25in}{
                \scriptsize
                operated by GlobalSign nv-sa \\
                subsid. of GMO Internet Group since 2007
            }
        };
        \node (globalsign other) at (3, -2) {\ldots};
        \node [cert,root weak] (goog root ca) at (-1, -2) {
            GTS Root R1 \\
            \parbox{2.25in}{
                \scriptsize
                operated by Google Trust Services LLC
            }
        };
        \node [cert] (goog subca) at (-1, -3.5) {
            GTS CA 1C3
        };
        \node (goog subother) at (2, -3.5) {\ldots};
        \node [cert] (goog) at (-1, -5) {
            www.google.com
        };
        \node (goog other) at (-4, -5) {\ldots};
        \begin{scope}[very thick,->]
            \draw (globalsign) -- (goog root ca);
            \draw (globalsign) -- (globalsign other);
            \draw (goog root ca) -- (goog subca);
            \draw (goog root ca) -- (goog subother);
            \draw (goog subca) -- (goog);
            \draw (goog subca) -- (goog other);
        \end{scope}
    \end{scope}

    \begin{visibleenv}<2>
        \begin{pgfonlayer}{fg}
        \begin{visibleenv}<2>
        \node[draw=yellow!50!black, fill=white, align=center,line width=3pt,fill=yellow!20] (over box) at (0, -5) {
            some ``trust anchors'' included with browsers and OSes \\
            (for GTS Root R1, only more recent browsers/OSes)
        };
        \end{visibleenv}
        \end{pgfonlayer}
        \node[fit=(over box),inner sep=.5cm,fill=white,fill opacity=0.5,overlay] {};
    \end{visibleenv}
\end{tikzpicture}
\end{frame}

\begin{frame}{how many trust anchors?}
    \begin{itemize}
    \item Mozilla Firefox (as of 27 Feb 2023)
        \begin{itemize}
            \item 155 trust anchors
            \item operated by 55 distinct entities
        \end{itemize}
    \item Microsoft Windows (as of 27 Feb 2023)
        \begin{itemize}
            \item 237 trust anchors
            \item operated by 86 distinct entities
        \end{itemize}
    \end{itemize}
\end{frame}

\begin{frame}{public-key infrastructure}
    \begin{itemize}
    \item ecosystem with certificate authorities \\
        and certificates for everyone
    \item called ``public-key infrastructure''
        \vspace{.5cm}
    \item several of these:
        \begin{itemize}
        \item for verifying identity of websites 
        \item for verifying origin of domain name records (kind-of)
        \item for verifying origin of applications in some OSes/app stores/etc.
        \item for encrypted email in some organizations
        \item \ldots
        \end{itemize}
    \end{itemize}
\end{frame}
