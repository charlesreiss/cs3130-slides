\begin{frame}{encryption is not enough}
    \begin{itemize}
    \item if B receives an encrypted message from A, and\ldots
    \item it makes sense when decrypted, why isn't that good enough?
    \vspace{.5cm}
    \item problem: an active attacker M \\
        can \textit{selectively} manipulate the encrypted message
    \end{itemize}
\end{frame}

\begin{frame}{manipulating encrypted data?}
\begin{itemize}
\item one example: common symmetric encryption approach:
    \begin{itemize}
    \item use random number + shared secret to\ldots
    \item produce sequence of hard-to-guess bits $x_i$ as long as the message
    \item produce ciphertext with xor: $c_i = m_i \oplus x_i$
    \item message = $m_0m_1m_2\ldots$; ciphertext = [random number]$c_0c_1c_2\ldots$
    \end{itemize}
\item means that flipping $c_i$ flips bit $m_i$ 
\item also means that we can shorten messages silently
\end{itemize}
\end{frame}

\begin{frame}{manipulating messages}
\begin{itemize}
\item as an active attacker
\vspace{.5cm}
\item if we know part of plaintext \\
    can sometimes make it read anything else by flipping bits
    \begin{itemize}
    \item ``Pay \$100 to Bob'' $\rightarrow$ ``Pay \$999 to Bob''
    \end{itemize}
\item we can shorten 
    \begin{itemize}
    \item ``Pay \$100 to ABC Corp if they \ldots'' $\rightarrow$ ``Pay \$100 to ABC Corp''
    \end{itemize}
\item we can corrupt selected parts of message and check the response is
    \begin{itemize}
    \item e.g. what changes don't make B reject message as malformed?
    \end{itemize}
\end{itemize}
\end{frame}
