\begin{frame}{encryption is not enough}
    \begin{itemize}
    \item if B receives an encrypted message from A, and\ldots
    \item it makes sense when decrypted, why isn't that good enough?
    \vspace{.5cm}
    \item problem: an active attacker M \\
        can \textit{selectively} manipulate the encrypted message
    \end{itemize}
\end{frame}

\begin{frame}{simple encryption idea (1)}
    \begin{itemize}
    \item suppose encrypting message
    \item one possible idea: generate unique number N (e.g. counter)
    \item combine N and key to produce message size-bit bitstring Y
    \item say Y=f(X, key) where $f$ is some `secure' function:
        \begin{itemize}
        \item $f$ is something like a hash function, but supports arbitrary size output
        \item $f$ is effectively irreversible
        \item $f$ is effectively equally/unpredictably distributed
        \end{itemize}
    \item use Y XOR message as encrypted value
    \end{itemize}
\end{frame}

\begin{frame}{simple encryption idea (2)}
    \begin{itemize}
    \item E(K, message) = (N, f(N, key) XOR message) = (N, C)
    \item If we know (N, C) and don't know key,
        can we figure out anything about \textit{message}?
        \begin{itemize}
        \item violates f(N, key) being equally/unpredictably distributed
        \end{itemize}
    \item If we know (N, C) and message, can we find out about \textit{key}
        \begin{itemize}
        \item violates f(N, key) being irreversible
        \end{itemize}
    \item If we know (N, C) and message, can we decrypt (N', C')?
        \begin{itemize}
        \item remember: each encrypted message chooses unique N
        \item not if f(N, key) and f(N', key) don't have predictable relationship
        \end{itemize}
    \end{itemize}
\end{frame}

\begin{frame}{manipulating simple encryption}
    \begin{itemize}
    \item E(K, message) = (N, f(N, key) xor message) = (N, C)
    \item If we know (N, C) and message, we can generate an encryption of (all zeroes):
    \item (N, C xor message) = (N, f(N, key) xor (message xor message)) = (N, f(N, key) xor 0)
    \item And can generate encryption of some other message Q:
    \item (N, C xor message xor Q) = (N, f(N, key) xor Q)
    \end{itemize}
\end{frame}

\begin{frame}{manipulating messages, more generally}
\begin{itemize}
\item as an active attacker
\vspace{.5cm}
\item if we know part of plaintext \\
    can sometimes make it read anything else by flipping bits
    \begin{itemize}
    \item ``Pay \$100 to Bob'' $\rightarrow$ ``Pay \$999 to Bob''
    \end{itemize}
\item we can sometimes shorten
    \begin{itemize}
    \item ``Pay \$100 to ABC Corp if they \ldots'' $\rightarrow$ ``Pay \$100 to ABC Corp''
    \end{itemize}
\item we can sometimes corrupt selected parts of message and check what the response is
    \begin{itemize}
    \item e.g. what changes don't make B reject message as malformed?
    \item with repeated tries, often reveals part of message's values
    \end{itemize}
\end{itemize}
\end{frame}

\begin{frame}{maybe don't xor?}
    \begin{itemize}
    \item these XOR-based constructions are very common
        \begin{itemize}
        \item example: probably used within most connections to websites
        \end{itemize}
    \item there are other ideas, but\ldots
    \item but can still generate meaningful manipulated messages
        \begin{itemize}
        \item usually just need to work on larger units than bits
        \end{itemize}
    \vspace{.5cm}
    \item actual solution: additional \textit{message authentication code}
        \begin{itemize}
        \item sometimes provided pre-combined with encryption and called \textit{authenticated encryption}
        \end{itemize}
    \end{itemize}
\end{frame}
