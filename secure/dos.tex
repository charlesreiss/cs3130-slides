\begin{frame}{denial of service (1)}
    \begin{itemize}
    \item so far: worried about network attacker disrupting confidentiality/authenticity
        \vspace{.5cm}
    \item what if we're just worried about just breaking things
    \item well, if they control network, nothing we can do\ldots 
    \item but often worried about less
    \end{itemize}
\end{frame}

\begin{frame}{denial of service (2)}
    \begin{itemize}
    \item if you just want to inconvenience\ldots
    \item attacker just sends lots of stuff to my server
    \item my server becomes overloaded?
    \item my network becomes overloaded?
    \vspace{.5cm}
    \item but: doesn't this require a lot of work for attacker?
    \item exercise: why is this often not a big obstacle
    \end{itemize}
\end{frame}

\begin{frame}{denial of service: asymmetry}
    \begin{itemize}
    \item work for attacker > work for defender
    \item how much computation per message?
        \begin{itemize}
        \item complex search query?
        \item something that needs tons of memory?
        \item something that needs to read tons from disk?
        \end{itemize}
    \item how much sent back per message?
    \vspace{.5cm}
    \item resources for attacker > resources of defender
    \item how many machines can attacker use?
    \end{itemize}
\end{frame}

\begin{frame}{denial of service: reflection/amplification}
    \begin{itemize}
    \item instead of sending messages directly\ldots
        attacker can send messages ``from'' you to third-party
    \item third-party sends back replies that overwhelm network
    \item example: short DNS query with lots of things in response
    \vspace{.5cm}
    \item ``amplification'' =
        \begin{itemize}
        \item third-party inadvertantly turns small attack into big one
        \end{itemize}
    \end{itemize}    
\end{frame} 
