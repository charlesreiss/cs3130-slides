\begin{frame}{page tricks generally}
\begin{itemize}
\item deliberately \myemph{make program trigger page/protection fault}
\item but \myemph{don't assume page/protection fault is an error}
\vspace{.5cm}
\item have \myemph{seperate data structures} represent logically allocated memory
    \begin{itemize}
    \item e.g. ``addresses {\tt 0x7FFF8000} to {\tt 0x7FFFFFFFF} are the stack''
    \end{itemize}
\item page table is for the hardware and not the OS
\end{itemize}
\end{frame}

\begin{frame}{example page table tricks}
    \begin{itemize}
    \item allocating space on demand
    \item loading code/data from files on disk on demand
    \item saving data temporarily to disk, reloading to memory on demand
        \begin{itemize}
        \item ``swapping''
        \end{itemize}
    \item stopping in a debugger when a variable is modified
    \item detecting whether memory was read/written recently
    \item sharing memory between programs on two different machines
    \item ``copy-on-write'' (later)
    \end{itemize}
\end{frame}

\begin{frame}{hardware help for page table tricks}
\begin{itemize}
\item information about the address causing the fault
    \begin{itemize}
    \item e.g. special register with memory address accessed
    \item harder alternative: OS disassembles instruction, look at registers
    \end{itemize}
\item (by default) rerun faulting instruction when returning from exception
\item precise exceptions: no side effects from faulting instruction or after
    \begin{itemize}
    \item e.g. {\tt pushq} that caused did not change {\tt \%rsp} before fault
    \item e.g. can't notice if instructions were executed in parallel
    \end{itemize}
\end{itemize}
\end{frame}
