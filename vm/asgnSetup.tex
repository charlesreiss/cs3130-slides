\usetikzlibrary{matrix}
\begin{frame}{pagetable assignment}
    \begin{itemize}
    \item pagetable assignment
    \item simulate page tables (on top of normal program memory)
        \begin{itemize}
            \item alternately: implement another layer of page tables \\
                on top of the existing system's
        \end{itemize}
    \vspace{.5cm}
    \item in assignment:
    \item virtual address $\sim$ arguments to your functions
    \item physical address $\sim$ your program addresses (normal pointers)
    \end{itemize}
\end{frame}

\begin{frame}[fragile,label=asgnAPI]{pagetable assignment API}
\lstset{language=C,style=smaller}
\begin{lstlisting}
/* configuration parameters */
#define POBITS ... /* page offset bits */
#define LEVELS /* later */
\end{lstlisting}
\hrule
\begin{lstlisting}
size_t ptbr; // page table base register
    // points to page table (array of page table entries)

// lookup "virtual" address 'va' in page table ptbr points to
// return (~0L) if invalid
size_t translate(size_t va);

// make it so 'va' is valid, allocating one page for its data
// if it isn't already
void page_allocate(size_t va)
\end{lstlisting}
\end{frame}

\begin{frame}{translate()}
    with POBITS=\myemph<2>{12}, LEVELS=1: \\
\begin{tikzpicture}
    \node (ptbreq) {ptbr = GetPointerToTable(};
    \matrix[tight matrix,anchor=west,
        nodes={minimum height=0.5cm},
        row 1/.style={
            nodes={draw=none},
        },
        column 1/.style={nodes={draw=none}},
        column 2/.style={
            nodes={text width=1.2cm}
        },
        column 3/.style={
            nodes={text width=2.2cm}
        },
    ] (tbl) at (ptbreq.east) {
        VPN \& valid? \& physical \\
        0 \& 0 \& --- \\
        1 \& 1 \& 0x9999 \\
        2 \& 0 \& --- \\
        3 \& 1 \& 0x3333 \\
        \ldots \& \ldots \& \ldots \\
    };
    \node[anchor=west] (ptbrparen) at (tbl.east) {)};
    \node[align=left,
    anchor=north west] at (tbl.south -| ptbreq.west) {
        translate(0x0\myemph<2>{FFF}) == (void*) \~~0L \\
        translate(0x1\myemph<2>{000}) == (void*) 0x9999\myemph<2>{000} \\
        translate(0x1\myemph<2>{001}) == (void*) 0x9999\myemph<2>{001} \\
        translate(0x2\myemph<2>{000}) == (void*) \~~0L \\
        translate(0x2\myemph<2>{001}) == (void*) \~~0L \\
        translate(0x3\myemph<2>{000}) == (void*) 0x3333\myemph<2>{000} \\
    };
    \node[anchor=west] (ptbrparen2) at (tbl.east) {)};
\end{tikzpicture}
\end{frame}

\begin{frame}{page\_allocate()}
with POBITS=\myemph<3>{12}, LEVELS=1: \\
\begin{tikzpicture}
    \node (ptbrzero) {ptbr == 0};
    \node[anchor=north west] (pgalloc) at (ptbrzero.south west){page\_allocate(0x1\myemph<3>{000}) \textit{or} page\_allocate(0x1\myemph<3>{001}) \textit{or} \ldots};
    \begin{visibleenv}<2->
    \node[anchor=north west] (ptbreq) at ([yshift=-2cm]pgalloc.south  west) {ptbr \textit{now ==} GetPointerToTable(};
    \matrix[tight matrix,anchor=west,
        nodes={minimum height=0.6cm},
        row 1/.style={
            nodes={draw=none},
        },
        column 1/.style={nodes={draw=none}},
        column 2/.style={
            nodes={text width=1.2cm}
        },
        column 3/.style={
            nodes={text width=2.2cm}
        },
    ] (tbl) at (ptbreq.east) {
        VPN \& valid? \& physical \\
        0 \& 0 \& --- \\
        1 \& 1 \& |[fill=red!10,alias=new]| \myemph<2>{(new)} \\
        2 \& 0 \& --- \\
        3 \& 0 \& --- \\
        \ldots \& \ldots \& \ldots \\
    };
    \node[anchor=west] (ptbrparen2) at (tbl.east) {)};
    \draw[very thick,red,<-] (new) -- ++(0cm, -2cm) node[below] {
        allocated with \texttt{posix\_memalign}
    };
    \end{visibleenv}
\end{tikzpicture}
\end{frame}

\begin{frame}[fragile,label=posixMemalign]{posix\_memalign}
\lstset{
    language=C
}
\begin{lstlisting}
void *result;
error_code =
    posix_memalign(&result, alignment, size);
\end{lstlisting}
\begin{itemize}
\item allocate \texttt{size} bytes
\item choosing address that is multiple of \myemph<2>{\texttt{alignment}}
    \begin{itemize}
    \item can make sure allocation starts at beginning of page
    \end{itemize}
\item \texttt{error\_code} indicates if out-of-memory, etc.
\item fills in \myemph<3>{\texttt{result} (passed via pointer)}
\end{itemize}
\end{frame}

\begin{frame}{parts}
    \begin{itemize}
    \item part 1 (next week): LEVELS=1, POBITS=12 and
        \begin{itemize}
        \item translate() OR
        \item page\_allocate()
        \end{itemize}
    \item part 2 (two weeks after break): all LEVELS, both functions
        \begin{itemize}
        \item in preparation for code review
        \item due Weds BEFORE LAB
        \end{itemize}
    \item part 3 (two weeks after break): final submission
        \begin{itemize}
        \item Friday after code review
        \item \myemph{most of grade} based on this
        \item \myemph{will test previous parts again}
        \end{itemize}
    \end{itemize}
\end{frame}
