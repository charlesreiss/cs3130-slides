\usetikzlibrary{matrix,decorations.pathreplacing}

\begin{frame}<1-5>[fragile,label=classicUnix]{the classic Unix design}
\begin{tikzpicture}
\tikzset{
    user/.style={fill=blue!10},
    iface/.style={fill=black!15},
    kernel/.style={fill=red!10},
    hardware/.style={},
}
% FIXME: show standard libraries/etc. can use HW directly
\matrix[tight matrix,
        nodes={draw,thick,text width=13cm,align=left},
        ] {
    |[minimum height=1cm,alias=apps,user]| applications \\
    |[minimum height=0.5cm,alias=slibiface,iface,text width=11cm,xshift=1cm]| standard library functions / shell commands \\
    |[minimum height=1cm,alias=stdlib,user,text width=11cm,xshift=1cm]| {standard libraries and \\utility programs} \\
    |[minimum height=0.5cm,alias=syscalls,iface,text width=9.5cm,xshift=1.75cm]| system call interface \\
    |[minimum height=2cm,alias=kernel,kernel,text width=9.5cm,xshift=1.75cm]| kernel \\
    |[minimum height=1cm,alias=hwIface,iface]| \hspace{2cm}\only<1>{hardware interface}~ \\
    |[minimum height=1.5cm,alias=hw,hardware]| hardware \\
};
\path[iface] (hwIface.north west) -- ++(3.5cm, 0cm) |- (stdlib.south west) |- (apps.south west) -- cycle;
\path[draw,very thick,black!15] ([xshift=3.5cm]hwIface.north west) -- (hwIface.north west);
\path[draw,thick] ([xshift=3.5cm]hwIface.north west) |- (stdlib.south west) |- (apps.south west) -- (hwIface.north west);
\begin{visibleenv}<2->
\path[draw,very thick,dotted] (kernel.south west) -- ++(0cm, -1cm);
\node[align=center,alt=<3>{red},alt=<6>{red}] at ([xshift=1.5cm,yshift=1cm]hwIface.north west) {
    user-mode \\
    hardware \\
    interface \\
    (limited)
};
\node[anchor=north,alt=<2>{red},alt=<6>{red}] at (kernel.south) {
    kernel-mode hardware interface (complete)
};
\end{visibleenv}
\tikzset{
    component parts/.style={font=\small,anchor=east,align=left}
}
\node[text width=8cm,minimum height=2cm,component parts] at (kernel.east) {
    \begin{tabular}{lll}
    CPU scheduler & filesystems & networking \\
    virtual memory & device drivers & signals \\
    pipes & swapping & \ldots
    \end{tabular}
};
\node[text width=6cm,minimum height=1cm,component parts] at (stdlib.east) {
    \begin{tabular}{ll}
    libc (C standard library) & the shell\\
    login & login\ldots \\
    \end{tabular}
};
\node[text width=10cm,minimum height=1cm,component parts] at (hw.east) {
    \begin{tabular}{lll}
    memory management unit & device controllers & \ldots \\
    \end{tabular}
};
\begin{visibleenv}<4>
\draw[ultra thick,decorate,decoration={brace}] ([xshift=.25cm]slibiface.north east) -- ([xshift=.25cm]kernel.south east) 
    node[midway,right] {the OS?};
\end{visibleenv}
\begin{visibleenv}<5>
\draw[ultra thick,decorate,decoration={brace}] ([xshift=.5cm]syscalls.north east) -- ([xshift=.5cm]kernel.south east) 
    node[midway,right] {the OS?};
\end{visibleenv}
\end{tikzpicture}
\end{frame}

\begin{frame}{aside: is the OS the kernel?}
    \begin{itemize}
    \item OS = stuff that runs in kernel mode?
    \item OS = stuff that runs in kernel mode + libraries to use it?
    \item OS = stuff that runs in kernel mode + libraries + utility programs (e.g. shell, finder)?
    \item OS = everything that comes with machine?
    \vspace{.5cm}
    \item no consensus on where the line is
    \item each piece can be replaced separately\ldots
    \end{itemize}
\end{frame}

