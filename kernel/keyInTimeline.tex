\usetikzlibrary{arrows.meta,patterns}

\begin{frame}{keyboard input timeline}
\myalttext{
\begin{tikzpicture}
\tikzset{
    prog1/.style={draw,fill=cyan},
    prog2/.style={draw,fill=green},
    prog3/.style={draw,fill=violet!30},
    proglabel/.style={font=\tt\scriptsize},
    labelprog1/.style={execute at begin node={\strut },pin={south:\tt\scriptsize read\_input.exe}},
    labelprog2/.style={execute at begin node={\strut }},
    labelprog3/.style={execute at begin node={\strut }},
}
\begin{scope}[xscale=1.5,yscale=1]
\foreach \s/\e/\p [count=\x] in {0/0.5/1,0.5/3/2,3/5.4/3,5.4/7/1} {
    \coordinate (s-\x) at (\s, 0);
    \coordinate (e-\x) at (\e, 0);
    \draw[prog\p] (\s, 0) rectangle (\e, 1) coordinate[midway] (mid-\x);
    \node[anchor=center,proglabel,labelprog\p] at (mid-\x) {};
    \begin{pgfonlayer}{fg}
    \draw[fill=white] ([xshift=-.05cm]e-\x) rectangle ([xshift=.05cm,yshift=1cm]e-\x);
    \draw[pattern=north west lines] ([xshift=-.05cm]e-\x) rectangle ([xshift=.05cm,yshift=1cm]e-\x);
    \end{pgfonlayer}
}
\end{scope}
% FIXME: mark types of exceptions at each transition
\draw[red,very thick,Latex-] ([xshift=-.05cm]e-1) -- (2, -3) node[fill=white,draw] {{\tt read} system call};
\draw[red,very thick,Latex-] ([xshift=-.05cm]e-3) -- (7, -4) node[fill=white,draw] {from keyboard};
\begin{scope}[xshift=2cm]
\draw[fill=white,pattern=north west lines] (0, -1) rectangle (1, -2);
\node[anchor=west] at (1, -1.5) {\strut = operating system};
\end{scope}
\end{tikzpicture}
}{%
Timeline showing processor activity on a keypress. A program called read\_input.exe makes a read system call,
causing the operating system to run other programs for a while. Then, an event from the keyboard triggers
the operating system to run, and it chooses to resume read\_input.exe
}
\end{frame}


