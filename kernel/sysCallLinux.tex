\begin{frame}{Linux x86-64 system calls}
\begin{itemize}
\item special instruction: {\tt syscall}
\item runs OS specified code in kernel mode
\end{itemize}
\end{frame}

\begin{frame}[fragile,label=LinuxSyscall]{Linux syscall calling convention}
\begin{itemize}
\item before {\tt syscall}:
\item \lstinline|%rax| --- system call number
\item \lstinline|%rdi|, \lstinline|%rsi|, \lstinline|%rdx|, \lstinline|%r10|, \lstinline|%r8|, \lstinline|%r9| --- args
\vspace{.5cm}
\item after {\tt syscall}:
\item \lstinline|%rax| --- return value
\item on error: \lstinline|%rax| contains -1 times ``error number''
\vspace{.5cm}
\item \myemph{almost} the same as normal function calls
\end{itemize}
\end{frame}

\begin{frame}{Linux x86-64 hello world}
\lstset{language=myasm,style=small,morekeywords={syscall,movq,.asciz,.globl}}
\lstinputlisting{../kernel/hello_world.s}
\end{frame}

\begin{frame}[fragile,label=sysHandler]{approx. system call handler}
\lstset{
    language=myasm,
    morekeywords={.quad,pushq,jmp},
    escapechar=@,
    style=small,
}
\begin{lstlisting}
sys_call_table:
    .quad handle_read_syscall
    .quad handle_write_syscall
    // ...

handle_syscall:
    ... // save old PC, etc.
    pushq %rcx // save registers
    pushq %rdi
    ...
    call *sys_call_table(,%rax,8)
    ...
    popq %rdi
    popq %rcx
    return_from_exception
\end{lstlisting}
\end{frame}

% FIXME: kernel stack diagram

\begin{frame}{Linux system call examples}
\begin{itemize}
\item {\tt mmap}, {\tt brk} --- allocate memory
\item {\tt fork} --- create new process
\item {\tt execve} --- run a program in the current process
\item {\tt \myemph<2>{\_exit}} --- terminate a process
\item {\tt open}, {\tt \myemph<2>{read}}, {\tt write} --- access files
\item {\tt socket}, {\tt accept}, {\tt getpeername} --- socket-related
\end{itemize}
\end{frame}

