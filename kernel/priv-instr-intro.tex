\begin{frame}{privileged operation: problem}
\begin{itemize}
\item how can hardware (HW) plus operating system (OS) allow:
    \begin{itemize}
    \item read your own files from hard drive
    \end{itemize}
\item but disallow:
    \begin{itemize}
    \item read others files from hard drive
    \end{itemize}
\end{itemize}
\end{frame}

\begin{frame}{some ideas}
\begin{itemize}
\item<1-> OS tells HW `okay' parts of hard drive before running program code
    \begin{itemize}
    \item complex for hardware and for OS
    \end{itemize}
\item<2-> OS verifies your program's code can't do bad hard drive access 
    \begin{itemize}
    \item no work for HW, but complex for OS
    \item may require compiling differently to allow analysis
    \end{itemize}
\item<3-> \myemph<3>{OS tells HW to only allow OS-written code to access hard drive}
    \begin{itemize}
    \item that code can enforce only `good' accesses
    \item requires program code to call OS routines to access hard drive
    \item relatively simple for hardware 
    \end{itemize}
\end{itemize}
\end{frame}


\begin{frame}{kernel mode}
\begin{itemize}
\item extra one-bit register: ``are we in \textit{kernel mode}''
    \begin{itemize}
    \item other names: privileged mode, supervisor mode, \ldots
    \end{itemize}
\item not in kernel mode = \textit{user mode}
\vspace{.5cm}
\item certain operations only allowed in kernel mode
    \begin{itemize}
    \item \textit{privileged instructions}
    \end{itemize}
\item example: talking to any I/O device
\end{itemize}
\end{frame}

\begin{frame}{what runs in kernel mode?}
\begin{itemize}
\item system boots in kernel mode
\item OS switches to user mode to run program code
\vspace{.5cm}
\item next topic: when does system switch back to kernel mode?
    \begin{itemize}
    \item how does OS tell HW where the (trusted) OS code is?
    \end{itemize}
\end{itemize}
\end{frame}
