\usetikzlibrary{matrix}
\begin{frame}{thread versus process state} 
\begin{itemize}
\item thread state 
    \begin{itemize}
    \item registers (including stack pointer, program counter)
    \item \ldots
    \end{itemize}
\item process state
    \begin{itemize}
    \item address space
    \item open files
    \item process id
    \item list of thread states
    \item \ldots
    \end{itemize}
\end{itemize}
\end{frame}

\begin{frame}{process info with threads}
\begin{tikzpicture}
\tikzset{
    pcb/.style={
        tight matrix,
        column 1/.style={nodes={draw,thin,text width=2.5cm,font=\small,minimum height=0.4cm}},
        column 2/.style={nodes={draw,thin,text width=5.5cm,font=\fontsize{9}{10}\tt\selectfont,minimum height=0.4cm}},
    },
    page/.style={
        draw,thick,
        pattern=north west lines,
        minimum width=2cm,
        node contents={~},
    },
    pointer/.style={
        draw,very thick,-Latex,
    },
    pointer light/.style={
        draw,thick,-Latex,
    },
    tall/.style={
        minimum height=0.85cm
    },
    very tall/.style={
        minimum height=2cm
    },
    one pt/.style={
        fill=blue!40,
    },
    one pt line/.style={
        draw=blue!80!black,
    },
    one memory/.style={
        fill=green!40,
    },
    one memory line/.style={
        draw=green!80!black,
    },
    two pt/.style={
        fill=orange!40,
    },
    two pt line/.style={
        draw=orange!80!black,
    },
    two memory/.style={
        fill=violet!40,
        alt=<1-2>{one memory},
    },
    two memory line/.style={
        draw=violet!80!black,
        alt=<1-2>{one memory line},
    },
    fork line/.style={
        draw=black!30,line width=2mm,-{Latex[length=6mm]}
    },
}
\matrix[pcb,label={[font=\small]north:parent process info}] (proc one) {
    |[very tall,draw=red]| thread infos \&
    |[very tall,draw=red]| 
    thread 0: \{PC = 0x123456, rax = 42, rbx = \ldots\} \\
    thread 1: \{PC = 0x584390, rax = 32, rbx = \ldots\} \\
    \\
    page tables \& |[one pt]| ~ \\
    |[tall]| open files \& |[tall]| {fd 0: \ldots \\ fd 1: \ldots } \\
    \ldots \& \ldots \\
};
\end{tikzpicture}
\end{frame}

\begin{frame}{Linux idea: task\_struct}
\begin{itemize}
\item Linux model: single ``task'' structure = thread
\item pointers to address space, open file list, etc.
\item pointers \myemph{can be shared}
    \begin{itemize}
    \item e.g. shared open files: open fd 4 in one task $\rightarrow$ all sharing can use fd 4
    \end{itemize}
\vspace{.5cm}
\item \texttt{fork()}-like system call ``clone'': \myemph{choose what to share}
    \begin{itemize}
    \item \texttt{clone(0, ...)} --- similar to \texttt{fork()}
    \item \texttt{clone(CLONE\_FILES, ...)} --- like fork(), but \textbf{sharing} open files
    \item \texttt{clone(CLONE\_VM, new\_stack\_pointer, ...)} --- like fork(), but \textbf{sharing} address space
    \end{itemize}
\vspace{.5cm}
\item<2-> advantage: no special logic for threads (mostly)
    \begin{itemize}
    \item two threads in same process = tasks sharing everything possible
    \end{itemize}
\end{itemize}
\end{frame}
