\begin{frame}{POSIX/NTFS ACLs}
    \begin{itemize}
    \item more flexible access control lists
    \vspace{.5cm}
    \item list of (user or group, read or write or execute or \ldots)
    \vspace{.5cm}
    \item supported by NTFS (Windows)
    \item a version standardized by POSIX, but usually not supported
    \end{itemize}
\end{frame}

\begin{frame}[fragile,label=posixAclSyntax]{POSIX ACL syntax}
\begin{lstlisting}[language={},style=small]
# group students have read+execute permissions
group:students:r-x
# group faculty has read/write/execute permissions
group:faculty:rwx
# user mst3k has read/write/execute permissions
user:mst3k:rwx
# user tj1a has no permissions
user:tj1a:---

# POSIX acl rule:
    # user take precedence over group entries
\end{lstlisting}
\end{frame}

\begin{frame}[fragile]{POSIX ACLs on command line}
\begin{Verbatim}
getfacl file
\end{Verbatim}
    \vspace{.1cm}
\hrule
\begin{Verbatim}
setfacl -m 'user:tj1a:---' file
\end{Verbatim}
add line to ACL
    \vspace{.1cm}
\hrule
\begin{Verbatim}
setfacl -x 'user:tj1a' file
\end{Verbatim}
REMOVE line from acl
    \vspace{.1cm}
\hrule
\begin{Verbatim}
setfacl -M acl.txt file
\end{Verbatim}
add to acl, but read what to add from a file
    \vspace{.1cm}
\hrule
\begin{Verbatim}
setfacl -X acl.txt file
\end{Verbatim}
remove from acl, but read what to remove from a file
    \vspace{.1cm}
\end{frame}
