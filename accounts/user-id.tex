\begin{frame}{user IDs}
    \begin{itemize}
    \item most common way OSes identify what \textit{domain} process belongs to:
    \vspace{.5cm}
    \item (unspecified for now) procedure sets user IDs
    \item every process has a user ID
    \item user ID used to decide what process is authorized to do
    \end{itemize}
\end{frame}

\begin{frame}[fragile,label=posixUID]{POSIX user IDs}
\begin{lstlisting}[language=C++,style=small]
uid_t geteuid(); // get current process's "effective" user ID
\end{lstlisting}
\begin{itemize}
\item process's user identified with unique number
\item kernel typically only knows about number
\item effective user ID is used for all permission checks
\item also some other user IDs --- we'll talk later
\vspace{.5cm}
\item<2-> standard programs/library maintain number to name mapping
    \begin{itemize}
    \item<2-> \texttt{/etc/passwd} on typical single-user systems
    \item<2-> network database on department machines
    \end{itemize}
\end{itemize}
\end{frame}
