\begin{frame}[fragile,label=xv6SpinlockAcquire]{xv6 spinlock: acquire}
\begin{lstlisting}[
    language=C++,
    style=smaller,
    morekeywords=mfence,
    moredelim={**[is][\btHL<2|handout:2>]{@2}{2@}},
    moredelim={**[is][\btHL<3|handout:3>]{@3}{3@}},
    moredelim={**[is][\btHL<4|handout:4>]{@4}{4@}},
    moredelim={**[is][\btHL<5|handout:5>]{@5}{5@}},
]
void
acquire(struct spinlock *lk)
{
  @2pushcli();2@ // disable interrupts to avoid deadlock.
  ...
  // The xchg is atomic.
  @3while(xchg(&lk->locked, 1) != 0)3@
    ; 

  // Tell the C compiler and the processor to not move loads or stores
  // past this point, to ensure that the critical section's memory
  // references happen after the lock is acquired.
  @4__sync_synchronize();4@
  ...
}
\end{lstlisting}
\begin{tikzpicture}[overlay,remember picture]
\coordinate (place) at ([yshift=1cm]current page.south);
\coordinate (place higher) at ([yshift=3cm]current page.south);
\tikzset{
    box/.style={draw=red,ultra thick,fill=white,align=left,at={(place)},anchor=south},
    box higher/.style={draw=red,ultra thick,fill=white,align=left,at={(place higher)},anchor=south},
}
\begin{visibleenv}<2>
    \node[box] {
        don't let us be interrupted after while have the lock \\
        problem: interruption might try to do something with the lock \\
        \ldots but that can never succeed until we release the lock \\
        \ldots but we won't release the lock until interruption finishes
    };
\end{visibleenv}
\begin{visibleenv}<3>
    \node[box] {
        xchg wraps the lock xchg instruction \\
        same loop as before
    };
\end{visibleenv}
\begin{visibleenv}<4>
    \node[box] {
        avoid load store reordering (including by compiler) \\
        on x86, \texttt{xchg} alone is enough to avoid processor's reordering \\
        (but compiler may need more hints)
    };
\end{visibleenv}
\end{tikzpicture}
\end{frame}

\begin{frame}[fragile,label=xv6SpinlockRelease]{xv6 spinlock: release}
\begin{lstlisting}[
    language=C++,
    style=smaller,
    morekeywords=mfence,
    moredelim={**[is][\btHL<2|handout:2>]{@2}{2@}},
    moredelim={**[is][\btHL<3|handout:3>]{@3}{3@}},
    moredelim={**[is][\btHL<4|handout:4>]{@4}{4@}},
    moredelim={**[is][\btHL<5|handout:5>]{@5}{5@}},
]
void
release(struct spinlock *lk)
  ...
  // Tell the C compiler and the processor to not move loads or stores
  // past this point, to ensure that all the stores in the critical
  // section are visible to other cores before the lock is released.
  // Both the C compiler and the hardware may re-order loads and
  // stores; __sync_synchronize() tells them both not to.
  @2__sync_synchronize();2@

  // Release the lock, equivalent to lk->locked = 0.
  // This code can't use a C assignment, since it might
  // not be atomic. A real OS would use C atomics here.
  @3asm volatile("movl $0, %0" : "+m" (lk->locked) : );3@

  @4popcli();4@
}
\end{lstlisting}
\begin{tikzpicture}[overlay,remember picture]
\coordinate (place) at ([yshift=1cm]current page.south);
\tikzset{
    box/.style={draw=red,ultra thick,fill=white,align=left,at={(place)},anchor=south},
}
\begin{visibleenv}<2>
    \node[box] {
        turns into instruction to tell processor not to reorder \\
        plus tells compiler not to reorder
    };
\end{visibleenv}
\begin{visibleenv}<3>
    \node[box] {
        turns into mov of constant 0 into \lstinline|lk->locked|
    };
\end{visibleenv}
\begin{visibleenv}<4>
    \node[box] {
        reenable interrupts (taking nested locks into account)
    };
\end{visibleenv}
\end{tikzpicture}
\end{frame}
