\begin{frame}[fragile,label=rwlockEffectsExer]{rwlock effects exercise}
\begin{tikzpicture}
\node (first) {
\begin{lstlisting}[language=C++,style=smaller]
pthread_rwlock_t lock;
void ThreadA() {
  pthread_rwlock_rdlock(&lock);
  puts("a");
  ...
  puts("A");
  pthread_rwlock_unlock(&lock);
}
void ThreadB() {
  pthread_rwlock_rdlock(&lock);
  puts("b");
  ...
  puts("B");
  pthread_rwlock_unlock(&lock);
}
\end{lstlisting}
};
\node[anchor=north west] (second) at (first.north east) {
\begin{lstlisting}[language=C++,style=smaller]
void ThreadC() {
  pthread_rwlock_wrlock(&lock);
  puts("c");
  ...
  puts("C");
  pthread_rwlock_unlock(&lock);
}
void ThreadD() {
  pthread_rwlock_wrlock(&lock);
  puts("d");
  ...
  puts("D");
  pthread_rwlock_unlock(&lock);
}
\end{lstlisting}
};
\end{tikzpicture}
exercise: which of these outputs are possible? \\
\begin{tabular}{lll}
1. aAbBcCdD & 2. abABcdDC & 3. cCabBAdD \\ 4. cdCDaAbB & 5. caACdDbB \\
\end{tabular}
\end{frame}
