\usetikzlibrary{arrows.meta,matrix}

\begin{frame}{connecting CPUs and memory}
    \begin{itemize}
    \item multiple processors, common memory
    \item how do processors communicate with memory?
    \end{itemize}
\end{frame}

\begin{frame}{shared bus}
\begin{tikzpicture}
\matrix[
    matrix of nodes,
    nodes in empty cells,
    row 1/.style={nodes={minimum height=1cm,minimum width=1cm}},
    row 2/.style={nodes={draw,rectangle,minimum height=1cm,minimum width=1cm}},
    column sep=5mm,
] (net) {
      \& \& \& \& \&  \\
    CPU1 \& CPU2 \& CPU3 \& CPU4 \& MEM1 \& MEM2\\
};
\foreach \x in {1,2,3,4,5,6} {
    \draw[thick] (net-2-\x.north) -- (net-1-\x.center);
}
\draw[thick,Latex-Latex] (net-1-1.west) -- (net-1-6.east);
\end{tikzpicture}
\begin{itemize}
\item one possible design
    \begin{itemize}
    \item we'll revisit later when we talk about I/O
    \end{itemize}
\item \myemph{tagged} messages --- everyone gets everything, filters
\item contention if multiple communicators
    \begin{itemize}
    \item some hardware enforces only one at a time
    \end{itemize}
\end{itemize}
\end{frame}

\begin{frame}{shared buses and scaling}
    \begin{itemize}
    \item shared buses perform poorly with ``too many'' CPUs
    \item so, there are other designs
        \vspace{.5cm}
    \item we'll gloss over these for now
    \end{itemize}
\end{frame}

\begin{frame}{shared buses and caches}
    \begin{itemize}
    \item remember caches?
    \item memory is \myemph{pretty slow}
    \item each CPU wants to keep local copies of memory
        \vspace{.5cm}
    \item what happens when multiple CPUs cache same memory?
    \end{itemize}
\end{frame}
