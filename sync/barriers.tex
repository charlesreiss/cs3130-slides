\begin{frame}{barriers}
\begin{itemize}
\item compute minimum of 100M element array with 2 processors
\item algorithm:
\vspace{.5cm}
\item compute minimum of 50M of the elements on each CPU
    \begin{itemize}
    \item one thread for each CPU
    \end{itemize}
\item \myemph<2>{wait for all computations to finish}
\item take minimum of all the minimums
\end{itemize}
\end{frame}

\begin{frame}[fragile,label=barrierAPI]{barriers API}
\begin{itemize}
\item barrier.Initialize(NumberOfThreads)
\item barrier.Wait() --- return after all threads have waited
\vspace{.5cm}
\item idea: multiple threads perform computations in parallel
\item threads wait for \myemph{all other threads} to call Wait()
\end{itemize}
\end{frame}

\begin{frame}[fragile,label=barrierWait]{barrier: waiting for finish}
\begin{tikzpicture}
\node[label={north:Thread 0}] (code one) {
\begin{lstlisting}[language=C++,style=smaller]
partial_mins[0] = 
    /* min of first
       50M elems */;

barrier.Wait();


total_min = min(
    partial_mins[0],
    partial_mins[1]
);
\end{lstlisting}
};
\node[anchor=south] (code zero) at ([yshift=1cm]code one.north) {
\begin{lstlisting}[language=C++,style=smaller]
barrier.Initialize(2);
\end{lstlisting}
};

\node[label={north:Thread 1},anchor=north west] (code two) at ([xshift=1cm]code one.north east) {
\begin{lstlisting}[language=C++,style=smaller]


partial_mins[1] =
    /* min of last
       50M elems */
barrier.Wait();
\end{lstlisting}
};
\end{tikzpicture}
\end{frame}

\begin{frame}[fragile,label=barrierReuse]{barriers: reuse}
\begin{tikzpicture}
\node[label={north:Thread 0}] (code one) {
\begin{lstlisting}[
    language=C++,style=smaller,
    moredelim={**[is][\btHL<2|handout:0>]{@2}{2@}},
    moredelim={**[is][\btHL<3|handout:0>]{@3}{3@}},
    moredelim={**[is][\btHL<3|handout:0>]{@4}{4@}},
]
@2results[0][0]2@ = getInitial(0);
barrier.Wait();

@4results[1][0]4@ =
    computeFrom(0, 
        results[0][0],
        @3results[0][1]3@
    );
barrier.Wait();

results[2][0] =
    computeFrom(0,
        results[1][0],
        results[1][1]
    );
\end{lstlisting}
};
\node[label={north:Thread 1},anchor=north west] (code two) at ([xshift=1cm]code one.north east) {
\begin{lstlisting}[
    language=C++,style=smaller,
    moredelim={**[is][\btHL<2|handout:0>]{@2}{2@}},
    moredelim={**[is][\btHL<3|handout:0>]{@3}{3@}},
    moredelim={**[is][\btHL<3|handout:0>]{@4}{4@}},
]
@3results[0][1]3@ = getInitial(1);
barrier.Wait();

results[1][1] =
    computeFrom(1, 
        @2results[0][0],2@
        results[0][1]
    );
barrier.Wait();

results[2][1] =
    computeFrom(1,
        @4results[1][0]4@,
        results[1][1]
    );
\end{lstlisting}
};
\end{tikzpicture}
\end{frame}

\begin{frame}[fragile,label=pthreadBarrier]{pthread barriers}
\begin{lstlisting}[language=C++,style=smaller]
pthread_barrier_t barrier;
pthread_barrier_init(
    &barrier,
    NULL /* attributes */,
    numberOfThreads
);
...
...
pthread_barrier_wait(&barrier);
\end{lstlisting}
\end{frame}

