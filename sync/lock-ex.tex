\begin{frame}[fragile,label=lockEx]{exercise}
    \vspace{-0.5cm}
\begin{lstlisting}[style=smaller]
pthread_mutex_t lock1 = PTHREAD_MUTEX_INITIALIZER;
pthread_mutex_t lock2 = PTHREAD_MUTEX_INITIALIZER;
string one = "init one", two = "init two";
void ThreadA() {
    pthread_mutex_lock(&lock1);
    one = "one in ThreadA";  // (A1)
    pthread_mutex_unlock(&lock1);
    pthread_mutex_lock(&lock2);
    two = "two in ThreadA";  // (A2)
    pthread_mutex_unlock(&lock2);
}
void ThreadB() {
    pthread_mutex_lock(&lock1);
    one = "one in ThreadB";  // (B1)
    pthread_mutex_lock(&lock2);
    two = "two in ThreadB";  // (B2)
    pthread_mutex_unlock(&lock2);
    pthread_mutex_unlock(&lock1);
}
\end{lstlisting}
possible values of one/two after A+B run?
\end{frame}

\begin{frame}<0>[fragile,label=lockExSln]{solution}
\begin{itemize}
\item B1+A2
    \begin{itemize}
    \item A: L(1) A1 U(1) L
    \item B: L(1) B1 L(2) B2 U(2) U(1)
    \item A: L(2) A2 U(2)
    \end{itemize}
\item NOT A1+B2
    \begin{itemize}
    \item would need to run B1 after A1 before A2
    \item not possible because Lock1 held for entire B1+B2 operation
    \item \ldots and A1 needs Lock1, too
    \end{iteimze}
\end{itemize}
\end{frame}

\begin{frame}[fragile,label=lockExAlt1]{exercise (alternate 1)}
    \vspace{-0.5cm}
\begin{lstlisting}[style=smaller]
pthread_mutex_t lock1 = PTHREAD_MUTEX_INITIALIZER;
pthread_mutex_t lock2 = PTHREAD_MUTEX_INITIALIZER;
string one = "init one", two = "init two";
void ThreadA() {
    pthread_mutex_lock(&lock2);
    two = "two in ThreadA";  // (A2)
    pthread_mutex_unlock(&lock2);
    pthread_mutex_lock(&lock1);
    one = "one in ThreadA";  // (A1)
    pthread_mutex_unlock(&lock1);
}
void ThreadB() {
    pthread_mutex_lock(&lock1);
    one = "one in ThreadB";  // (B1)
    pthread_mutex_lock(&lock2);
    two = "two in ThreadB";  // (B2)
    pthread_mutex_unlock(&lock2);
    pthread_mutex_unlock(&lock1);
}
\end{lstlisting}
possible values of one/two after A+B run?
\end{frame}

\begin{frame}[fragile,label=lockExAlt2]{exercise (alternate 2)}
    \vspace{-0.5cm}
\begin{lstlisting}[style=smaller]
pthread_mutex_t lock1 = PTHREAD_MUTEX_INITIALIZER;
pthread_mutex_t lock2 = PTHREAD_MUTEX_INITIALIZER;
string one = "init one", two = "init two";
void ThreadA() {
    pthread_mutex_lock(&lock2);
    two = "two in ThreadA";  // (A2)
    pthread_mutex_unlock(&lock2);
    pthread_mutex_lock(&lock1);
    one = "one in ThreadA";  // (A1)
    pthread_mutex_unlock(&lock1);
}
void ThreadB() {
    pthread_mutex_lock(&lock1);
    one = "one in ThreadB";  // (B1)
    pthread_mutex_unlock(&lock1);
    pthread_mutex_lock(&lock2);
    two = "two in ThreadB";  // (B2)
    pthread_mutex_unlock(&lock2);
}
\end{lstlisting}
possible values of one/two after A+B run?
\end{frame}
