\usetikzlibrary{arrows.meta,matrix,fit}
\begin{frame}{putting it all together (1)}
    \begin{itemize}
    \item 1. \myemph<2>{connect to local wifi network}
        \begin{itemize}
        \item<2-> a. \myemph<2>{ask local network for configuration} --- DHCP
        \item<2-> a. \myemph<2>{find out MAC addresses on local network}
        \end{itemize}
    \item 2. \myemph<3>{open http://foo.com/bar in web browser}
        \begin{itemize}
        \item<3-> a. \myemph<3>{lookup foo.com} --- DNS
        \item<3-> b. \myemph<3>{start connection to foo.com + correct port}
        \item<3-> c. \myemph<3>{translate URL into HTTP message + read response}
        \end{itemize}
    \end{itemize}
\end{frame}

% FIXME: network diagram

\begin{frame}{putting it all together (2)}
    \begin{itemize}
    \item 1. connect to local wifi network
        \begin{itemize}
        \item a. \myemph<2>{ask local network for configuration}
        \end{itemize}
    \vspace{.5cm}
    \item<2-> (DHCP) us -> all on local network: give me an address
    \item<2-> (DHCP) local router -> us: use the following: 
    \begin{tabular}{l|l}
    your IP & 192.0.2.43 \\
    local network & 192.0.2.0 through 192.0.2.255 \\
    gateway to other networks & 192.0.2.1 \\
    DNS server & 198.51.100.34 \\
    valid for & 8 hours (ask later to renew) \\
    \end{tabular}
    \end{itemize}
\end{frame}

\begin{frame}{putting it all together (3)}
    \begin{itemize}
    \item 1. connect to local wifi network
        \begin{itemize}
        \item b. \myemph<2>{find out MAC addresses on local network}
        \end{itemize}
    \vspace{.5cm}
    \item<2-> us -> all local: who has 192.0.2.1 (geteway's IP address)?
    \item<2-> gateway -> us: I am 192.0.2.1, my MAC address is 00:00:5E:00:53:03
    \end{itemize}
\end{frame}

\begin{frame}{putting it all together (4a)}
    \begin{itemize}
    \item 2. open http://foo.com/bar in web browser
        \begin{itemize}
        \item a. \myemph<2>{lookup foo.com}
        \end{itemize}
    \end{itemize}
\begin{visibleenv}<2->
\begin{tikzpicture}
\matrix[draw,alt=<2>{draw=red},ultra thick,label={[label distance=-2mm,draw,fill=white]north:wifi frame}] {
    \node {from: (us)  | to: (gateway) 00:00:5E:00:53:03}; \\[1.2em]
    \node (ip) {from: (us) \myemph<4>{192.0.2.24} | to: (DNS server) 198.51.100.34 }; \\[1.2em]
    \node (udp) {to port: 53 (DNS) | from port: \ldots | message: foo.com address = ???}; \\
};
\begin{pgfonlayer}{bg}
\node[inner sep=-1mm,fit=(ip) (udp),draw,very thick,label={[label distance=-.5mm,inner sep=1mm,draw,fill=white]north:IP packet},fill=yellow!10,alt=<3>{red,fill=red!10}] (ip box) {};
\node[inner sep=-1.5mm,fit=(udp),draw,thick,label={[label distance=-.5mm,inner sep=1mm,draw,fill=white]north:UDP packet},fill=blue!10] {};
\end{pgfonlayer}
\begin{visibleenv}<3>
\draw[ultra thick,red,Latex-] ([xshift=-3cm]ip box.north east) -- ++(.5cm, 1cm) node[above,align=left,fill=white,fill opacity=0.95,draw=red] {copied + forwarded \\ by several routers \\ to get to DNS server};
\end{visibleenv}
\begin{visibleenv}<4>
\draw[ultra thick,red,Latex-] ([xshift=-9cm]ip box.north east) -- ++(.5cm, 1cm) node[above,align=left,fill=white,fill opacity=0.95,draw=red,xshift=3cm] {
    assumption here: our machine's IP is global one \\
    often, instead private --- if so \\
    one router will ``translate'' to public one \\
    (table of public IP+port <-> private IP+port in use)
};
\end{visibleenv}
\end{tikzpicture}
\end{visibleenv}
\end{frame}

\begin{frame}{putting it all together (4c)}
    \begin{itemize}
    \item 2. open http://foo.com/bar in web browser
        \begin{itemize}
        \item a. \myemph<1>{lookup foo.com}
        \end{itemize}
    \vspace{.5cm}
    \item ISP's DNS server receives request
    \item either sends back \myemph<2>{cached response} (if recent, valid one)
    \item or looks up in hierarchy of DNS servers
        \begin{itemize}
        \item ISP server -> root server: who is foo.com
        \item root server -> ISP server: try .com servers at 200.4.3.2
        \item ISP server -> .com servers: \ldots
        \item \ldots
        \end{itemize}
    \end{itemize}
\end{frame}

\begin{frame}{putting it all together (5)}
    \begin{itemize}
    \item 2. \myemph<2>{open http://foo.com/bar in web browser}
        \begin{itemize}
        \item b. \myemph<2>{start connection to foo.com + correct port}
        \end{itemize}
    \vspace{.5cm}
\begin{tikzpicture}
\end{tikzpicture}
    \item<2-> web browser creates socket, asks to connect to foo.com \\
    In OS:
\small
\begin{tabular}{llll}
source port & destination IP & dest port & program/pid/fd \\ \hline
\ldots & \ldots & \ldots & \ldots \\
(OS assigned) & 203.0.113.44 (foo.com) & 80 (http) & browser/705/41 \\
\ldots & \ldots & \ldots & \ldots \\
\end{tabular}
    \item<2-> OS sends message (via multiple routers) to start connection
    \end{itemize}
\end{frame}

\begin{frame}{putting it all together (6)}
    \begin{itemize}
    \item 2. \myemph<2>{open http://foo.com/bar in web browser}
        \begin{itemize}
        \item c. \myemph<2>{translate URL to HTTP message + read response}
        \end{itemize}
    \vspace{.5cm}
    \item<2-> browser: \texttt{write(fd, "GET /bar HTTP/1.1\ldots", \ldots)}
    \item<2-> browser: read response
    \vspace{.5cm}
    \item<2-> message is split into multiple chunks
        \begin{itemize}
        \item<2-> (and forwarded through gateway)
        \end{itemize}
    \item<2-> acknowledgments, resending, etc. done by OSes at both ends
    \end{itemize}
\end{frame}

