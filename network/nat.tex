\begin{frame}{network address translation}
\begin{itemize}
\item IPv4 addresses are kinda scarce
\item solution: \textit{convert} many private addrs. to one public addr.
\item locally: use private IP addresses for local addresses
\item outside: private IP addresses become a single public one
\item commonly how home networks work (and some ISPs)
\end{itemize}
\end{frame}

\begin{frame}{NAT idea}
\begin{tikzpicture}
\tikzset{network/.style={draw,very thick,cloud,aspect=2}}
\node[align=center,network] (outside) at (0, 0){external};
\node[draw,very thick,minimum height=2cm,align=center](router) at (5, 0) {
    router \\
    {\fontsize{9}{10}\tt 203.0.113.43} \\
    {\fontsize{9}{10}\tt 192.168.1.1} 
};
\node[align=center,network] (inside) at (9.5, 0){internal\\\fontsize{9}{10}\tt 192.168.1.*};
\foreach \x/\v in {2/2, 1/3, 0/4, -1/5, -2/6} {
    \node[draw,very thick,circle,
        label={[overlay,font=\fontsize{8}{9}\selectfont]right:192.168.1.\v}
    ] (i-\x) at (12, \x) {};
    \draw (inside) -- (i-\x);
}
\begin{pgfonlayer}{bg}
\begin{scope}[line width=1mm,black!50]
    \draw (outside) -- (router);
    \draw (router) -- (inside);
\end{scope}
\end{pgfonlayer}
\tikzset{
    box/.style={draw,ultra thick,align=center,fill=white,font=\fontsize{13}{14}\selectfont}
}
\node[box] (before) at ([yshift=-3cm]$(outside.east)!0.3!(router)$) {
    1.2.3.4:443 \\ $\rightarrow$ \\ \myemph{203.0.113.43}:54923
};
\node[box] (after) at ([yshift=-3cm]$(router)!0.8!(inside.west)$) {
    1.2.3.4:443 \\ $\rightarrow$ \\ \myemph{192.168.1.4}:39129
};
\draw[dotted,thick] (before.north) -- (before.north |- outside.east);
\draw[dotted,thick] (after.north) -- (after.north |- outside.east);

\end{tikzpicture}
\end{frame}


\begin{frame}{NAT illusion}
    \begin{itemize}
    \item NAT illusion:
    \item private IP address communicating directly with public IP
    \vspace{.5cm}
    \item inside network, talking to outside:
        \begin{itemize}
        \item use private local address
        \item use public remote address
        \item never see router's address
        \end{itemize}
    \item outside network, talking to inside
        \begin{itemize}
        \item use public local address
        \item use router's public address
        \end{itemize}
    \end{itemize}
\end{frame}

\begin{frame}{implementing NAT}
\small
\begin{tabular}{llll}
remote host + port & outside local port number & inside IP & inside port number \\ \hline
128.148.17.3:443 & 54033 & 192.168.1.5 & 43222 \\
11.7.17.3:443 & 53037 & 192.168.1.5 & 33212 \\
128.148.31.2:22 & 54032 & 192.168.1.37 & 43010 \\
128.148.17.3:443 & 63039 & 192.168.1.37 & 32132 \\
\end{tabular}
\begin{itemize}
\item table of the translations
\item need to update as new connections made
\end{itemize}
\end{frame}
