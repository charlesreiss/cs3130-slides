
\begin{frame}{IPv4 addresses and routing tables}
\begin{tikzpicture}
\tikzset{>=Latex},
\node[draw,thick] (router) {router};
\node[aspect=3,draw,cloud,anchor=east,minimum width=2.7cm] (network 1) at ([xshift=-1cm,yshift=.5cm] router.west) {
    network 1
};
\node[aspect=3,draw,cloud,anchor=west,minimum width=2.7cm] (network 2) at ([xshift=1cm,yshift=.5cm] router.east) {
    network 2
};
\node[aspect=3,draw,cloud,anchor=south,minimum width=2.7cm] (network 3) at ([yshift=1cm] router.north) {
    network 3
};
\foreach \x in {1,2,3} { \draw[<->,very thick] (router) -- (network \x); }
\node[anchor=north,font=\small,draw,thick] (table) at ([yshift=-.25cm]router.south) {
\begin{tabular}{l|l}
if I receive data for\ldots & send it to\ldots \\ \hline
128.143.0.0---128.143.255.255 & network 1 \\
192.107.102.0--192.107.102.255 & network 1 \\
\ldots & \ldots \\
4.0.0.0--7.255.255.255 & network 2 \\
64.8.0.0--64.15.255.255 & network 2 \\
\ldots & \ldots \\
anything else & network 3 \\
\end{tabular}
};
\draw[thick,dotted] (router.south west) -- (table.north west);
\draw[thick,dotted] (router.south east) -- (table.north east);
\end{tikzpicture}
\end{frame}


