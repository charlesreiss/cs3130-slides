\begin{frame}[fragile]{HTTP: HyperText Transfer Protocol}
\begin{itemize}
    \item primary application-layer protocol for the Web
    \item standard port: 80 (non-encrypted), 443 (encrypted)
    \item uses TCP (or TCP + TLS for encrypted version)
    \item client-server protocol
        \begin{itemize}
        \item server = always on machine, never initiaites contact
        \item client = sometimes on machine, initiates contact
        \end{itemize}
    \item ex. URL: \texttt{http://www.foo.com/bar}
\end{itemize}
\end{frame}

\begin{FragileFrame}
\frametitle{HTTP: HyperText Transfer Protocol}
\begin{itemize}
    \item \texttt{http://www.foo.com/bar}
    \item HTTP client (example: web browser) connects to www.foo.com
    \item sends something like
\end{itemize}
\begin{Verbatim}
GET /bar HTTP/1.1
Host: www.foo.com
...
\end{Verbatim}
    \begin{itemize}
    \item server replies with status code + (usually) some data
        \begin{itemize}
        \item 200 OK, 403 Unauthorized 404 Not Found, 500 Internal Server Error, \ldots
        \end{itemize}
    \end{itemize}
\end{FragileFrame}

\begin{FragileFrame}
    \frametitle{HTTP miscellany}
    \begin{itemize}
        \item other HTTP commands besides GET
        \item POST --- sending forms
        \item HEAD --- get metadata about file (without getting its data)
        \item PUT, DELETE, \ldots 
        \vspace{.5cm}
        \item all requests/replies have many possible headers
            \begin{itemize}
            \item filetype inforation
            \item login-related information (usually)
            \item metadata used for caching webpages
            \item \ldots
            \end{itemize}
    \end{itemize}
\end{FragileFrame}
