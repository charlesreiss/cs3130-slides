\begin{frame}{upcoming lab}
    \begin{itemize}
    \item request + receive message split into pieces
    \item you are responsible for:
        \begin{itemize}
        \item requesting parts in order
        \item \myemph<2>{resending requests if messages lost/corrupted}
        \end{itemize}
    \item \myemph<2>{``acknowledge'' receiving part X to request part X+1}
    \end{itemize}
\end{frame}

\begin{frame}{protocol}
    \begin{itemize}
    \item {\tt GET$x$} --- retrieve message $x$ (x = 0, 1, 2, or 3)
        \begin{itemize}
        \item other end acknowledges by giving data
        \item if they don't reply, you need to send again
        \item higher numbered messages have errors/etc. that are harder to handle
        \end{itemize}
    \item {\tt ACK$n$ }
        \begin{itemize}
        \item request message part $n + 1$ by acknowledging message part $n$
        \item not quite same purpose as acknowledgments in prior examples
        \item \myemph<2>{(in lab, the response is your `acknowledgment' of your request;} \\
            \myemph<2>{you retry if you don't get it)}
        \end{itemize}
    \end{itemize}
\end{frame}


\begin{frame}[fragile]{callback-based programming (1)}
\begin{Verbatim}[fontsize=\small]
/* library code you don't write */
/* lab: part of waitForAllTimeoutsAndMessagesThenExit() */
void mainLoop() {
    while (notExiting) {
        Event event = waitForAndGetNextEvent();
        if (event.type == RECIEVED) {
            recvd(...);
        } else if (event.type == TIMEOUT) {
            (event.timeout_function)(...);
        }
        ...
    }
}
\end{Verbatim}
\end{frame}

\begin{frame}[fragile]{callback-based programming (2)}
\begin{Verbatim}[fontsize=\small]
/* your code, called by library */
void recvd(...) {
    ...
    setTimeout(..., timerCallback, ...);
}

void timerCallback(...) {
    ...
}

int main() {
    send(.../* first message */);
    ... /* other initial setup */
    waitForAllTimeoutsAndMessagesThenExit(); // runs mainLoop()
}
\end{Verbatim}
\end{frame}

\begin{FragileFrame}
\frametitle{callback-based programing (3)}
\begin{Verbatim}[fontsize=\fontsize{9.5}{10.5},frame=single]
packet = getNextPacket()
doSomething(packet);
sleep(10);
doAnotherThing();
packet = getNextPacket();
doYetAnotherThing(packet);
\end{Verbatim}
turns into code like:
\begin{Verbatim}[fontsize=\fontsize{9.5}{10.5},frame=single]
afterTimeout() {
    doAnotherThing(); mode = 2;
}
recvdPacket(packet) {
    if (mode == 1) {
        doSomething(packet);
        setTimeout(10, afterTimeout);
    } else if (mode == 2)
        doYetAnotherThing(packet);
    } else ...
}
\end{Verbatim}
\end{FragileFrame}


\begin{frame}{callback-based programming uses}
    \begin{itemize}
    \item writing scripts in a webpage
    \item many graphical user interface libraries
    \item sometimes servers that handle lots of connections
    \end{itemize}
\end{frame}
