\begin{frame}{message corrupted}
\begin{itemize}
\item corruption: e.g., a bit flip
	\begin{itemize}
	\item sent: \texttt{I LIKE CATS} (\texttt{49204C494B452043415453})
	\item recv: \texttt{I LIKE BATS} (\texttt{49204C494B4520\myemph{42}415453})
	\end{itemize}
\item instead of sending ``message'', send ``message'' + checksum
	\begin{itemize}
	\item checksum is some calculation using the original message
	\item ex: checksum(\texttt{I LIKE CATS}) = 0xD9
	\item send \texttt{49204C494B452043415453\myemph{D9}}
	\end{itemize}
% \vspace{.5cm}
% \item say Hash(``message'') = 0xABCDEF12
% \item then send ``0xABCDEF12,message''
\vspace{.5cm}
\item receiver then also computes the checksum with the same data
	\begin{itemize}
	\item if matches: keep the message
	\item if does not match: pretend like the message was lost (i.e., drop the message)
	\end{itemize}
% \item pretend message lost if does not match
\end{itemize}
\end{frame}

% \begin{frame}{``checksum''}
% \begin{itemize}
% \item these hashes commonly called ``checksums''
% \item in UDP/TCP, hash function: treat bytes of messages as array of integers; then add integers together
% \end{itemize}
% \end{frame}
