\begin{frame}[fragile,label=pipelines]{pipelines}
\begin{itemize}
\item \verb!./program1 | ./program2!
    \begin{itemize}
    \item program2's input is program1's output
    \end{itemize}
\item example: \verb!ls -1 | grep foo!
    \begin{itemize}
    \item \verb!ls -1!: list files (one per line)
    \item \verb!grep foo!: output all lines that contain ``foo''
    \item \verb!ls -l | grep foo! list all files whose names contain ``foo''
    \end{itemize}
\item similar to \verb!ls -1 >temp1.txt! then \verb! grep foo <temp1.txt!
\begin{itemize}
\item but: no waiting
\end{itemize}
\end{itemize}
\end{frame}

\begin{frame}[fragile,label=pipeFileTypePreview]{pipes}
\begin{itemize}
\item in \verb!ls -1 | grep foo!, what is \verb!ls -1!'s output
\item trick: in POSIX, there are many types of files
    \begin{itemize}
    \item special \textit{kind of file} called a \myemph{pipe}
    \end{itemize}
\end{itemize}
\end{frame}
