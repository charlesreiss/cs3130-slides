
\begin{frame}{fork example}
\lstinputlisting[
    language=C,
    basicstyle=\sourcecodepro\fontsize{9}{10}\selectfont,
    moredelim={**[is][\btHL<2|handout:0>]{@2}{2@}},
    moredelim={**[is][\btHL<3|handout:0>]{@3}{3@}},
    moredelim={**[is][\btHL<4|handout:0>]{@4}{4@}},
]{../unix-api/fork1.c}
\begin{tikzpicture}[overlay,remember picture]
\coordinate (explain location) at ([xshift=-1cm,yshift=-2cm]current page.north east);
\tikzset{
    explain box/.style={
        at={(explain location)},
        anchor=north east,
        draw=red,
        very thick,
        align=left,
        fill=white
    },
}
\begin{visibleenv}<2|handout:0>
\node[explain box] {
\texttt{getpid} --- returns current process pid
};
\end{visibleenv}
\begin{visibleenv}<3|handout:0>
\node[explain box] {
cast in case \texttt{pid\_t} isn't int \\
POSIX doesn't specify (some systems it is, some not\ldots) \\
(not necessary if you were using C++'s cout, etc.)
};
\end{visibleenv}
\begin{visibleenv}<4|handout:0>
\node[explain box] {
prints out \texttt{Fork failed: \textit{error message}} \\
(example \textit{error message}: ``Resource temporarily unavailable'') \\
from error number stored in special global variable \texttt{errno} 
};
\end{visibleenv}
\begin{visibleenv}<5|handout:2>
\node[explain box] {
Example output: \\
\texttt{Parent pid: 100} \\
\texttt{[100] parent of [432]} \\
\texttt{[432] child}
};
\end{visibleenv}
\end{tikzpicture}
\end{frame}


% FIXME: fork_race.c?

% Other process management

