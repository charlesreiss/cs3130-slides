\begin{frame}{wait/waitpid}
\begin{itemize}
\item \texttt{pid\_t waitpid(pid\_t pid, int *status, \\
              \hspace{5cm} int options)}
\item wait for a child process (with pid=\texttt{pid}) to finish
\item sets \texttt{*status} to its ``status information''
\vspace{.5cm}
\item \texttt{pid=-1} $\rightarrow$ wait for any child process instead
\item options? see manual page (command \texttt{man waitpid})
    \begin{itemize}
        \item \texttt{0} --- no options
        %\item \myemph<2>{\texttt{WNOHANG}} --- return 0 rather than hanging if process not yet done
    \end{itemize}
\end{itemize}
\end{frame}

\begin{frame}[fragile,label=exitStatuses]{exit statuses}
\begin{lstlisting}[
    language=C++,
    moredelim={**[is][\btHL<1-|handout:1->]{@1}{1@}},
]
int main() {
    return @101@;  /* or exit(0);  */
}
\end{lstlisting}
\end{frame}

\begin{frame}[fragile,label=waitpidExample]{waitpid example}
\begin{lstlisting}[
    language=C++,
    style=smaller
]
#include <sys/wait.h>
...
  child_pid = fork();
  if (child_pid > 0) {
      /* Parent process */
      int status;
      waitpid(child_pid, &status, 0);
  } else if (child_pid == 0) {
      /* Child process */
      ...
\end{lstlisting}
\end{frame}

\begin{frame}[fragile,label=extractStatus]{the status}
\begin{lstlisting}[
    language=C++,
    style=smaller,
    moredelim={**[is][\btHL<2|handout:0>]{@2}{2@}},
]
#include <sys/wait.h>
...
  waitpid(child_pid, &status, 0);
  if (@2WIFEXITED(status)2@) {
    printf("main returned or exit called with %d\n",
           @2WEXITSTATUS(status)2@);
  } else if (WIFSIGNALED(status)) {
    printf("killed by signal %d\n", WTERMSIG(status));
  } else {
      ...
  }
\end{lstlisting}
``status code'' encodes \myemph{both return value and if exit was abnormal} \\
W* macros to decode it
\end{frame}
