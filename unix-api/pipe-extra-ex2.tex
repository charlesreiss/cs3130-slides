
\begin{frame}<1>[fragile,label=pipeExtraEx2]{exercise}
\vspace{-.25cm}
\begin{lstlisting}[language=C++,basicstyle=\tt\fontsize{9.5}{10.5}\selectfont]
int pipe_fds[2]; pipe(pipe_fds);
pid_t p = fork();
if (p == 0) {
  close(pipe_fds[0]);
  for (int i = 0; i < 10; ++i) {
    char c = '0' + i;
    write(pipe_fds[1], &c, 1);
  }
  exit(0);
}
close(pipe_fds[1]);
char buffer[10];
ssize_t count = read(pipe_fds[0], buffer, 10);
for (int i = 0; i < count; ++i) {
  printf("%c", buffer[i]);
}
\end{lstlisting}
Which of these are possible outputs {\small (if pipe, read, write, fork don't fail)}?
\begin{tabular}{lll}
A. \texttt{0123456789} & B. \texttt{0} & C. (nothing) \\
\myemph<2>{D.} A and B & E. A and C & F. A, B, and C \\
\end{tabular}
\end{frame}

\iftoggle{heldback}{}{\againframe<2>{pipeExtraEx2}}

\begin{frame}<0>[fragile,label=pipeExtraEx2More]{empirical evidence}
\begin{Verbatim}
      8 0
    374 01
    210 012
     30 0123
     12 01234
      3 012345
      1 0123456
      2 01234567
      1 012345678
    359 0123456789
\end{Verbatim}
\end{frame}

\iftoggle{heldback}{}{\againframe<2>{pipeExtraEx2More}}

\begin{frame}{partial reads}
\begin{itemize}
\item read returning 0 always means end-of-file
    \begin{itemize}
    \item by default, read always waits \textit{if no input available yet}
    \item but can set read to return \textit{error} instead of waiting
    \end{itemize}
\item read can return less than requested if not available
    \begin{itemize}
        \item e.g. child hasn't gotten far enough
    \end{itemize}
\end{itemize}
\end{frame}
