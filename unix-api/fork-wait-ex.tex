\begin{frame}[fragile,label=fork-wait-ex1]{exercise (1)}
\vspace{-.25cm}
\begin{lstlisting}[basicstyle=\fontsize{9.5}{10.5}\tt\selectfont]
int main() {
    pid_t pids[2]; const char *args[] = {"echo", "ARG", NULL};
    const char *extra[] = {"L1", "L2"};
    for (int i = 0; i < 2; ++i) {
        pids[i] = fork();
        if (pids[i] == 0) {
            args[1] = extra[i];
            execv("/bin/echo", args);
        }
    }
    for (int i = 0; i < 2; ++i) {
        waitpid(pids[i], NULL, 0);
    }
}
\end{lstlisting}
\newcommand{\shownewline}{ {\fontsize{9}{10}\selectfont(newline)} }
Assuming fork and execv do not fail, which are possible outputs?
\begin{tabular}{llll}
\bfseries A. & \tt L1\shownewline L2 & \bfseries D. & A and B \\
\bfseries B. & \tt L1\shownewline L2\shownewline L2 & \bfseries E. & A and C \\
\bfseries C. & \tt L2\shownewline L1 & \bfseries F. & all of the above \\
    ~ &  ~ & \bfseries G. & something else \\
\end{tabular}
\end{frame}

\iftoggle{heldback}{\providecommand{\correct}[1]{#1}}{\providecommand{\correct}[1]{\myemph<2>{#1}}}

\begin{frame}[fragile,label=fork-wait-ex2]{exercise (2)}
\vspace{-.4cm}
\begin{lstlisting}[basicstyle=\fontsize{9.5}{10.5}\tt\selectfont]
int main() {
    pid_t pids[2]; const char *args[] = {"echo", "0", NULL};
    for (int i = 0; i < 2; ++i) {
        pids[i] = fork();
        if (pids[i] == 0) { execv("/bin/echo", args); }
    }
    printf("1\n"); fflush(stdout);
    for (int i = 0; i < 2; ++i) {
        waitpid(pids[i], NULL, 0);
    }
    printf("2\n"); fflush(stdout);
}
\end{lstlisting}
\newcommand{\shownewline}{ {\normalfont\fontsize{9}{10}\selectfont(newline)} }
Assuming fork and execv do not fail, which are possible outputs?
\begin{tabular}{llll}
\bfseries A. & \tt 0\shownewline 0\shownewline 1\shownewline 2 & \bfseries E. & \correct{A, B, and C} \\
\bfseries B. & \tt 0\shownewline 1\shownewline 0\shownewline 2 & \bfseries F. & C and D \\
\bfseries C. & \tt 1\shownewline 0\shownewline 0\shownewline 2 & \bfseries G. & all of the above \\
\bfseries D. & \tt 1\shownewline 0\shownewline 2\shownewline 0 & \bfseries H. & something else \\
\end{tabular}
\end{frame}
