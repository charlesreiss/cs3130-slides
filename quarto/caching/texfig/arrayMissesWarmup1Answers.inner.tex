\begin{tikzpicture}
    % FIXME: picture
        % array versus cache blocks layout
        % show each each block maps to array, with cache blocks lining up properly
    \foreach \offset in {-2,-1,0,1,2,...,31,32,33} {
        \draw[black!25,thin] (\offset * \mywidth, 0) rectangle ++(\mywidth, -1);
    }
    \node[font=\large,anchor=east] at (-2 * \mywidth, -.5) {\ldots};
    \node[font=\large,anchor=west] at (34 * \mywidth, -.5) {\ldots};
    \foreach \offset/\name in {8/0,12/1,16/2,20/3} {
        \draw[thin,fill=blue!10] (\offset * \mywidth, 0) rectangle ++(\mywidth * 4, -1)
            node[fill=none,opacity=1.0,black,midway,font=\fontsize{9}{10}\tt\selectfont] {array[\name]};
    }
    \begin{visibleenv}<1>
    \node[anchor=north west,align=left] at (0, -1.25) {
        Q1: how do cache blocks correspond to array elements? \\
        not enough information provided!
    };
    \end{visibleenv}
    \begin{visibleenv}<2>
    \node[anchor=north west,align=left] at (0, -1.25) {
        if array[0] starts at beginning of a cache block\ldots \\
        array split across two cache blocks
    };
    \begin{scope}
    \clip (-2.1 * \mywidth, .2) rectangle (34.1 * \mywidth, -1.1);
    \foreach \offset in {-8,0,8,16,24,32} {
        \draw[ultra thick,red!30!black] (\offset * \mywidth, 0)
            rectangle ++(\mywidth * 8, -1);
    }
    \end{scope}
        \draw[very thick, decorate, decoration={brace}] (16 * \mywidth, 0.05) -- ++(8 * \mywidth, 0)
            node[above,midway] {one cache block};
    \matrix[tight matrix,
        nodes={minimum height=0.7cm},
        column 1/.append style={nodes={text width=6cm}},
        column 2/.append style={nodes={text width=6cm,font=\tt}},
        row 1/.append style={nodes={font=\bfseries}},
        anchor=north west] at (0, -3) {
        memory access \& cache contents afterwards \\ 
        --- \& (empty) \\
        read \lstinline|array[0]| (miss) \& \{array[0], array[1]\} \\
        read \lstinline|array[1]| (hit) \& \{array[0], array[1]\} \\
        read \lstinline|array[2]| (miss) \& \{array[2], array[3]\} \\
        read \lstinline|array[3]| (hit) \& \{array[2], array[3]\} \\
    };
    \end{visibleenv}
    \begin{visibleenv}<3>
    \node[anchor=north west,align=left] at (0, -1.25) {
        if array[0] starts right in the middle of a cache block \\
        array split across three cache blocks
    };
    \begin{scope}
    \clip (-2.1 * \mywidth, .2) rectangle (34.1 * \mywidth, -1.1);
    \foreach \offset in {-4,4,12,20,28} {
        \draw[ultra thick,red!30!black] (\offset * \mywidth, 0)
            rectangle ++(\mywidth * 8, -1);
    }
    \end{scope}
        \draw[very thick, decorate, decoration={brace}] (12 * \mywidth, 0.05) -- ++(8 * \mywidth, 0)
            node[above,midway] {one cache block};
    \foreach \offset/\name in {4/****,24/++++} {
        \path (\offset * \mywidth, 0) rectangle ++(\mywidth * 4, -1)
            node[fill=none,opacity=1.0,black,midway,font=\fontsize{9}{10}\tt\selectfont] {\name};
    }
    \matrix[tight matrix,
        nodes={minimum height=0.7cm},
        column 1/.append style={nodes={text width=6cm}},
        column 2/.append style={nodes={text width=6cm,font=\tt}},
        row 1/.append style={nodes={font=\bfseries}},
        anchor=north west] at (0, -3) {
        memory access \& cache contents afterwards \\ 
        --- \& (empty) \\
        read \lstinline|array[0]| (miss) \& \{****, array[0]\} \\
        read \lstinline|array[1]| (miss) \& \{array[1], array[2]\} \\
        read \lstinline|array[2]| (hit) \& \{array[1], array[2]\} \\
        read \lstinline|array[3]| (miss) \& \{array[3], ++++\} \\
    };
    \end{visibleenv}
    \begin{visibleenv}<4>
    \node[anchor=north west,align=left] at (0, -1.25) {
        if array[0] starts at an odd place in a cache block, \\
        need to read two cache blocks to get most array elements
    };
    \begin{scope}
    \clip (-2.1 * \mywidth, .2) rectangle (34.1 * \mywidth, -1.1);
    \foreach \offset in {-7,1,9,17,25,33} {
        \draw[ultra thick,red!30!black] (\offset * \mywidth, 0)
            rectangle ++(\mywidth * 8, -1);
    }
    \end{scope}
        \draw[very thick, decorate, decoration={brace}] (9 * \mywidth, 0.05) -- ++(8 * \mywidth, 0)
            node[above,midway] {one cache block};
    \foreach \offset/\name in {4/****,24/++++} {
        \path (\offset * \mywidth, 0) rectangle ++(\mywidth * 4, -1)
            node[fill=none,opacity=1.0,black,midway,font=\fontsize{9}{10}\tt\selectfont] {\name};
    }
    \matrix[tight matrix,
        nodes={minimum height=0.5cm},
        column 1/.append style={nodes={text width=6cm,font=\small}},
        column 2/.append style={nodes={text width=6cm,align=center,font=\fontsize{8}{9}\selectfont}},
        row 1/.append style={nodes={font=\bfseries}},
        anchor=north west] at (0, -3) {
        memory access \& cache contents afterwards \\ 
        --- \& (empty) \\
        read \lstinline|array[0]| byte 0 (miss) \& \{ ****, array[0] byte 0 \} \\
        read \lstinline|array[0]| byte 1-3 (miss) \& \{ array[0] byte 1-3, array[2], array[3] byte 0 \} \\
        read \lstinline|array[1]| (hit) \&  \{ array[0] byte 1-3, array[2], array[3] byte 0 \} \\
        read \lstinline|array[2]| byte 0 (hit) \& \{ array[0] byte 1-3, array[2], array[3] byte 0 \} \\
        read \lstinline|array[2]| byte 1-3 (miss) \& \{part of array[2], array[3], ++++\} \\
        read \lstinline|array[3]| (hit) \& \{part of array[2], array[3], ++++\} \\
    };
    \end{visibleenv}
\end{tikzpicture}
