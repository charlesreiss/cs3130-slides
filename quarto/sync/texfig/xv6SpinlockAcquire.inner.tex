\begin{tikzpicture}
\node[minimum width=15cm,minimum height=9cm] (current page) {};
\begin{scope}[overlay]

\coordinate (place) at ([yshift=1cm]current page.south);
\coordinate (place higher) at ([yshift=3cm]current page.south);
\tikzset{
    box/.style={draw=red,ultra thick,fill=white,align=left,at={(place)},anchor=south},
    box higher/.style={draw=red,ultra thick,fill=white,align=left,at={(place higher)},anchor=south},
}
\begin{visibleenv}<2>
    \node[box] {
        don't let us be interrupted after while have the lock \\
        problem: interruption might try to do something with the lock \\
        \ldots but that can never succeed until we release the lock \\
        \ldots but we won't release the lock until interruption finishes
    };
\end{visibleenv}
\begin{visibleenv}<3>
    \node[box] {
        xchg wraps the lock xchg instruction \\
        same loop as before
    };
\end{visibleenv}
\begin{visibleenv}<4>
    \node[box] {
        avoid load store reordering (including by compiler) \\
        on x86, \texttt{xchg} alone is enough to avoid processor's reordering \\
        (but compiler may need more hints)
    };
\end{visibleenv}
\end{scope}
\end{tikzpicture}
