\begin{tikzpicture}
\node[minimum width=15cm,minimum height=9cm] (current page) {};
\begin{scope}[overlay]

\coordinate (explain location) at ([xshift=-1cm,yshift=-2cm]current page.north east);
\coordinate (explain location 2) at ([xshift=-1cm,yshift=-6cm]current page.north east);
\tikzset{
    thread/.style={decoration={snake},decorate,thick},
    explain box/.style={
        at={(explain location)},
        anchor=north east,
        draw=red,
        very thick,
        align=left,
        fill=white
    },
    explain box 2/.style={
        at={(explain location 2)},
        anchor=north east,
        draw=red,
        very thick,
        align=left,
        fill=white
    },
}
\begin{visibleenv}<2>
\node[explain box] {
\texttt{getpid} --- returns current process pid
};
\end{visibleenv}
\begin{visibleenv}<3>
\node[explain box] {
cast in case \texttt{pid\_t} isn't int \\
POSIX doesn't specify (some systems it is, some not\ldots) \\
(not necessary if you were using C++'s cout, etc.)
};
\end{visibleenv}
\begin{visibleenv}<4>
\node[explain box] {
prints out \texttt{Fork failed: \textit{error message}} \\
(example \textit{error message}: ``Resource temporarily unavailable'') \\
from error number stored in special global variable \texttt{errno} 
};
\end{visibleenv}
\begin{visibleenv}<5>
\coordinate (thread start) at ([xshift=-6cm,yshift=5cm]explain location 2);
\draw[overlay,thread] (thread start) -- ++(0, -2) node[below] (fork) {fork()} node[midway,right] {parent pid: \ldots};
\draw[thread] (fork) -- ++ (0, -1.5) node[right]{parent of \ldots};
\draw[thick,dotted] (fork) -- ++(4, -.5) coordinate (new thread start);
\draw[thread] (new thread start) -- ++(0, -1.5) node[right] {child \ldots};
\node[explain box 2] {
Example output: \\
\texttt{Parent pid: 100} \\
\texttt{[100] parent of [432]} \\
\texttt{[432] child}
};
\end{visibleenv}
\end{scope}
\end{tikzpicture}
