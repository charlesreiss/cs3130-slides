\date{}
\title{}
\date{}
\begin{document}
\begin{frame}
    \titlepage
\end{frame}

\begin{frame}
    \frametitle{last time [networking]}
    \begin{itemize}
    \item callback-based programming
    \item names versus addresses
    \item DNS = distributed database
        \begin{itemize}
        \item recursive resolvers; caching of responses
        \end{itemize}
    \item IP address format; routing tables
    \item port numbers; 4-tuples to ID connections
    \item HTTP, briefly
    \item network address translation
    \end{itemize}
\end{frame}

\begin{frame}
    \frametitle{last time [secure comm. / 11am]}
    \begin{itemize}
    \item secure communication problem
        \begin{itemize}
        \item same with networks versus non-networks
        \end{itemize}
    \item `principals' for examples
        \begin{itemize}
        \item A, B, C (communicators)
        \item E, M (attackers)
        \end{itemize}
    \item security properties we might want
        \begin{itemize}
        \item confidentiality, authenticity
        \item possibility of other properties
        \end{itemize}
    \end{itemize}
\end{frame}

\begin{frame}
    \frametitle{last time [secure comm. / 2pm]}
    \begin{itemize}
    \item secure communication problem
    \item `principals' for examples
        \begin{itemize}
        \item A, B, C (communicators)
        \item E, M (attackers)
        \end{itemize}
    \item security properties we might want
        \begin{itemize}
        \item confidentiality, authenticity
        \item possibility of other properties
        \end{itemize}
    \item using secrets for confidentiality/authenticity
    \item symmetric encryption, authentication
    \end{itemize}
\end{frame}

\begin{frame}
    \frametitle{why secure communication here [2pm]}
    \begin{itemize}
    \item tools we want you to learn before graduating
        \begin{itemize}
        \item CSO2 was the course selected to put it in the CS degrees
        \end{itemize}
    \item won't be going over actual math
        \begin{itemize}
        \item some professional mathematician does that
        \item yes, which algorithm matters, but\ldots
        \item \ldots not the most common cause of security problems
        \end{itemize}
    \item goal: not shoot yourself in the foot
    \end{itemize}
\end{frame}

\begin{frame}
    \frametitle{what authenticity means [2pm]}
    \begin{itemize}
    \item \textit{authenticity} can be confusing concept
    \item we've said MAC shows ``message really came from (A or B)''
    \vspace{.5cm}
    \item authenticity means message really came from correct party \\
        \textit{including being an unchanged message they intended to send}
    \item problem we'll encounter using MACs:
        \begin{itemize}
        \item MAC needs to `cover' message contents
        \item MAC needs to `cover' relevant context
        \end{itemize}
    \end{itemize}
\end{frame}

\subsection{motivation: need for authentication}
\begin{frame}{secrecy properties}
    \begin{itemize}
    \item actually that's not secret enough, usually want to resist \\
        recovery of info about message \myemph<2>{or key} even given\ldots
    \vspace{.5cm}
    \item \myemph<3>{partial info about the message}, or
    \item \myemph<4>{lots of other (message, ciphertext) pairs}, or
        \begin{itemize}
        \item ``known plaintext''
        \end{itemize}
    \item lots of (message, ciphertext) pairs for \textit{\myemph<5>{other messages the attacker chooses}}, or
        \begin{itemize}
        \item ``chosen plaintext''
        \end{itemize}
    \item lots of (message, ciphertext) pairs encrypted under similar keys, or
        \begin{itemize}
        \item ``related key''
        \end{itemize}
    \item \ldots
    \end{itemize}
\end{frame}
\begin{frame}{encryption is not enough}
    \begin{itemize}
    \item if B receives an encrypted message from A, and\ldots
    \item it makes sense when decrypted, why isn't that good enough?
    \vspace{.5cm}
    \item problem: an active attacker M \\
        can \textit{selectively} manipulate the encrypted message
    \end{itemize}
\end{frame}

\begin{frame}{manipulating encrypted data?}
\begin{itemize}
\item one example: common symmetric encryption approach:
    \begin{itemize}
    \item use random number + shared secret to\ldots
    \item produce sequence of hard-to-guess bits $x_i$ as long as the message
    \item produce ciphertext with xor: $c_i = m_i \oplus x_i$
    \item message = $m_0m_1m_2\ldots$; ciphertext = [random number]$c_0c_1c_2\ldots$
    \end{itemize}
\item means that flipping $c_i$ flips bit $m_i$ 
\item also means that we can shorten messages silently
\end{itemize}
\end{frame}

\begin{frame}{manipulating messages}
\begin{itemize}
\item as an active attacker
\vspace{.5cm}
\item if we know part of plaintext \\
    can make it read anything else by flipping bits
    \begin{itemize}
    \item ``Pay \$100 to Bob'' $\rightarrow$ ``Pay \$999 to Bob''
    \end{itemize}
\item we can shorten 
    \begin{itemize}
    \item ``Pay \$100 to ABC Corp if they \ldots'' $\rightarrow$ ``Pay \$100 to ABC Corp''
    \end{itemize}
\item we can corrupt and what the response is
    \begin{itemize}
    \item e.g. what changes don't make B reject message as malformed?
    \end{itemize}
\end{itemize}
\end{frame}


\section{tools with shared keys}
\subsection{secrets generally}
\begin{frame}{secrets}
    \begin{itemize}
    \item if A is talking to B are communicating, \\
        what stops M from pretending to be B?
    \vspace{.5cm}
    \item assumption: B knows some \myemph{secret information} that M does not
    \vspace{.5cm}
    \item<2-> start: assume A and B have a \textit{shared secret} they both know
        \begin{itemize}
        \item (and M, E do not)
        \end{itemize}
    \item<2-> (later: easier to setup assumptions)
    \end{itemize}
\end{frame}

\begin{frame}{bad ways to use shared secret}
    \begin{itemize}
    \item A $\rightarrow$ B: What's the password?
    \item B $\rightarrow$ A: It's `\texttt{Abc\$xyM\$e}'.
    \item A $\rightarrow$ B: That's right! Here's my confidential information.
    \vspace{.5cm}
    \item<2-> well, this doesn't really help: 
        \begin{itemize}
        \item against E, who can read the password AND confidential info
        \item against M, who can also pretend to be A for B
        \end{itemize}
    \end{itemize}
\end{frame}


\subsection{symmetric encryption}
\begin{frame}{symmetric encryption}
    \begin{itemize}
    \item some magic math!
    \vspace{.5cm}
    \item we'll be given two functions by expert:
        \begin{itemize}
        \item encrypt: $E(\text{key}, \text{message}) = \text{ciphertext}$
        \item decrypt: $D(\text{key}, \text{ciphertext}) = \text{message}$
        \end{itemize}
    \item key = shared secret
        \begin{itemize}
        \item ideally small (easy to share) and chosen at random
        \item unsolved problem: how to share it?
        \end{itemize}
    \end{itemize}
\end{frame}

\begin{frame}{symmetric encryption properties (1)}
    \begin{itemize}
    \item our functions:
        \begin{itemize}
        \item encrypt: $E(\text{key}, \text{message}) = \text{ciphertext}$
        \item decrypt: $D(\text{key}, \text{ciphertext}) = \text{message}$
        \end{itemize}
    \item knowing $E$ and $D$, it should be hard to \\
         \myemph<2>{learn anything about the message} from the ciphertext without key
    \item ``hard'' $\approx$ would have to try every possible key
    \end{itemize}
\end{frame}


\begin{frame}{secrecy properties}
    \begin{itemize}
    \item actually that's not secret enough, usually want to resist \\
        recovery of info about message \myemph<2>{or key} even given\ldots
    \vspace{.5cm}
    \item \myemph<3>{partial info about the message}, or
    \item \myemph<4>{lots of other (message, ciphertext) pairs}, or
        \begin{itemize}
        \item ``known plaintext''
        \end{itemize}
    \item lots of (message, ciphertext) pairs for \textit{\myemph<5>{other messages the attacker chooses}}, or
        \begin{itemize}
        \item ``chosen plaintext''
        \end{itemize}
    \item lots of (message, ciphertext) pairs encrypted under similar keys, or
        \begin{itemize}
        \item ``related key''
        \end{itemize}
    \item \ldots
    \end{itemize}
\end{frame}

\begin{frame}{using?}
    \begin{itemize}
    \item in advance: A and B share encryption key
    \vspace{.5cm}
    \item A computes $E$(key, `The secret formula is\ldots') = ***
    \item send on network: \\
    A $\rightarrow$ B: ***
    \item<2-> B computes $D$(key, ***) = `The secret formula is \ldots'
    \end{itemize}
\end{frame}

\begin{frame}{symmetric encryption procedure}
    \begin{itemize}
    \item in advance: A and B share encryption key
    \vspace{.5cm}
    \item A computes $E$(key, `The secret formula is\ldots') = ***
    \item send on network: \\
    A $\rightarrow$ B: ***
    \item<2-> B computes $D$(key, ***) = `The secret formula is \ldots'
    \end{itemize}
\end{frame}

\subsection{aside: terminology}
\begin{frame}{calling things encryption}
    \begin{itemize}
    \item in this class, \textit{(symmetric) encryption} means condidentiality but not authenticity
        \begin{itemize}
        \item has malleability problme
        \end{itemize}
    \item matches most common thing a library calls encryption
    \vspace{.5cm}
    \item but, sometimes encryption will be\ldots
        \begin{itemize}
        \item ``authenticated encryption'' = encryption + message authentication code, or
        \item some lower level tool (similar to $f$ function earlier) that needs extra steps to encrypt messages securely
        \end{itemize}
    \end{itemize}
\end{frame}


\subsection{message authentication codes}
\begin{frame}{message authentication codes (MACs)}
    \begin{itemize}
    \item goal: use shared secret \textit{key} to verify message origin
    \vspace{.5cm}
    \item one function: $MAC(\text{key}, \text{message}) = \text{tag}$
    \item knowing $MAC$ and the message and the tag, it should be hard to:
        \begin{itemize}
        \item find the value of $MAC(\text{key}, \text{other message})$
            \begin{itemize}
            \item ``forging the tag''
            \end{itemize}
        \item find the key
        \end{itemize}
    \end{itemize}
\end{frame}


% FIXME: place for later sectoin
\section{exercise}
\begin{frame}{exercise}
    \begin{itemize}
    \item suppose A, B have shared keys $K_1,K_2$
        \begin{itemize}
        \item assume attackers do not have keys
        \end{itemize}
    \item E/D = encrypt/decrypt function
    \item A asks B to pay Sue \$100 by sending message with these parts:
        \begin{itemize}
        \item ``2023-11-03: pay \$100''
        \item $E(K_1, \text{``2023-11-03 Sue''})$
        \item $MAC(K_2, \text{``2023-11-03: pay \$100''})$
        \end{itemize}
    \item 1. can eavesdropper learn: (a) who is being paid, (b) how much?
    \item 2. can machine-in-middle change: (a) who is being paid, (b) how much?
    \end{itemize}
\end{frame}


\section{motivation: distributing shared secrets?}
\begin{frame}[fragile,label=sharedSecretProblem]{shared secrets impractical}
    \begin{itemize}
    \item problem: shared secrets usually aren't practical
    \vspace{.5cm}
    \item \myemph<2>{need secure communication before I can do secure communication?}
    \item \myemph<3>{scaling problems}
        \begin{itemize}
        \item millions of websites $\times$ billions of browsers = how many keys?
        \item hard to talk to new people
        \end{itemize}
    \end{itemize}
\end{frame}

% \begin{frame}{bootstrapping keys?}
%     \begin{itemize}
%     \item will still need to have some sort of secure communication to setup!
%     \item because we need some way to know we aren't talking to attacker
%     \item<2-> but\ldots
%     \vspace{.5cm}
%     \item<3-> \myemph<3>{can be broadcast communication}
%         \begin{itemize}
%         \item don't need full new sets of keys for each web browser
%         \end{itemize}
%     \item<4-> \myemph<4>{only with smaller number of trusted authorities}
%         \begin{itemize}
%         \item don't need to have keys for every website in advance
%         \end{itemize}
%     \end{itemize}
% \end{frame}


\begin{frame}{alternative encryption scheme}
    \begin{itemize}
    \item want to avoid needing to pre-share secret keys
    \vspace{.5cm}
    \item but, ok making some sacrifices:
    \item<2-> \myemph<2>{confidentiality in only one direction}
        \begin{itemize}
        \item e.g., one set of keys only enables secure communication from A $\rightarrow$ B
        \end{itemize}
    \item<3-> \myemph<3>{slower performance}
        \begin{itemize}
        \item encryption and decryption can take longer than symmetric encryption
        \end{itemize}
    \end{itemize}
\end{frame}


\section{tools without shared keys}

\subsection{asymmetric encryption}

\begin{frame}{asymmetric encryption}
\begin{itemize}
\item we'll have two functions:
    \begin{itemize}
    \item encrypt: $PE(\text{public key}, \text{message}) = \text{ciphertext}$
    \item decrypt: $PD(\text{private key}, \text{ciphertext}) = \text{message}$
    \end{itemize}
\item (public key, private key) = ``key pair''
\end{itemize}
\end{frame}

\begin{frame}{key pairs}
    \begin{itemize}
    \item `private key' = kept secret
        \begin{itemize}
        \item usually not shared with \textit{anyone}
        \end{itemize}
    \item `public key' = safe to give to everyone
        \begin{itemize}
        \item usually some hard-to-reverse function of public key
        \end{itemize}
    \vspace{.5cm}
    \item concept will appear in some other cryptographic primitives
    \end{itemize}
\end{frame}

\begin{frame}{asymmetric encryption properties}
\begin{itemize}
\item functions:
    \begin{itemize}
    \item encrypt: $PE(\text{public key}, \text{message}) = \text{ciphertext}$
    \item decrypt: $PD(\text{private key}, \text{ciphertext}) = \text{message}$
    \end{itemize}
\item should have:
    \begin{itemize}
    \item knowing $PE$, $PD$, the public key, and ciphertext shouldn't make it too easy to find message
    \item knowing $PE$, $PD$, the public key, ciphertext, and message shouldn't help in finding private key
    \end{itemize}
\end{itemize}
\end{frame}

\begin{frame}{secrecy properties with asymmetric}
    \begin{itemize}
    \item not going to be able to make things as hard as ``try every possibly private key''
    \item but going to make it impractical
    \vspace{.5cm}
    \item like with symmetric encryption want to prevent recovery of \textit{any info about message}
    \item also have some other attacks to worry about:
        \begin{itemize}
        \item e.g. no info about key should be revealed based on our reactions to decrypting maliciously chosen ciphertexts
        \end{itemize}
    \end{itemize}
\end{frame}

\begin{frame}{using asymmetric v symmetric}
\begin{itemize}
    \item both:
        \begin{itemize}
            \item use secret data to generate key(s)
        \end{itemize}
    \item asymmetric (AKA public-key) encryption
        \begin{itemize}
            \item one ``keypair'' per recipient
            \item private key kept by recipient
            \item public key sent to all potential senders
            \item encryption is one-way without private key
        \end{itemize}
    \item symmetric encryption
        \begin{itemize}
            \item one key per (recipient + sender)
            \item secret key kept by recipient + sender
            \item if you can encrypt, you can decrypt
        \end{itemize}
\end{itemize}
\end{frame}

\begin{frame}{using?}
    \begin{itemize}
    \item in advance: B generates private key + public key
    \item in advance: B sends public key to A (and maybe others) securely
    \vspace{.5cm}
    \item A computes $PE$(public key, `The secret formula is\ldots') = *******
    \item send on network: \\
    A $\rightarrow$ B: ********
    \item B computes $PD$(private key, *******) = `The secret formula is \ldots'
    \end{itemize}
\end{frame}


\subsection{digital signatures}

\begin{frame}{digital signatures}

symmetric encryption : asymetric encryption :: \\
message authentication codes : digital signatures
\end{frame}

\begin{frame}{digital signatures}
    \begin{itemize}
    \item pair of functions:
        \begin{itemize}
        \item sign: $S(\text{private key}, \text{message}) = \text{signature}$
        \item verify: $V(\text{public key}, \text{signature}, \text{message}) = 1$ (``yes, correct signature'') 
        \end{itemize}
    \item (public key, private key) = key pair (like asymmetric encryption)
        \begin{itemize}
        \item public key can be shared with everyone
        \item knowing $S$, $V$, $\text{public key}$, $\text{message}$, $\text{signature}$ \\
            doesn't make it too easy to find another message + signature so that\\
            $V(\text{public key}, \text{other message}, \text{other signature}) = 1$
        \end{itemize}
    \end{itemize}
\end{frame}



 \section{encryption + authentication pitfalls}
 \begin{frame}{tools, but...}
    \begin{itemize}
    \item have building blocks, but less than straightforward to use
    \vspace{.5cm}
    \item lots of issues from using building blocks poorly
    \item start of art solution: formal proof sytems
    \end{itemize}
\end{frame}


\subsection{replay attacks}
\begin{frame}{replay attacks}
    \begin{itemize}
    \item A$\rightarrow$B: Did you order lunch? [signed by A]
    \item B$\rightarrow$A: Yes. [signed by B]
    \item A$\rightarrow$B: Vegetarian? [signed by A]
    \item B$\rightarrow$A: No, not this time. [signed by B]
    \item \ldots
    \item A$\rightarrow$B: There's a guy at the door, says he's here to repair the TV. Should I let him in? [signed by A]
    \item how can attacker hijack the reponse to A's inquiry?
    \vspace{.5cm}
    \item<2-> as an attacker, I can copy/paste B's earlier message!
        \begin{itemize}
        \item it's still signed!
        \end{itemize}
    \end{itemize}
\end{frame}

\begin{frame}{nonces}
    \begin{itemize}
    \item one solution to replay attacks:
    \item A$\rightarrow$B: \myemph{\#1} Did you order lunch? [signed by A]
    \item B$\rightarrow$A: \myemph{\#1} Yes. [signed by B]
    \item A$\rightarrow$B: \myemph{\#2} Vegetarian? [signed by A]
    \item B$\rightarrow$A: \myemph{\#2} No, not this time. [signed by B]
    \item \ldots
    \item A$\rightarrow$B: \myemph{\#54} There's a guy at the door, says he's here to repair the TV. Should I let him in? [signed by A]
    \vspace{.5cm}
    \item (assuming A actually checks the numbers)
    \end{itemize}
\end{frame}



 \subsection{other attacks}

 \begin{frame}{other attacks without breaking math}
     \begin{itemize}
         \item cautionary tale: \\
             it's easy to accidentally use secure encryption/signature/MAC/etc. algorithm \\ in a way that is very insecure
     \end{itemize}
 \end{frame}

 \begin{frame}{TLS state machine attack}
    \begin{itemize}
    \item from \url{https://mitls.org/pages/attacks/SMACK}
    \item protocol:
        \begin{itemize}
        \item step 1: verify server identity
        \item step 2: receive messages from server
        \end{itemize}
    \item attack:
        \begin{itemize}
        \item if server sends ``here's your next message'', \\
            instead of ``here's my identity'' \\
        \item then broken client ignores verifying server's identity
        \end{itemize}
    \end{itemize}
\end{frame}



\section{on the lab}
\begin{frame}{on the lab}
\end{frame}

\section{certificate authorities}
\begin{frame}{getting public keys?}
    \begin{itemize}
    \item browser talking to websites \\
    needs public keys of every single website?
    \vspace{.5cm}
    \item not really feasible, but\ldots
    \end{itemize}
\end{frame}

\begin{frame}{certificate authorities}
    \begin{itemize}
    \item instead, have public keys of trusted \textit{certificate authorities}
    \item only 10s of them, probably
    \vspace{.5cm}
    \item websites go to certificates authorities with their public key
    \item certificate authorities sign messages like: \\
        ``The public key for foo.com is XXX.''
    \item these signed messages called ``certificates''
    \end{itemize}
\end{frame}

\begin{frame}[fragile]{example web certificate (1)}
\begin{Verbatim}[fontsize=\scriptsize]
Certificate:
    Data:
        Version: 3 (0x2)
        Serial Number:
            81:13:c9:49:90:8c:81:bf:94:35:22:cf:e0:25:20:33
        Signature Algorithm: sha256WithRSAEncryption
        Issuer:
            commonName                = InCommon RSA Server CA
            organizationalUnitName    = InCommon
            organizationName          = Internet2
            localityName              = Ann Arbor
            stateOrProvinceName       = MI
            countryName               = US
        Validity
            Not Before: Feb 28 00:00:00 2022 GMT
            Not After : Feb 28 23:59:59 2023 GMT
        Subject:
            commonName                = collab.its.virginia.edu
            organizationalUnitName    = Information Technology and Communication
            organizationName          = University of Virginia
            stateOrProvinceName       = Virginia
            countryName               = US
.....
\end{Verbatim}
\end{frame}

\begin{frame}[fragile]{example web certificate (1)}
\begin{Verbatim}[fontsize=\scriptsize]
Certificate:
    Data:
....
        Subject Public Key Info:
            Public Key Algorithm: rsaEncryption
                RSA Public-Key: (2048 bit)
                Modulus:
                    00:a2:fb:5a:fb:2d:d2:a7:75:7e:eb:f4:e4:d4:6c:
                    94:be:91:a8:6a:21:43:b2:d5:9a:48:b0:64:d9:f7:
                    f1:88:fa:50:cf:d0:f3:3d:8b:cc:95:f6:46:4b:42:
....
        X509v3 extensions:
....
            X509v3 Extended Key Usage: 
                TLS Web Server Authentication, TLS Web Client Authentication
....
            X509v3 Subject Alternative Name: 
                DNS:collab.its.virginia.edu
                DNS:collab-prod.its.virginia.edu
                DNS:collab.itc.virginia.edu
    Signature Algorithm: sha256WithRSAEncryption
         39:70:70:77:2d:4d:0d:0a:6d:d5:d1:f5:0e:4c:e3:56:4e:31:
....
\end{Verbatim}
\end{frame}

\begin{frame}{certificate chains}
    \begin{itemize}
    \item That certificate signed by ``InCommon RSA Server CA''
    \item CA = certificate authority
    \vspace{.5cm}
    \item so their public key, comes with my OS/browser?
        \begin{itemize}
        \item not exactly\ldots
        \end{itemize}
    \item they have their own certificate signed by ``USERTrust RSA Certification Authority''
    \item and their public key comes with your OS/browser?
    \vspace{.5cm}
    \item (but both CAs now operated by UK-based Sectigo)
    \end{itemize}
\end{frame}


\subsection{how certificates verified}
\begin{frame}{exercise}
    \begin{itemize}
    \item exercise: how should certificates verify identity?
    \end{itemize}
\end{frame}

\begin{frame}{how do certificate authorities verify}
    \begin{itemize}
        \item for web sites, set by CA/Browser Forum
        \item organization of:
            \begin{itemize}
            \item everyone who ships code with list of valid certificate authorities
                \begin{itemize}
                \item Apple, Google, Microsoft, Mozilla, Opera, Cisco, Qihoo 360, Brave, \ldots
                \end{itemize}
            \item certificate authorities
            \end{itemize}
        \item decide on rules (``baseline requirements'') for what CAs do
    \end{itemize}
\end{frame}

\begin{frame}{BR domain name identity validation}
    \begin{itemize}
        \item options involve CA choosing random value and:
        \vspace{.5cm}
        \item sending it to domain contact (with domain registrar) and receive response with it, or
        \item observing it placed in DNS or website or sent from server in other specific way
        \vspace{1cm}
        \item exercise: problems this doesn't deal with?
    \end{itemize}
\end{frame}

\begin{frame}{some other things public CAs do}
    \begin{itemize}
    \item keep their private keys in tamper-resistant hardware
    \item maintain publicly-accessible database of \textit{revoked} certificates
        \begin{itemize}
        \item some browsers check these
        \end{itemize}
    \item certificate transparency
        \begin{itemize}
            \item public logs of every certificate issued
            \item some browsers reject non-logged certificates
            \item so you can tell if bad certificate exists for your website
        \end{itemize}
    \item `CAA' records in the domain name system
        \begin{itemize}
        \item can indicate which CAs are allowed to issue certificates in DNS
        \item (but CAs apparently not required to use DNSSEC (certificate infrastructure for signing domain name records) when looking this up)
        \end{itemize}
    \end{itemize}
\end{frame}


\section{preview: additional tools}
\begin{frame}{additional crypto tools needed for web security}
    \begin{itemize}
    \item cryptographic hash functions (summarize data)
    \item `secure' random numbers
    \item key agreement
    \end{itemize}
\end{frame}

\section{cryptographic hashes}

% FIXME: hashes
\begin{frame}{motivation for hashes: summary for signature}
    \begin{itemize}
    \item digital signatures typically have size limit
    \item \ldots but we want to sign very large messages
    \vspace{.5cm}
    \item solution: get secure ``summary'' of message
    \end{itemize}
\end{frame}

\begin{frame}{cryptographic hash}
    \begin{itemize}
    \item hash(M) = X
    \vspace{.5cm}
    \item given X: hard to find message other than by guessing
    \item given X, M: hard to find second message so that hash(second message) = X
    \vspace{.5cm}
    \item example uses:
        \begin{itemize}
        \item substitute for original message in digital signature
        \item building message authentication codes
        \end{itemize}
    \item example algorithm: SHA256
    \end{itemize}
\end{frame}



\section{random numbers}
\begin{frame}{random numbers}
    \begin{itemize}
    \item need a lot of keys that no one else knows
    \vspace{.5cm}
    \item common task: choose a \textit{random} number
    \item question: what does \textit{random} mean here?
    \end{itemize}
\end{frame}

\begin{frame}{cryptographically secure random numbers}
    \begin{itemize}
        \item security properties we might want for random numbers:
        \vspace{.5cm}
    \item attacker cannot guess (part of) number better than chance
    \item knowing prior `random' numbers shouldn't help predict next `random' numbers
    \item compromising machine now shouldn't reveal older random numbers
    \end{itemize}
\end{frame}

\begin{frame}{exercise: how to generate?}
\end{frame}

\begin{frame}{/dev/urandom}
    \begin{itemize}
    \item Linux kernel random number generator
    \vspace{.5cm}
    \item collects ``entropy'' from hard-to-predict events
        \begin{itemize}
        \item e.g. exact timing of I/O interrupts
        \item e.g. some processor's built-in random number circuit
        \end{itemize}
    \item turned into as many random bytes as you want
    \end{itemize}
\end{frame}

\begin{frame}{turning `entropy' into random bytes}
    \begin{itemize}
    \item lots of ways to do this; one (rough/incomplete) idea:
    \item internal variable \textit{state}
    \item to add `entropy'
        \begin{itemize}
        \item state $\leftarrow$ SecureHash(state + entropy)
        \end{itemize}
    \item to extract value:
        \begin{itemize}
        \item random bytes $\leftarrow$ SecureHash(1 + state) \\
            \small give bytes that can't be reversed to compute state
                \vspace{.5cm}
        \item state $\leftarrow$ SecureHash(2 + state) \\
            \small change state so attacker can't take us back to old state if compromised
        \end{itemize}
    \end{itemize}
\end{frame}


\section{key agreement}
\begin{frame}{just asymmetric?}
    \begin{itemize}
    \item given public-key encryption + digital signatures\ldots
    \item why bother with the symmetric stuff?
    \vspace{.5cm}
    \item symmetric stuff much faster
    \item symmetric stuff much better at supporting larger messages
    \end{itemize}
\end{frame}

\begin{frame}{key agreement}
    \begin{itemize}
    \item problem: A has B's public encryption key \\
        wants to choose shared secret 
    \vspace{.5cm}
    \item some ideas:
        \begin{itemize}
        \item A chooses a key, sends it encrypted to B
        \item A sends a public key encrypted B, B chooses a key and sends it back
        \end{itemize}
    \item<2-> alternate model:
        \begin{itemize}
        \item both sides generate random values
        \item derive public-key like ``key shares'' from values
        \item use math to combine ``key shares''
        \item kinda like A + B both sending each other public encryption keys
        \end{itemize}
    \end{itemize}
\end{frame}

\begin{frame}{Diffie-Hellman key agreement (2)}
\begin{itemize}
\item A and B want to agree on shared secret
\vspace{.5cm}
\item A chooses random value Y
\item A sends public value derived from Y (``key share'')
\item B chooses random value Z
\item B sends public value derived from Z (``key share'')
\item A combines Y with public value from B to get number
\item B combines Z with public value from A to get number
    \begin{itemize}
    \item and b/c of math chosen, both get same number
    \end{itemize}
\end{itemize}
\end{frame}

\begin{frame}{Diffie-Hellman key agreement (1)}
    \begin{itemize}
    \item math requirement:
        \begin{itemize}
        \item some $f$, so $f(f(X, Y), Z) = f(f(X, Z), Y)$
        \item (that's hard to invert, etc.)
        \end{itemize}
    choose X in advance and:
    \end{itemize}
\begin{tabular}{l|l}
A randomly chooses $Y$ & B randomly chooses $Z$ \\
A sends $f(X, Y)$ to B & B sends $f(X, Z)$ to A \\
A computes $f(f(X, Z), Y)$ & B computes $f(f(X, Y), Z)$ \\
\end{tabular}
\end{frame}



\section{putting it together: TLS}
\begin{frame}{why not just asymmetric encryption?}
    \begin{itemize}
    \item given public-key encryption + digital signatures\ldots
    \item why bother with the symmetric stuff?
    \vspace{.5cm}
    \item symmetric stuff much faster
    \item symmetric stuff much better at supporting larger messages
    \end{itemize}
\end{frame}

\begin{frame}{TLS: Transport Layer Security}
    \begin{itemize}
    \item the secure communication protocol for the web
    \item what makes accessing websites secure
    \item the "s" in \textit{https}
    \end{itemize}
\end{frame}



\subsection{handshake}
\usetikzlibrary{arrows.meta,shapes.callouts,positioning}

\begin{frame}{typical TLS handshake}
\begin{tikzpicture}
    \tikzset{
        >=Latex,
        comp box/.style={draw, thick, align=center, minimum width=1.5cm,minimum height=6.25cm},
        explain box/.style={draw=red,very thick, align=left},
        msg/.style={font=\small},
        cmd/.style={font=\small},
    }
        \node[comp box] (client) at (-6.5, 0) {client};
        \node[draw,cloud,line width=1pt,minimum width=4cm,minimum height=2cm,aspect=3,opacity=0.25] (network) at (0,0) {~~};
        \node[comp box] (server) at (6.5, 0) {server};
        \draw[very thick,->] ([yshift=-.5cm]client.north east) -- ([yshift=-.5cm]server.north west) 
            node[midway,below,msg] (client key share) {ClientHello,KeyShare};
        \draw[very thick,<-] ([yshift=-1.5cm]client.north east) -- ([yshift=-1.5cm]server.north west) 
            node[midway,below,msg] (server key share) {ServerHello,KeyShare};
        \draw[very thick,<-] ([yshift=-2.5cm]client.north east) -- ([yshift=-2.5cm]server.north west) 
            node[midway,below,msg] (certificate) {Certificate,CertificateVerify};
        \draw[very thick,<-] ([yshift=-3.5cm]client.north east) -- ([yshift=-3.5cm]server.north west) 
            node[midway,below,msg] (finished1) {Finished};
        \draw[very thick,->] ([yshift=-4.5cm]client.north east) -- ([yshift=-4.5cm]server.north west) 
            node[midway,below,msg] (finished2) {Finished};
    \begin{visibleenv}<2>
        \node[my callout2=client key share,anchor=north] at ([yshift=-1cm]client key share) {
            KeyShare = key parts for key exchange
        };
    \end{visibleenv}
    \begin{visibleenv}<3>
        \node[my callout2=certificate,anchor=north,align=left] at ([yshift=-1cm]certificate) {
            Certificate = certificate (``foo.com's public key is X'' + CA signature) \\
            \myemph<3>{CertificateVerify = Sign(foo.com's private key, server's key share)}
        };
    \end{visibleenv}
    \begin{visibleenv}<4-5>
        \node[my callout2=finished1,anchor=north,align=left] at ([yshift=-1cm]finished1) {
            MAC(\myemph<4>{key made from key shares}, Hash(everything so far))
        };
    \end{visibleenv}
    \begin{visibleenv}<6>
        \node[my callout2=finished2,anchor=north,align=left] at ([yshift=-1cm]finished2) {
            MAC(\myemph<4>{key made from key shares}, Hash(everything so far))
        };
    \end{visibleenv}
\end{tikzpicture}
\end{frame}


\subsection{after handshake}
\begin{frame}{TLS: after handshake}
    \begin{itemize}
    \item use key shares results to get \textbf{several} symmetric keys
        \begin{itemize}
        \item take hash(something + shared secret) to derive each key
        \end{itemize}
    \item separate keys for each direction (server $\rightarrow$ client and vice-versa)
    \item often separate keys for encryption and MAC
    \vspace{.5cm}
    \item later messages use encryption + MAC + nonces
    \end{itemize}
\end{frame}


\subsection{TLS properties}
\begin{frame}{things modern TLS usually does}
    \begin{itemize}
        \item (not all these properties provided by all TLS versions and modes)
    \vspace{.5cm}
    \item confidentiality/authenticity 
        \begin{itemize}
        \item server = one ID'd by certificate
        \item client = same throughout whole connection
        \end{itemize}
    \item forward secrecy
        \begin{itemize}
        \item can't decrypt old conversations (data for KeyShares is temporary)
        \end{itemize}
    \item fast
        \begin{itemize}
        \item most communication done with more efficient symmetric ciphers
        \item 1 set of messages back and forth to setup connection
        \end{itemize}
    \end{itemize}
\end{frame}




\subsection{summary}
\begin{frame}{network security summary (1)}
    \begin{itemize}
    \item communicating securely with math
        \begin{itemize}
        \item secret value (shared key, public key) that attacker can't have
        \item symmetric: shared keys used for (de)encryption + auth/verify; fast
        \item asymmetric: public key used by any for encrypt + verify; slower
        \item asymmetric: private key used by holder for decrypt + sign; slower
        \end{itemize}
    \item protocol attacks --- repurposing encrypt/signed/etc. messages
    \item certificates --- verifiable forwarded public keys 
    \item key agreement --- for generated shared-secret ``in public''
        \begin{itemize}
        \item publish key shares from private data
        \item combine private data with key share for shared secret
        \end{itemize}
    \end{itemize}
\end{frame}

\begin{frame}{network security summary (2)}
    \begin{itemize}
    \item TLS: combine all cryptography stuff to make ``secure channel''
    \item denial-of-service --- attacker just disrupts/overloads (not subtle)
    \item firewalls
    \end{itemize}
\end{frame}


\section{backup slides}
\begin{frame}{backup slides}
\end{frame}


\subsection{password hashing}
\begin{frame}{password hashing}
    \begin{itemize}
        \item cryptographic hash functions are good at requiring guesses to `reverse'
        \item idea: store cryptographic hash of password instead of password
            \begin{itemize}
            \item attacker who gets hash doesn't get password
            \item can still check entered password is correct
            \end{itemize}
            \vspace{.5cm}
        \item<2-> problem: with fast hash function, can try lots of guesses fast
        \item<3-> solution: special slow/resource-intensive cryptographic hash functions
            \begin{itemize}
            \item Argon2i
            \item scrypt
            \item PBKDF2
            \end{itemize}
    \end{itemize}
\end{frame}


\subsection{aside: key agreement to public key encrypt}
\begin{frame}{key agreement and asym. encryption}
    \begin{itemize}
    \item can construct public-key encryption from key agreeement
    \vspace{.5cm}
    \item private key: generated random value Y
    \item public key: key share generated from that Y
    \item<2-> PE(public key, message) =
        \begin{itemize}
        \item generate random value Z
        \item combine with public key to get shared secret
        \item use symmetric encryption + MAC using shared secret as keys
        \item output: (key share generated from Z) (sym. encrypted data) (mac tag)
        \end{itemize}
    \item<3-> PD(private key, message) =
        \begin{itemize}
        \item extract (key share generated from Z)
        \item combine with private key to get shared secret, \ldots
        \end{itemize}
    \end{itemize}
\end{frame}


\section{misc. security issues}

\subsection{denial of service}
\begin{frame}{denial of service}
    \begin{itemize}
    \item if you just want to inconvenience\ldots
    \item attacker just sends lots of stuff to my server
    \item my server becomes overloaded?
    \item my network becomes overloaded?
    \vspace{.5cm}
    \item but: doesn't this require a lot of work for attacker?
    \item exercise: why is this often not a big obstacle
    \end{itemize}
\end{frame}

\begin{frame}{denial of service: asymmetry}
    \begin{itemize}
    \item work for attacker > work for defender
    \item how much computation per message?
        \begin{itemize}
        \item complex search query?
        \item something that needs tons of memory?
        \item something that needs to read tons from disk?
        \end{itemize}
    \item how much sent back per message?
    \vspace{.5cm}
    \item resources for attacker > resources of defender
    \item how many machines can attacker use?
    \end{itemize}
\end{frame}

\begin{frame}{denial of service: reflection/amplification}
    \begin{itemize}
    \item instead of sending messages directly\ldots
        attacker can send messages ``from'' you to third-party
    \item third-party sends back replies that overwhelm network
    \item example: short DNS query with lots of things in response
    \vspace{.5cm}
    \item ``amplification'' =
        \begin{itemize}
        \item third-party inadvertantly turns small attack into big one
        \end{itemize}
    \end{itemize}    
\end{frame} 

    % FIXME: example of small request w/ big response

\subsubsection{amplification example}
% FIXME: https://blog.cloudflare.com/technical-details-behind-a-400gbps-ntp-amplification-ddos-attack/

\subsection{firewalls} % FIXME: complete
\begin{frame}{firewalls}
    \begin{itemize}
    \item don't want to expose network service to everyone?
    \item solutions:
        \begin{itemize}
        \item service picky about who it accepts connections from
        \item filters in OS on machine with services
        \item filters on router
        \end{itemize}
    \item later two called ``firewalls''
    \end{itemize}
\end{frame}

\begin{frame}{firewall rules?}
\end{frame}

V
\subsection{how certificates verified}
\begin{frame}{exercise}
    \begin{itemize}
    \item exercise: how should certificates verify identity?
    \end{itemize}
\end{frame}

\begin{frame}{how do certificate authorities verify}
    \begin{itemize}
        \item for web sites, set by CA/Browser Forum
        \item organization of:
            \begin{itemize}
            \item everyone who ships code with list of valid certificate authorities
                \begin{itemize}
                \item Apple, Google, Microsoft, Mozilla, Opera, Cisco, Qihoo 360, Brave, \ldots
                \end{itemize}
            \item certificate authorities
            \end{itemize}
        \item decide on rules (``baseline requirements'') for what CAs do
    \end{itemize}
\end{frame}

\begin{frame}{BR domain name identity validation}
    \begin{itemize}
        \item options involve CA choosing random value and:
        \vspace{.5cm}
        \item sending it to domain contact (with domain registrar) and receive response with it, or
        \item observing it placed in DNS or website or sent from server in other specific way
        \vspace{1cm}
        \item exercise: problems this doesn't deal with?
    \end{itemize}
\end{frame}

\begin{frame}{some other things public CAs do}
    \begin{itemize}
    \item keep their private keys in tamper-resistant hardware
    \item maintain publicly-accessible database of \textit{revoked} certificates
        \begin{itemize}
        \item some browsers check these
        \end{itemize}
    \item certificate transparency
        \begin{itemize}
            \item public logs of every certificate issued
            \item some browsers reject non-logged certificates
            \item so you can tell if bad certificate exists for your website
        \end{itemize}
    \item `CAA' records in the domain name system
        \begin{itemize}
        \item can indicate which CAs are allowed to issue certificates in DNS
        \item (but CAs apparently not required to use DNSSEC (certificate infrastructure for signing domain name records) when looking this up)
        \end{itemize}
    \end{itemize}
\end{frame}


\subsection{password hashing}
\begin{frame}{password hashing}
    \begin{itemize}
        \item cryptographic hash functions are good at requiring guesses to `reverse'
        \item idea: store cryptographic hash of password instead of password
            \begin{itemize}
            \item attacker who gets hash doesn't get password
            \item can still check entered password is correct
            \end{itemize}
            \vspace{.5cm}
        \item<2-> problem: with fast hash function, can try lots of guesses fast
        \item<3-> solution: special slow/resource-intensive cryptographic hash functions
            \begin{itemize}
            \item Argon2i
            \item scrypt
            \item PBKDF2
            \end{itemize}
    \end{itemize}
\end{frame}




\begin{frame}{cryptographic hash uses}
    \begin{itemize}
    \item find shorter `summary' to substitute for data
        \begin{itemize}
        \item what hashtables use them for, but\ldots
        \item we care that adversaries can't cause collisions!
        \end{itemize}
    \vspace{.5cm}
    \item<2-> deal with message limits in signatures/etc.
    \item<2-> password hashing --- but be careful! [next slide]
    \item<2-> constructing message authentication codes
        \begin{itemize}
        \item hash message + secret info (+ some other details)
        \end{itemize}
    \end{itemize}
\end{frame}




\begin{frame}{Matrix vulnerabilties}
    \begin{itemize}
    \item one example from \url{https://nebuchadnezzar-megolm.github.io/static/paper.pdf}
    \item system for confidential multi-user chat
    \vspace{.5cm}
    \item protocol + goals:
        \begin{itemize}
        \item each device (my phone, my desktop) has public key
        \item to talk to me, you verify one of my public keys
        \item to add devices, my client can forward my other devices' public keys
        \end{itemize}
    \item bug:
        \begin{itemize}
        \item when receiving new keys, clients did not check who they were forwarded from correctly
        \end{itemize}
    \end{itemize}
\end{frame}


\subsection{aside: key agreement to public key encrypt}
\begin{frame}{key agreement and asym. encryption}
    \begin{itemize}
    \item can construct public-key encryption from key agreeement
    \vspace{.5cm}
    \item private key: generated random value Y
    \item public key: key share generated from that Y
    \item<2-> PE(public key, message) =
        \begin{itemize}
        \item generate random value Z
        \item combine with public key to get shared secret
        \item use symmetric encryption + MAC using shared secret as keys
        \item output: (key share generated from Z) (sym. encrypted data) (mac tag)
        \end{itemize}
    \item<3-> PD(private key, message) =
        \begin{itemize}
        \item extract (key share generated from Z)
        \item combine with private key to get shared secret, \ldots
        \end{itemize}
    \end{itemize}
\end{frame}



\end{document}
