\begin{frame}{redundancy (1)}

{\tt
\begin{tabular}{l}
program: main.o extra.o \\
▶\hspace{2.5cm}clang -o program main.o extra.o \\
~ \\
extra.o: extra.c extra.h \\
▶\hspace{2.5cm}clang -o extra.o -c extra.c \\
main.o: main.c main.h extra.h \\
~ \\
▶\hspace{2.5cm}clang -o main.o -c main.c \\
\end{tabular}
}
\begin{itemize}
\item what if I want to run \texttt{clang} with \texttt{-Wall}?
\item what if I want to change to \texttt{gcc}?
\end{itemize}
\end{frame}

\begin{frame}{variables/macros (1)}

{\tt\small
\begin{tabular}{l}
\myemph{CC = gcc} \\
\myemph{CFLAGS = -Wall -pedantic -std=c11 -fsanitize=address} \\
\myemph{LDFLAGS = -Wall -pedantic -fsanitize=address} \\
\myemph{LDLIBS = -lm} \\
~ \\
program: main.o extra.o \\
▶\hspace{2cm}\$(CC) \$(LDFLAGS) -o program main.o extra.o \$(LDLIBS) \\
~ \\
extra.o: extra.c extra.h \\
▶\hspace{2cm}\$(CC) \$(CFLAGS) -o extra.o -c extra.c \\
~ \\
main.o: main.c main.h extra.h \\
▶\hspace{2cm}\$(CC) \$(CFLAGS) -o main.o -c main.c \\
\end{tabular}
}
\end{frame}

\begin{frame}{variables/macros (2)}

{\tt
\begin{tabular}{l}
CC = gcc \\
CFLAGS = -Wall \\
LDFLAGS = -Wall \\
LDLIBS = -lm \\
~ \\
program: main.o extra.o \\
▶\hspace{2cm}\$(CC) \$(LDFLAGS) -o \myemph{\$@} \myemph{\$\textasciicircum} \$(LDLIBS) \\
~ \\
extra.o: extra.c extra.h \\
▶\hspace{2cm}\$(CC) \$(CFLAGS) -o \myemph{\$@} -c \myemph{\$<} \\
~ \\
main.o: main.c main.h extra.h \\
▶\hspace{2cm}\$(CC) \$(CFLAGS) -o \myemph{\$@} -c \myemph{\$<} \\
\end{tabular}
}

{\small aside: \texttt{\$\textasciicircum} works on GNU make (usual on Linux), but not portable.}
\end{frame}

\begin{frame}{suffix rules}

{\tt
\begin{tabular}{l}
CC = gcc \\
CFLAGS = -Wall \\
LDFLAGS = -Wall \\
~ \\
program: main.o extra.o \\
▶\hspace{2cm}\$(CC) \$(LDFLAGS) -o {\$@} {\$\textasciicircum} \\
~ \\
\myemph{.c.o:} \\
▶\hspace{2cm}\$(CC) \$(CFLAGS) -o {\$@} -c {\$<} \\
~ \\
extra.o: extra.c extra.h \\
main.o: main.c main.h extra.h \\
\end{tabular}
} \\
{\small aside: \texttt{\$\textasciicircum} works on GNU make (usual on Linux), but not portable.}
\end{frame}


\begin{frame}{pattern rules}

{\tt
\begin{tabular}{l}
CC = gcc \\
CFLAGS = -Wall \\
LDFLAGS = -Wall \\
LDLIBS = -lm \\
~ \\
program: main.o extra.o \\
▶\hspace{2cm}\$(CC) \$(LDFLAGS) -o {\$@} {\$\textasciicircum} \$(LDLIBS) \\
~ \\
\myemph{\%.o: \%.c} \\
▶\hspace{2cm}\$(CC) \$(CFLAGS) -o {\$@} -c {\$<} \\
~ \\
extra.o: extra.c extra.h \\
main.o: main.c main.h extra.h \\
\end{tabular}
} \\
{\small aside: these rules work on GNU make (usual on Linux), but less portable than suffix rules.}
\end{frame}

\begin{frame}{built-in rules}
\begin{itemize}
\item `make' has the `make .o from .c' rule built-in already, so:
\end{itemize}
{\tt
\begin{tabular}{l}
CC = gcc \\
CFLAGS = -Wall \\
LDFLAGS = -Wall \\
LDLIBS = -lm \\
~ \\
program: main.o extra.o \\
▶\hspace{2cm}\$(CC) \$(LDFLAGS) -o {\$@} {\$\textasciicircum} \$(LDLIBS) \\
~ \\
extra.o: extra.c extra.h \\
main.o: main.c main.h extra.h \\
\end{tabular}
}
\begin{itemize}
\item (don't actually need to write supplied rule!)
\end{itemize}
\end{frame}


